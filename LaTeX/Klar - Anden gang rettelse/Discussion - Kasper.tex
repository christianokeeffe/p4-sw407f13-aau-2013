\section{Discussion}
\label{sec:discussion}
This section contains a debate about the goals of this project, and if they has been met. There is also a section which contains reflections about the project.

\subsection{Characteristic of SPLAD}
this section compares the design criteria set forward in section \ref{sec:DesignCriteria}, and checks if these criteria has actually been met.

"Simplicity" was set to be high for SPLAD. This, as explained in section \ref{sec:simplespladc} has been achieved. 

%%DENNE SUBSECTION ER KUN KLAR TIL AT BLIVE RETTET FØRSTE GANG
%%
%%
%%
\subsection{Simplicity of SPALD compared with C}
\label{sec:simplespladc}
This section will compare pieces of code written in SPLAD, with pieces of code written in C. The first code example, seen on listring \ref{lst:helloworldsplad}, is an example of a simple SPLAD program, which prints out "HelloWorld" to the LCD.

\begin{code}{helloworldsplad}{Hello world program in SPLAD}
\begin{lstlisting}
function pour return nothing using(int a, double b)
begin
	return nothing;
end

function RFIDFound return nothing using(int a, int b)
begin
	string message <-- "Hello World!";
	/* Print message on line 1 on LCD */
	call LCDPrint(message, 1);

	return nothing;
end
\end{lstlisting}
\end{code}

Line 1-4 of \ref{lst:helloworldsplad} contains the pour function, which is required by the SPLAD compiler, but is not relevant in this small example. The string "message" is declared on line 7, which is what will be printed to the LCD. On line 9, the function LCDPrint, which is provided by the SPLAD compiler is called, and the message is printed to the display. The message will be printed when an RFID-tag is found by the RFID-reader.

A hello world program written in Arudino C/C++, can be seen on listing \ref{lst:helloworldc}. It should be noted that this is an example provided by the Arduino IDE \citep{LCDtut}. Line 2 of this example includes the LCD Library. On line 5 the LCD display must be initialized with the pins on the Arduino. On line 9, the LCD display is setup, which means that the appropriate number of columns and rows is set depending on the particular model used. At last on line 11, the message is printed to the LCD.
 
\begin{code}{helloworldc}{Hello world program in Arduino C/C++}
\begin{lstlisting}
// include the library code:
#include <LiquidCrystal.h>

// initialize the library with the numbers of the interface pins
LiquidCrystal lcd(12, 11, 5, 4, 3, 2);

void setup() {
  // set up the LCD's number of columns and rows: 
  lcd.begin(16, 2);
  char message[ ] = "Hello World";
  // Print a message to the LCD.
  lcd.print(message);
}

void loop() {

}
\end{lstlisting}
\end{code}
The difference between the SPLAD program, and the Arduino program is clear: The SPLAD program is much more specialized with the target being drinks machine, which means that there is an LCD print function provided by default. This is not the case in a normal Arduino program, because the Arduino is aimed much more at general purpose. This means that users of the SPALD language need not think about which pins the LCD is connected to while writing programs. The Arduino does not provide a string-type, which means that strings are implemented by char arrays. In the SPLAD language there is a string type, which might seem more intuitive for novice programmers. The assignment is a bit different in SPLAD compared to C-notation. In SPLAD an assignment is denoted by '<--', which makes it completely clear what is assigned to what. In C, the assignment is denoted by '=', which might confuse novice programmers, because '=' generally is used to denote equality in for example mathematics. Also when using the '=', it is not really clear what is assigned to what.

\subsection{Other Criteria}

"Orthogonality" was set to be medium for our language. Since it is not possible to create classes or custom constructs in our language. But it is possible to call a function in an expression and to define functions. Therefore this criteria has also been met to an acceptable extend.

"Data types" was set to medium for our SPLAD. Because SPLAD only have five primitive types and two special types, SPLAD has relatively many types, but not too many types to confuse people, which makes it simple to keep an overview of them. On top of that, it is not possible to make classes. Therefore the criteria about data types is met.

"Syntax design" was set to be high for SPLAD. Since most of SPLAD has been created in the way to spawn a better understanding about different constructs, such as for-loops, and beginning and endings of block statements. Therefore this criteria has been achieved in SPLAD.

"Support for abstraction" was set to be medium for SPLAD, because our language is an abstraction of the Arduino programming language. Examples of the abstraction, which makes programming drinks machines based on the Arduino platform easier, can be seen in section \ref{sec:simplespladc}.

The criteria "Expressivity" was set to low for SPLAD. Since SPLAD has as focus to be as simple as possible, the expressivity is not in focus for SPLAD. 

"Type checking" was set to be high for SPLAD. To make sure that all type errors could be found is found when a program is compiled, a lot of work and tests have been made while creating the type checker. There were some problems, like the build-in functions that an Arduino program has acess to. This make it hard to make a type checker that will find all errors. But all types, expressions, parameters which are in SPLAD are type-checked  therefore this criteria is met. 