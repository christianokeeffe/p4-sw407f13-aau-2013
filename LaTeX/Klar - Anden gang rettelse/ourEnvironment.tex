\section{Environment for this Project}
The environment in this project is an Arduino-based drinks-machine. The basic idea behind this environment, is that the drinks-machine resides in a bar or to a party, where customers will buy RFID-tags, which contains information about a specific drink. These RFID-tags can be read by the drinks-machine by a RFID-reader. The machine will then automatically mix the drink on the RFID-tag.

\subsection{Solution in Bars}
Currently bars and clubs often have multiple bartenders who mixes the drinks and serves the customers. The bartender handles both receiving the order, mixing the drink and handles the payment of the drink. This process can be done more efficiently. If a bar had a drinks-machine like the one described above, the bar would require only one cashier instead of multiple bartenders. The cashier would handle selling and programming the RFID-tags. The customers themselves then places the RFID-tag on the RFID-reader on the drinks-machine, and the machine mixes the appropriate drink, and either decreases the count on the RFID-tag, or disables the tag, and display an appropriate message to the customer on the display of the machine. The RFID-tags are programmed by the cashier which encodes in the tag a drink id, and a drink count. This allows the customer to buy for example 10 cosmopolitans on the same tag. The machine will simply check if the drink-count on the tag is above zero, and display an error if it is not.
%nyt fra her
The system can also be use together with a ice-cream machine, juice machine or even a food dispenser in a restaurant to help with self-service, It can also be use in a cinema for regular customers to get popcorn, coke, or others forms of candy for the movies. There might even be a refill machine where the film is shown. There are many possibilities how to use such a system, but in this project a drink-mixer will be the focus.
%til her
\subsection{Programming an Arduino-based Drinks-mixer}
To make it easier to program the machine described above, it would be nice to have a programming language aimed specifically at programming drinks-machines. This would make it easier for programmers with little or no experience to program the machine, and thereby making it easier for e.g. bar-owners to program and install their own drinks-machines in their bar. Therefore it has been decided to make a programming language aimed at this problem. The SPLAD language: \textbf{S}imple \textbf{P}rogramming \textbf{L}anguage for \textbf{A}rduino \textbf{D}rink-mixer. The SPLAD will be described more thoroughly in section \ref{sec:problemstatement}