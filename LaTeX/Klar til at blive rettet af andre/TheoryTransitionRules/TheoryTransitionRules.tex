\documentclass[a4paper,11pt,fleqn,twoside,openright]{memoir} % Brug openright hvis chapters skal starte på højresider; openany, oneside

%%%% PACKAGES %%%%

%  Oversættelse og tegnsætning  %
\usepackage[utf8]{inputenc}					% Gør det muligt at bruge æ, ø og å i sine .tex-filer

\raggedbottom
\usepackage[all]{nowidow}
\usepackage[T1]{fontenc}  % Hjælper med orddeling ved æ, ø og å. Sætter fontene til at være ps-fonte, i stedet for bmp	
\usepackage{syntax}
\usepackage{everyshi}
\makeatletter
\let\totalpages\relax
\newcounter{mypage}
\EveryShipout{\stepcounter{mypage}}
\AtEndDocument{\clearpage
   \immediate\write\@auxout{%
    \string\gdef\string\totalpages{\themypage}}}
\makeatother
\usepackage{longtable}
\usepackage{lscape}
\usepackage[lined,boxed,linesnumbered]{algorithm2e}
\usepackage{latexsym}										% LaTeX symboler
\usepackage{xcolor,ragged2e,fix-cm}			% Justering af elementer
\usepackage{pdfpages} % Gør det muligt at inkludere pdf-dokumenter med kommandoen \includepdf[pages={x-y}]{fil.pdf}	
\usepackage{fixltx2e}					% Retter forskellige bugs i LaTeX-kernen
\usepackage{color}
\definecolor{darkgray}{rgb}{0.95,0.95,0.95}
\usepackage{listings}
\usepackage{tikz}
\usepackage{qtree}


 \lstloadlanguages{% Check Dokumentation for further languages ...
         %[Visual]Basic
         %Pascal
         C
	%[Sharp]C
         %C++
         %XML
         %HTML
        % Java
 }
\lstset{ %
inputencoding=utf8,
literate=%
{æ}{{\ae}}1
{å}{{\aa}}1
{ø}{{\o}}1
{Æ}{{\AE}}1
{Å}{{\AA}}1
{Ø}{{\O}}1,
language=[Sharp]C,                % the language of the code
basicstyle=\footnotesize\ttfamily,       % the size of the fonts that are used for the code
float = H,
xleftmargin = 10pt,
xrightmargin = 10pt,
rulecolor = \color{black},
numbers=left,                   % where to put the line-numbers
numberstyle=\footnotesize,      % the size of the fonts that are used for the line-numbers
stepnumber=1,                   % the step between two line-numbers. If it's 1, each line 
                                % will be numbered
numbersep=5pt,                  % how far the line-numbers are from the code
showspaces=false,               % show spaces adding particular underscores
showstringspaces=false,         % underline spaces within strings
showtabs=false,                 % show tabs within strings adding particular underscores
tabsize=2,                      % sets default tabsize to 2 spaces
captionpos=b,                   % sets the caption-position to bottom
breaklines=true,                % sets automatic line breaking
breakatwhitespace=false,        % sets if automatic breaks should only happen at whitespace
               % show the filename of files included with \lstinputlisting;
                                % also try caption instead of title
escapeinside={\%*}{*)},         % if you want to add a comment within your code
keywordstyle=\color[rgb]{0,0,1},
commentstyle=\color[rgb]{0.133,0.545,0.133},
stringstyle=\color[rgb]{0.627,0.126,0.941},
morekeywords={begin, function, end, nothing, bool, string, OR, AND, using, from, to, step, container, HIGH, LOW}
}

% add frame environment
\usepackage[%
    style=1,
    skipbelow=\topskip,
    skipabove=\topskip
]{mdframed}
\mdfsetup{%
    leftmargin=0pt,
    rightmargin=0pt,
    backgroundcolor=darkgray,
    middlelinecolor=black,
    roundcorner=10
}

% needed for \lstcapt
\def\ifempty#1{\def\temparg{#1}\ifx\temparg\empty}

% make new caption command for listings
\usepackage{caption}
\newcommand{\lstcapt}[2][]{%
    \ifempty{#1}%
        \captionof{lstlisting}{#2}%
    \else%
        \captionof{lstlisting}[#1]{#2}%
    \fi%
    \vspace{0.75\baselineskip}%
}

\usepackage{tabularx}
\usepackage{changepage}

																			
%  Figurer og tabeller floats %
\pdfoptionpdfminorversion=6	% Muliggør inkludering af pdf dokumenter, af version 1.6 og højere
\usepackage{graphicx} 		% Pakke til jpeg/png billeder
\usepackage{rotating}	

%  Matematiske formler og maskinkode 
\usepackage{amsmath,amssymb,stmaryrd} 	% Bedre matematik og ekstra fonte
\usepackage{textcomp}                 	% Adgang til tekstsymboler
\usepackage{mathtools}			% Udvidelse af amsmath-pakken.
\usepackage{siunitx}			% Flot og konsistent præsentation af tal og enheder med \SI{tal}{enhed}

%  Referencer, bibtex og url'er  %
\usepackage{url}	% Til at sætte urler op med. Virker sammen med ref
%\usepackage[danish]{varioref} % Giver flere bedre mulighed for at lave krydshenvisninger
\usepackage[english]{varioref} % Giver flere bedre mulighed for at lave krydshenvisninger
\usepackage{natbib}	% Litteraturliste med forfatter-år og nummerede referencer
\usepackage{xr}		% Referencer til eksternt dokument med \externaldocument{<NAVN>}
\usepackage{nomencl}	% Pakke til at danne nomenklaturliste
\makenomenclature		% Nomenklaturliste

%  Floats  %
\let\newfloat\relax 	% Memoir har allerede defineret denne, men det gør float pakken også
\usepackage{float}
%\usepackage[footnote,draft,danish,silent,nomargin]{fixme}	% Indsæt rettelser og lignende med \fixme{...} Med final i stedet for draft, udløses en error for hver fixme, der ikke er slettet, når rapporten bygges.
\usepackage[draft,silent]{fixme}

%%%% CUSTOM SETTINGS %%%%

%  Marginer  %
\setlrmarginsandblock{3.5cm}{2.5cm}{*}	% \setlrmarginsandblock{Indbinding}{Kant}{Ratio}
\setulmarginsandblock{2.5cm}{3.0cm}{*}	% \setulmarginsandblock{Top}{Bund}{Ratio}
\checkandfixthelayout 

%  Litteraturlisten  %
\bibpunct[,]{[}{]}{;}{a}{,}{,} 	% Definerer de 6 parametre ved Harvard henvisning (bl.a. parantestype og seperatortegn)
\bibliographystyle{bibtex/harvard}	% Udseende af litteraturlisten. Ligner dk-apali - mvh Klein

%  Indholdsfortegnelse  %
\setsecnumdepth{subsubsection}	% Dybden af nummerede overkrifter (part/chapter/section/subsection)
\maxsecnumdepth{subsubsection}	% Ændring af dokumentklassens grænse for nummereringsdybde
\settocdepth{subsection} 		% Dybden af indholdsfortegnelsen


%  Visuelle referencer  %
\usepackage[colorlinks, bookmarksnumbered, bookmarksdepth=4]{hyperref} % Giver mulighed for at ens referencer bliver til klikbare hyperlinks. .. [colorlinks]{..}
%\usepackage{memhfixc}
\hypersetup{pdfborder = 0 0 0}	% Fjerner ramme omkring links i fx indholsfotegnelsen og ved kildehenvisninger 
\hypersetup{			%	Opsætning af farvede hyperlinks
    colorlinks = false,
    linkcolor = black,
    anchorcolor = black,
    citecolor = black
}

\definecolor{gray}{gray}{0.80}					% Definerer farven grå

%  Kapiteludssende  %
\definecolor{numbercolor}{gray}{0.7}			% Definerer en farve til brug til kapiteludseende
\newif\ifchapternonum

\makechapterstyle{jenor}{			% Definerer kapiteludseende -->
  \renewcommand\printchaptername{}
  \renewcommand\printchapternum{}
  \renewcommand\printchapternonum{\chapternonumtrue}
  \renewcommand\chaptitlefont{\fontfamily{pbk}\fontseries{db}\fontshape{n}\fontsize{25}{35}\selectfont\raggedleft}
  \renewcommand\chapnumfont{\fontfamily{pbk}\fontseries{m}\fontshape{n}\fontsize{1in}{0in}\selectfont\color{numbercolor}}
  \renewcommand\printchaptertitle[1]{%
    \noindent
    \ifchapternonum
    \begin{tabularx}{\textwidth}{X}
    {\let\\\newline\chaptitlefont ##1\par} 
    \end{tabularx}
    \par\vskip-2.5mm\hrule
    \else
    \begin{tabularx}{\textwidth}{Xl}
    {\parbox[b]{\linewidth}{\chaptitlefont ##1}} & \raisebox{-15pt}{\chapnumfont \thechapter}
    \end{tabularx}
    \par\vskip2mm\hrule
    \fi
  }
}			% <--

\chapterstyle{jenor}	% Valg af kapiteludseende - dette kan udskiftes efter ønske
\usepackage{wrapfig}


%\renewcommand{\headrulewidth}{0.4pt}
%\renewcommand{\footrulewidth}{0.4pt}

\usepackage{enumitem}
% Sidehoved %

\makepagestyle{custom} % Definerer sidehoved og sidefod - kan modificeres efter ønske -->
\makepsmarks{custom}{																						
\def\chaptermark##1{\markboth{\itshape\thechapter. ##1}{}} % Henter kapitlet den pågældende side hører under med kommandoen \leftmark. \itshape gør teksten kursiv
\def\sectionmark##1{\markright{\thesection. ##1}{}}	% Henter afsnittet den pågældende side hører under med kommandoen \rightmark
} % Sidetallet skrives med kommandoen \thepage	
\makeevenhead{custom}{\leftmark}{P4, Aalborg University}{Group SW407F13} % Definerer lige siders sidehoved efter modellen \makeevenhead{Navn}{Venstre}{Center}{Højre}
\makeoddhead{custom}{Group SW407F13}{P4, Aalborg University}{\leftmark} % Definerer ulige siders sidehoved efter modellen \makeoddhead{Navn}{Venstre}{Center}{Højre}

\usepackage{lastpage}
\usepackage{ifthen}
\usepackage{intcalc}
\usepackage{nth}

\makeevenfoot{custom}{Side \thepage}{}{}	% Definerer lige siders sidefod efter modellen \makeevenfoot{Navn}{Venstre}{Center}{Højre}
\makeoddfoot{custom}{}{}{Side \thepage}% Definerer ulige siders sidefod efter modellen \makeoddfoot{Navn}{Venstre}{Center}{Højre}		
\makeheadrule{custom}{\textwidth}{0.5pt}	 % Tilføjer en streg under sidehovedets indhold
\makefootrule{custom}{\textwidth}{0.5pt}{1mm}	% Tilføjer en streg under sidefodens indhold

\copypagestyle{nychapter}{custom} % Følgende linier sørger for, at sidefoden bibeholdes på kapitlets første side
\makeoddhead{nychapter}{Group SW407F13, Aalborg Universitet}{P4}{\leftmark}
\makeevenhead{nychapter}{Group SW407F13, Aalborg Universitet}{P4}{\leftmark}
\makeheadrule{nychapter}{\textwidth}{0pt}
\aliaspagestyle{chapter}{nychapter}	% <--
\aliaspagestyle{cleared}{custom}

\pagestyle{custom}

%%%% CUSTOM COMMANDS %%%%

% Billede hack %
\newcommand{\figur}[4]{
		\begin{figure}[H] \centering
			\includegraphics[width=#1\textwidth]{billeder/#2}
			\caption{#3}\label{#4}
		\end{figure} 
}

% højrepil %
\newcommand{\ra}[0]{\rightarrow}
% epsilon %
\newcommand{\eps}{\varepsilon}

% Vektor hack %
\newcommand{\vektor}[3]{
			$\begin{pmatrix}
				#1 \\ #2 \\ #3
			\end{pmatrix}$
}

%kode
\newcommand{\kode}[3]{
\noindent
\begin{minipage}{\textwidth}
\begin{mdframed}
		\lstinputlisting{kode/#3}

\end{mdframed}\lstcapt{#1}\label{lst:#2}
\end{minipage}
}

\newenvironment{code}[2]
{\def\fooNoI{#1} \def\fooNoII{#2}\noindent \begin{minipage}{\textwidth}\begin{mdframed}}{\end{mdframed}\lstcapt{\fooNoII}\label{lst:\fooNoI}\end{minipage}}

% quotering
\newcommand{\gaas}[{1}]{``#1''}

% lilleitem
%\newenvironment{noindlist}
 %{\vspace{-5mm}\begin{list}{\labelitemi}{\leftmargin=1em \itemindent=0em }
%\addtolength{\itemsep}{-0.5\baselineskip}}
 %{\end{list}
%\vspace{-20em}}

\newenvironment{noindlist}
 {\vspace{-5mm}\begin{list}{\labelitemi}{\leftmargin=1em \itemindent=0em }
        \setlength{\topsep}{0pt}
        \setlength{\parskip}{0pt}
        \setlength{\partopsep}{0pt}
        \setlength{\parsep}{0pt}         
        \setlength{\itemsep}{0pt} }
 {\end{list}}

\newcommand{\doublesignaturestart}[2]{%
  \parbox{\textwidth}{
    \centering Aalborg \today\\
    \vspace{2cm}

    \parbox{7cm}{
      \centering
      \rule{6cm}{1pt}\\
       #1 
    }
    \hfill
    \parbox{7cm}{
      \centering
      \rule{6cm}{1pt}\\
      #2
    }
  }
}

\newcommand{\longtablesetting}[1]{
\endhead
\multicolumn{#1}{c}{\textit{Continued on the next page}} \\
\endfoot
\endlastfoot
}

\newcommand{\subsubsubsection}[1]{
\textbf{#1}
}

\newcommand{\doublesignature}[2]{%
  \parbox{\textwidth}{
\vspace{2cm}
    \parbox{7cm}{
      \centering
      \rule{6cm}{1pt}\\
       #1 
    }
    \hfill
    \parbox{7cm}{
      \centering
      \rule{6cm}{1pt}\\
      #2
    }
  }
}


%%%% ORDDELING %%%%

\hyphenation{hvad hvem hvor}
\newcolumntype{R}{>{\raggedright\arraybackslash}X}

%----------------SPROG------------------------
%----------------dk
%\usepackage[danish]{babel}							% Dansk sporg, f.eks. tabel, figur og kapitel
%\renewcommand{\algorithmcfname}{Algoritme}
%\renewcommand*{\lstlistingname}{Kodeudsnit}
%----------------en
\usepackage[english]{babel}
\begin{document}
\section{Transition Rules}
In this section some of the transition rules in SPLAD will be explained. The complete list of all the rules can be seen in appendix \ref{sec:transitionrules}.
In the following text we use the following names to represent different syntactic categories.
\begin{itemize}
\item $n \in \textbf{Num}$ - Numerals
\item $x \in \textbf{Var}$ - Variables 
\item $r \in \textbf{Arrays}$ - Array names
\item $a \in \mathbf{A_{exp}}$ - Arithmetic expression
\item $b \in \mathbf{B_{exp}}$ - Boolean expression
\item $S \in \textbf{Stm}$ - Statements
\item $R \in \textbf{Stm}$ - Root statements
\item $p \in \textbf{Pnames}$ - Procedure names
\item $D_V \in \textbf{DecV}$ - Variable declarations
\item $D_P \in \textbf{DecP}$ - Procedure declarations
\item $D_A \in \textbf{DecA}$ - Array declarations
\end{itemize}

\subsection{Abstract Syntax}
In order to describe the behavior of a program, one must first account for its syntax. We use the notion of abstract syntax, since it will allow us to describe the essential structure of the program. An abstract syntax is defined as follows. We assume a collection of syntactic categories and for each syntactic category we give a finite set of formation rules which defines how the the inhabitants of the category can be build \citep{HHTree}.
Here the abstract syntax of SPLAD can be seen. 
\begin{align*}
R ::= \; & D_P \; D_A \; D_V \; | R_1 \; R_2 \; \\
S::= \; & x := a \; | \; r[a_1] := a_2 \; | \; S_1; \; S_2 \; | \; \text{if} \; b \; \text{begin} \; S \; \text{end} \; | \; \text{if} \; b \; \text{begin} \; S_1 \; \text{end} \; \text{else begin} \; S_2 \; \text{end} \\
~ & \text{while} \; b \; \text{begin} \; S \; \text{end} \; | \; \text{from} \; x := a_1 \; \text{to} \; a_2 \; \text{step} \; a_3 \; \text{begin} \; S \; \text{end} \; | \; \text{call} \; p(\vec{x}) \; | \; D_V \; | \; D_A \; \\
~ & | \; \text{switch}(a) \; \text{begin} \; \text{case } a_1: \; S_1 \; \text{break}; \; \dots \; \text{case } a_k: \; S_k \; \text{break}; \; \text{default}: \; S \; \text{break} \; \text{end} \\
a::= \; & n \; | \; x \; | \; a_1 + a_2 \; | \; a_1 - a_2 \; | \; a_1 * a_2 \; | \; a_1 / a_2 \; | \; (a) \; | \; r[a_i]\\
b::= \; &a_1 = a_2 \; | \; a_1 > a_2 \; | \; a_1 < a_2 \; | \; \neg b \; | \; b_1 \; \wedge \; b_2 \; | \; b_1 \; \vee \; b_2 \; | \; (b)\\
D_V::= \; & \text{var} \; x := a \;|\; \eps\\
D_P::= \; & \; \text{func} \; p(\vec{x}) \; \text{is} \; \text{begin} \; S \; \text{end}  \;|\; \eps\\
D_A::= \; & \text{array} \; r[a_1]  \;|\; \eps\\
\end{align*}

\subsection{Transition Systems}


\subsection{Big-step-semantic and small-step-semantic}
\fxfatal{Forklar hvad det er, og at vi bruger big-step.}

\subsection{Environment-Store Model}
In our project we use the \textit{environment-store model} to represent how a variable is bound to a storage cell (called a \textit{location}), in the computer, and that the value of the variable is the content of the bound location. All the possible locations are denoted by \textbf{Loc} and a single location as $l \in \textbf{Loc}$. We assume all locations are integer, and therefore $\textbf{Loc} = \mathbb{N}$. Since all locations are integers we can define a function to find the next location: $\textbf{Loc} \rightarrow \textbf{Loc}$, where $l = l + 1$. 

We define the set of stores to be the mappings from locations to values $\textbf{Sto } = \textbf{ Loc } \rightharpoonup \mathbb{Z}$, where $sto$ is an single element in $\textbf{Sto}$.

An variable-environment is like a symbol table containing each variable and store the variables address. The store then describe which values that is on each address.

The following names represent the different environments. 
\begin{itemize}
\item $env_V \in Env_V$ - Variable environment
\item $env_A \in Env_A$ - Array environment
\item $env_P \in Env_P$ - Procedure environment
\end{itemize}

\subsection{Statements}
The transition rules for the statements are on the form: $env_V, env_P \vdash \langle S, sto \rangle \rightarrow sto'$. The transition system are defined by: $(\Gamma_{\mathbf{Stm}}, \rightarrow, \Tau_{\mathbf{Stm}})$ and the configurations are defined by $\Gamma_{\mathbf{Stm}} = \textbf{Stm} \times \textbf{Sto} \cup \textbf{Sto}$. The end configurations are defined by $\Tau_{\mathbf{Stm}} = \textbf{Sto}$.

On table \ref{tab:VarAssign} the assignment rule for variables can be seen. The rule states, that if $a$ evaluates to $v$, and $x$ points to the location $l$, then $v$ is stored in the $l$.

\begin{longtable}{l l}
\longtablesetting{2}
[VAR-ASS] & $env_V, env_P \vdash \langle x <-- a, sto \rangle \rightarrow sto[l \mapsto v]$ \\
~ & ~ \\
~ & \indent\indent where $env_V, sto \vdash a \rightarrow_a v$ \\
~ & \indent\indent and $env_V \; x = l$ \\
~ & ~ \\
\caption{Transition rule for variable assignment.}
\label{tab:VarAssign}
\end{longtable}

\subsection{Arithmetic Expressions}
The transition rules for the arithmetic expressions are on the form: $env_V, sto \vdash a \rightarrow_a v$. The transition system are defined by: $(\Gamma_{\mathbf{Aexp}}, \rightarrow_a, \Tau_{\mathbf{Aexp}})$ and the configurations are defined by $\Gamma_{\mathbf{Aexp}} = \textbf{Aexp} \cup \mathbb{Z}$. The end configurations are defined by $\Tau_{\mathbf{Aexp}} = \mathbb{Z}$.

The transition rule for multiplication in SPLAD can be seen on table \ref{tab:MultExp}. The rule states, that if $a_1$ evaluates to $v_1$ and $a_2$ evaluates to $v_2$, using any of the rules from the arithmetic expressions, then $a_1 \cdot a_2$ evaluates to $v$ where $v = v_1 \cdot v_2$.

\begin{longtable}{l l}
\longtablesetting{2}
[MULT] & $\dfrac{env_V, sto \vdash a_1 \rightarrow_a v_1 \; \; \; env_V, sto \vdash a_2 \rightarrow_a v_2}{env_V, sto \vdash a_1 \cdot a_2 \rightarrow_a v}$ \\
~ & ~ \\
~ & \indent\indent where $v = v_1 \cdot v_2$ \\
~ & ~ \\
\caption{The transition rule for the arithmetic multiplication expression.}
\label{tab:MultExp}
\end{longtable}

\subsection{Boolean Expression}
The transition rules for boolean expressions are on the form: $env_V, sto \vdash b \rightarrow_b t$. The transition system are defined by: $(\Gamma_{\mathbf{Bexp}}, \rightarrow_b, \Tau_{\mathbf{Bexp}})$ and the configurations are defined by $\Gamma_{\mathbf{Bexp}} = \textbf{Bexp} \cup \{tt, ff\}$. The end configurations are defined by $\Tau_{\mathbf{Bexp}} = \{tt, ff\}$.

The transition rule for logical-or in SPLAD can be seen on table \ref{tab:OrExp}. The rules have two parts: [OR-TRUE] and [OR-FALSE]. The [OR-TRUE] rule states that either $b_1$ or $b_2$ evaluates to \textit{TRUE}, using any of the rules from the boolean expressions, then the expression $b_1 \; \text{OR} \; b_2$ evaluates to \textit{TRUE}. [OR-FALSE] states that if both $b_1$ and $b_2$ evaluate to \textit{FALSE} then the expression $b_1 \; \text{OR } \; b_2$ evaluates to \textit{FALSE}.

\begin{longtable}{l l}
\longtablesetting{2}
[OR-TRUE] & $\dfrac{env_V, sto \vdash b_i \rightarrow_b \text{TRUE}}{env_V, sto \vdash b_1 \vee b_2 \rightarrow_b \text{TRUE}}$ \\
~ & ~ \\
~ & \indent\indent where $i \in {1,2}$ \\
~ & ~ \\

[OR-FALSE] & $\dfrac{env_V, sto \vdash b_1 \rightarrow_b \text{FALSE} \; \; \; env_V, sto \vdash b_2 \rightarrow_b \text{FALSE}}{env_V, sto \vdash b_1 \vee b_2 \rightarrow_b \text{FALSE}}$ \\
~ & ~ \\
\caption{Transition rule for the boolean expression logical-or.}
\label{tab:OrExp}
\end{longtable}

\subsection{Variable Declaration}
The transition rules for the variable declarations are on the form: $\langle D_V, env_V, sto \rangle \rightarrow_{DV} (env_V', sto')$. The transition system are defined by: $\Gamma_{\mathbf{ErkV}}, \rightarrow_{DV}, \Tau_{\mathbf{ErkV}}$ and the configurations are defined by $\Gamma_{DV} = (\textbf{ErkV} \times \textbf{EnvV} \times \textbf{Sto}) \cup (\textbf{EnvV} \times \textbf{Sto})$ and $\Tau_{DV} = (\textbf{EnvV} \times \textbf{Sto})$. The end configurations are defined by $\Tau_{\mathbf{ErkV}} = \textbf{EnvV} \times \textbf{Sto}$.

On table \ref{tab:VarDec} the transition rules for variable declaration can be seen. It is done by binding $l$ to the next available location and binding $x$ to this location. The function $new$ is then used to point at the next available location. Then $env_V$ is updated to include the new variable, while the store remains unchanged.

\begin{longtable}{l l}
\longtablesetting{2}
[VAR-DEC] & $\dfrac{\langle D_V, env_V'', sto[l \mapsto v] \rangle \rightarrow_{DV} (env_V', sto'}{\text{var} \; x <-- a; D_V, env_V, sto \rangle \rightarrow_{DV} (env_V', sto')}$ \\
~ & ~ \\
~ & \indent\indent where $env_V, sto \vdash a \rightarrow_a v$ \\
~ & \indent\indent and $l = env_V \; \text{next}$ \\
~ & \indent\indent and $env_V'' = env_V[x \mapsto l][\text{next} \mapsto \text{new} \; l]$ \\
~ & ~ \\
\caption{Transition rules for the variable declarations.}
\label{tab:VarDec}
\end{longtable}

\subsection{Procedure Declaration}
The transition rules for the procedure declarations are on the form: $env_V \vdash \langle D_P, env_P \rangle \rightarrow_{DP} env_P'$. The transition system are defined by: $(\Gamma_{\mathbf{ErkP}}, \rightarrow_{DP}, \Tau_{\mathbf{ErkP}})$ and the configurations are defined by $\Gamma_{DP} = (\textbf{ErkP} \times \textbf{EnvP}) \cup \textbf{EnvP}$ and $\Tau_{DP} = \textbf{EnvP}$. The end configurations are defined by $\Tau_{\mathbf{ErkP}} = \textbf{EnvP}$.

On table \ref{tab:ProcDec} the transitions rules for the procedure declaration with none or multiple parameters can be seen. The rule states that the new procedure is stored in the procedure environment along with the statement, formal parameters, and procedure- and variable-bindings from the time of declaration.

\begin{longtable}{l l}
\longtablesetting{2}

[PROC-PARA-DEC] & $\dfrac{env_V \vdash \langle D_P, env_P[p \mapsto(S, \vec{x}, env_V, env_P)] \rangle \rightarrow_{DP} \; env_P'}{env_V \vdash \langle \text{func} \; p(\text{var} \; \vec{x}) \; \text{is begin} \; S \; \text{end}, \; env_P \rangle \rightarrow_{DP} env_P'}$ \\
~ & ~ \\
\caption{Transition rules for the procedure declarations.}
\label{tab:ProcDec}
\end{longtable}

\subsection{Array Declaration}
The transition rules for the variable declarations are on the form: $\langle D_A, env_V, sto \rangle \rightarrow_{DA} (env_V', sto')$. The transition system are defined by: $(\Gamma_{\mathbf{ErkA}}, \rightarrow_{DA}, \Tau_{\mathbf{ErkA}})$ and the configurations are defined by $\Gamma_{DA} = (\textbf{ErkA} \times \textbf{EnvV} \times \textbf{Sto}) \cup (\textbf{EnvV} \times \textbf{Sto})$ and $\Tau_{DA} = \textbf{EnvV} \times \textbf{Sto}$. The end configurations are defined by $\Tau_{\mathbf{ErkA}} = \textbf{EnvV} \times \textbf{Sto}$.

On table \ref{tab:ArrayDec} the transitions rules for the array declaration can be seen. The rule states that a number of location equal to the length of the array plus one is allocated, and the pointer is set to point at the next available location. The length of the array are then stored in the first location of the array. 

\begin{longtable}{l l}
\longtablesetting{2}
[ARRAY-DEC] & $\dfrac{\langle D_A, env_V[r \mapsto l, \text{next} \mapsto l + v + 1],  sto[l \mapsto v] \rangle \rightarrow_{DA} (env_V', sto')}{\langle \text{array} \; r[a_1], env_V, sto \rangle \rightarrow_{DA} (env_V', sto')}$ \\
~ & ~ \\
~ & \indent\indent where $env_V, sto \vdash a_1 \rightarrow_a v$ \\
~ & \indent\indent and $l = env_V \text{next}$ \\
~ & \indent\indent and $v > 0$ \\
~ & ~ \\
\caption{Transition rules for the array declarations.}
\label{tab:ArrayDec}
\end{longtable}
\end{document}
