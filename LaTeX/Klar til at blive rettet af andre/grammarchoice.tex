\section{Choice of grammar}
The programmer, using this projects language, could be a hobby programmer, who would want to program a custom drink machine, but does not possess a high level of education in programming. Therefore it was decided that the grammar should have a high level of readability because this will ensure that it is easier for the person to read and understand their program - also useful if the code has to be edited later on. This on the other hand can decrease the level of write-ability because it has to be written in a specific way and will need to contain some extra words or symbols to mimic a language closer to human language rather than a computer language.


The method to assign a value to a variable is by typing "$variable$ <-- $value to assign$" this approach have been chosen, instead of the more commonly used "$=$" symbol, because a person not accustomed to programming might confuse which side of the "$=$" is assigned to the other. Thus by using the arrow, it is more clearly indicated that the value is assigned to the variable, and therefore ensuring readability - especially for the hobby programmer.

To get a more symmetrical structure in the code the functions must always return something, but it can return the value "nothing". This will ensure a better understanding and readability of the code when the programmer can see what it returns, even if no value is parsed. To indicate that $return$ is the last thing that will be executed in a function, the $return$ must always be at the end of the function. To indicate that a program is called "call $functionname$" must be written.
Words are used instead of symbols, when suitable, to improve the understanding of the program(compared to most other programming languages).
"begin" and "end" are used to indicate a block (eg. an "if" statement). To combine logical operators the words "AND" and "OR" are used. The ";" symbol is used to improve readability by making it easier to see when the end of a line has been reached.

It would be appropriate to design a grammar that is a subset of $LL(1)$ grammars. This is based on the idea that it easier to implement a parser for $LL(1)$ grammars by hand compared to $LR$ grammars. This approach means it would be possible to both implement a parser by hand or use some of the already existing tools. This way both approaches are possible which are a suited solution for the project because it allows the project group to later go back and make the parser by hand instead of using a tool if so desired.

If the purpose was to create an efficient compiler it would be more appropriate to design the grammar as a subset of the $LALR$ grammar class. A parser for $LALR$ is balanced between power and efficiency which makes it more desirable than $LL$ and other $LR$ grammars, see section \ref{sec:grammar} for more on the grammars. $LR$ parsers can be made by hand but it is much more difficult than the $LL$ parsers.