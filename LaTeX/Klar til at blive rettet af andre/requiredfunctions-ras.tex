\section{Requirements to the Programmer}
There are certain requirements to the programmer, when he is writing in SPLAD. He must implement the functions:
\begin{itemize}
	\item {\color[rgb]{0,0,1}function} pour {\color[rgb]{0,0,1}return} {\color[rgb]{0.545,0.133,0.133}nothing} {\color[rgb]{0,0,1}using}({\color[rgb]{0.545,0.133,0.133}container} cont, {\color[rgb]{0.545,0.133,0.133}int} Amount)
	\item {\color[rgb]{0,0,1}function} RFIDFound {\color[rgb]{0,0,1}return} {\color[rgb]{0.545,0.133,0.133}nothing} {\color[rgb]{0,0,1}using}({\color[rgb]{0.545,0.133,0.133}int} DrinkID, {\color[rgb]{0.545,0.133,0.133}int} Amount)
\end{itemize}
The programmer must implement these two functions, so the SPLAD framework can call the functions. The pour function is called, when a ingredient must be poured. The programmer should then handle how to pour from his container.
The RFIDFound function is called when a RFID-tag is found, and the programmer should then, with the content of the RFID-tag as parameters, handle what to do with the tag.

\section{Functions provided by SPLAD}
There are certain functions provided by SPLAD. The functions are:
\begin{itemize}
	\item {\color[rgb]{0,0,1}function} LCDPrint {\color[rgb]{0,0,1}return} {\color[rgb]{0.545,0.133,0.133}nothing} {\color[rgb]{0,0,1}using}({\color[rgb]{0.545,0.133,0.133}string} StringToPrint, {\color[rgb]{0.545,0.133,0.133}int} Line)
	\item {\color[rgb]{0,0,1}function} LCDClear {\color[rgb]{0,0,1}return} {\color[rgb]{0.545,0.133,0.133}nothing} {\color[rgb]{0,0,1}using}()
	\item {\color[rgb]{0,0,1}function} RFIDWrite {\color[rgb]{0,0,1}return} {\color[rgb]{0.545,0.133,0.133}bool} {\color[rgb]{0,0,1}using}({\color[rgb]{0.545,0.133,0.133}int} DrinkID, {\color[rgb]{0.545,0.133,0.133}int} Amount)
	\item {\color[rgb]{0,0,1}function} pourDrink {\color[rgb]{0,0,1}return} {\color[rgb]{0.545,0.133,0.133}nothing} {\color[rgb]{0,0,1}using}({\color[rgb]{0.545,0.133,0.133}drink} DrinkToPour)
\end{itemize}
The LCDPrint function is provided to the programmer, so they can print to the LCD. The LCDClear will clear both lines on the LCD. The RFIDWrite function is provided, so the programmer can write a DrinkID and an Amount to the RFIDTag. When a drink should be poured, the programmer can call pourDrink, which will call the pour function provided by the programmer for each ingredient in the drink.