I dette kapitel præsenteres konklusion på projektet, hvor der vil set nærmere på om problemformuleringen og delproblemerne er blevet løst.
Problemformuleringen for dette projekt lyder således: 

\textbf{\gaas{Hvordan kan et IT-system laves, der kan hjælpe brugeren med at finde de bedste tilbud på de varer der er på en given indkøbsliste, hjælpe brugeren med at koordinere indkøb med andre brugere, samt finde tilbud af interesse ud fra brugerens indkøbsvaner?}} 

Problemformuleringen kan ses i afsnit \ref{sec:problem}. For at løse problemformuleringen, skal produktet hjælpe en bruger med at koordinere indkøbene, finde tilbud på ønskede varer, samt forslå tilbud baseret på tidligere indkøb.

Hjemmesiden kan hjælpe med at finde tilbud ved, at bruge den API gruppen har fået midlertidigt adgang til. Hjemmesiden finder tilbud ved hjælp af en søgefunktion der tager søgeordet og finder de tilbud, hvor søgeordet står først, sidst eller alene i ordene der er i tilbudsnavnet og ikke midt i et ord. Et eksempel kan være, hvis man søger efter \gaas{øl} vil søgningen ikke tage pølser med som et resultat. Hjemmesiden giver også brugerne mulighed for selv at bestemme hvilke butikker de ønsker der skal vises tilbud fra. Altså har brugerne mulighed for at rimelig relevante tilbud baseret på de informationer der er blevet angivet.

Hjemmesiden hjælper også brugeren med at holde styr på deres indkøbslister, ved at lade dem oprette lister med et specificeret navn. Brugeren kan derefter tilføje varerne på den ønskede indkøbsliste som huskes af hjemmesiden. Indkøbslisterne kan også sorteres på forskellige måder, som er: Pris, navn, butik, mængdepris eller kategori. Dog kan systemet ikke automatisk finde kategorien, den enkelte varer skal være i, derfor er det lagt over til brugeren at specificere en varens kategori ud fra nogle givne valgmuligheder. Videre arbejde kunne være at udvikle en funktion, så systemet selv kunne finde ud af, hvilken kategori en vare tilhører.

Indkøbslisterne lader brugerne markere hvilke varer, der er købt og hvilke, der ikke er blevet købt. Samtidig noteres det også hvem, der har købt en vare og derved kan brugerne altid se, hvem der har købt de forskellige varer.
Hjemmesiden lader nemlig også brugerne dele listen med andre brugere, som de ønsker skal have adgang til listen. Dette samt noteringen er med til at give brugerne mulighed for at kunne overskue og koordinere indkøbene mellem hinanden.

Herved er problemerne i problemformuleringen løst.