\chapter{Systemudvikling}

%RET START
Til dette projekts arbejdsmetode og rapportopbygning er der blevet brugt viden fra \gaas{Systemudviklingskurset}. Selve rapportens opbygning består af formalia, analysedokumentet, designdokumentet, udvikling, test, diskussion og konklusion. Analyse- og designdokumentets struktur og indhold er udledt fra OOAD(Objekt orienteret analyse og design) bogen \citep{ooaogd}. Delene er taget med, da de fungerer som god dokumentation og vejledning for den arbejdsproces som hører til afsnittene og som netop er blevet brugt i dette projekt. Til udviklingen af systemet har \gaas{Systemudvikling} hjulpet til at bestemme hvorledes systemets dele skulle prioriteres, jævnfør kvalitetsmålene sektion \ref{sec:kvalitetsmaal}.
%RET SLUT

\section{Generel fremgangsmetode for systemudvikling}
I analysen laves et rigt billede, der bruges til at udvikle et klassediagram som beskriver relationerne mellem objekter i problemområdet. Objekterne beskrives ved hjælp af klasser. Ud fra klassediagrammet, blev der set nærmere på de forskellige klassers tilstandsdiagrammer, som er med til at beskrive deres tilstande og vilkårene for tilstandsskift. For at forstå, hvad der påvirker klasserne bliver der derefter kigget på hændelser i problemområdet der kan påvirke klasserne. Sammenhængen mellem hændelser og klasser bliver beskrevet i en hændelsestabel.

Anvendelsesområdet omhandler aktører, brugsmønstre, brugergrænseflader og tekniske platforme. Aktørerne beskrives ved hjælp af aktør-specifikationer og ud fra dem, findes tilhørende brugsmønstre, der beskriver aktørernes brug af systemet. Brugsmønstrene bliver brugt til at udvikle funktioner, som systemet skal indeholde. Brugergrænseflade beskriver systemets struktur samt de forskellige vinduers funktioner. Tekniske platforme bruges til at beskrive hvilke tekniske omstændigheder systemet vil optræde i.

Analysedelen af rapporten lægger basen for designet af systemet. I designdokumentet er der et afsnit baseret på aktiviteten \gaas{design komponent}, hvori systemets model-, funktions- og brugergrænsefladekomponenter beskrives således, at selve systemet kan udvikles baseret på disse komponenter. Udover komponentbeskrivelsen bliver arkitekturen, designsproget, basisprogrammel og udstyr også beskrevet, dette er med til at sætte rammerne for hvordan og hvilken platform systemet skal fungere på.
 
Denne fremgangsmetode er blevet anvendt i dette projekt, fordi den egner sig til at involvere brugerne i udviklingen. En fordel ved denne fremgangsmetode er, at den er logisk og giver en god model til at dokumentere det arbejde, der er blevet lavet. En anden fordel ved denne model er, at et problem ikke findes gennem læsematerialer, men i stedet ved at observere og analysere selve problemområdet samt de aktører som befinder sig i dette område. En ulempe er dog at det kræver en del arbejdstid i at undersøge og observere virkeligheden.

\section{Iterativ metode vs. vandfalds-metode}
Vandfalds-modellen går ud på at arbejde i en lige linje og kun lave én del færdig ad gangen. En af fordelene ved denne tilgang er, at det er nemt at holde overblik over, hvor i projektet man befinder sig og hvad der videre skal laves. Dette betyder at analysen bliver fuldendt inden designbeslutninger bliver foretaget og derved skulle alle nødvendige informationer være fundet. En ulempe ved denne model er at gruppen ikke kan arbejde ordenligt videre, før det nuværende trin er færdiggjort, hvilket betyder at egentlig programmering kommer langt inde i forløbet. Samtidig er det i den iterative nemmere at gå tilbage og ændre tidligere dele af projektet, hvis en aktivitet har skabt et ændret syn på en tidligere aktivitet. Dette betyder at vandfalds-modellen ikke egner sig til tidlig brugertestning.

Den iterative model handler om at dele projektarbejdet op i dele hvori der bliver arbejdet med alle områder på en gang, som var det et mindre projekt ad gangen. I hver af iterationerne er der så lagt mere fokus på en del af den overordnede proces.
Dette betyder, at hver del af projektarbejdet bliver gennemgået flere gange og udarbejdet hele tiden. Samtidig kan det været muligt at arbejde på flere dele af projektet samtidig. Som et eksempel på hvordan den iterative metode har forløbet sig i dette projekt, kan der på figur \ref{fig:iterationrettelser} ses hvad der er blevet rettet fra iteration til iteration i forhold til systemet.

\figur{1}{Iterationsdiagram.png}{Rettelser lavet til systemet under de forskellige iterationer.}{fig:iterationrettelser}

Denne model egner sig til tidlig brugertestning og derved involvere brugerne tidligt i systemudviklingen. En ulempe for denne model er at tidligere arbejde måske bliver irrelevant eller skal laves om, idet viderearbejde har ledt til en anden konklusion end den oprindelige.

Den iterative model har været egnet i dette projekt idet brugerne involveres så tidligt i projektet som muligt. Dette er valgt baseret på kvalitetsmålet testbart\ref{sec:kvalitetsmaal}, der er sat til at være et meget vigtig kvalitetsmål. Derfor bliver systemet udviklet i så tæt sammenarbejde med brugerne som muligt under projektet. Dette ville vandfalds-modellen ikke egne sig til, idet testning af programmet først kommer efter udviklingen af programmet og tidligere trin er færdige.
Et andet argument for at bruge den iterative model frem for vandfalds-modellen, er at den egner sig til systemudvikling i den forstand, at der bliver arbejdet på programmet samtidig med rapporten, hvorfra det er muligt at tjekke om designplanerne egentlig er realiserbare. Den iterative model egner sig også ved, at der hele tiden arbejdes med de forskellige afsnit, og at de derved rettes således de passer til de nye antagelser og beslutninger.
Til at hjælpe med at administrere prioriteringen af arbejdet i de forskellige iterationer, er der blevet lavet en graf som illustrerer hvorledes fokusset er bleven prioriteret i projektet. Illustrationen kan ses på figur \ref{fig:illugraf}. Ud af den vertikale aske ses prioriteringen og ud af den horisontale aske ses iterationerne. Grafen er baseret på princippet i Rational Unified Process (RUP) \citep{RUP}, hvor man deler projektet op i forskellige faser og derefter prioriter forskellige delprocesser alt efter deres relevans. RUP er en model til software udviklingsproces og bruges til administrere hele udviklingsprojektet.

\figur{1}{graf.png}{Illustration af hvilket område der har været i fokus, i forhold til de iterationer projektgruppen har vært igennem.}{fig:illugraf}

\section{Vigtigste erfaringer opnået gennem projektarbejdet}
I dette projekt er der blevet arbejdet objekt-orienteret med udgangspunkt i \citep{ooaogd}. Projektet er blevet lavet i fem iterationer, hvor der i hver iteration er blevet set på alle dele af projektet dog med varierende fokus.
Gennem projektarbejdet har projektgruppen opnået erfaring i at arbejde iterativt og objekt-orienteret med hensyn til udviklingsrapporten. Gruppen har opdaget værdien i, hele tiden at gå tilbage og revurdere tidligere arbejde. Samtidig har gruppen fundet ud af at det er vigtigt ikke at have ejerforhold til kode- og rapportdele. Ved at tænke hele projektet som et stort fælles stykke arbejde, er det nemmere at droppe dele, for derved at højne kvaliteten. Dette sikres også ved at alle gruppemedlemmer på skift har arbejdet med alle dele af projektet. Derudover har gruppen fået vigtige kompetencer fra semesterets kurser, heriblandt at designe og bruge klassediagrammer, at lave og bruge forskellige brugertests og at skrive og bruge diverse algoritmer. 