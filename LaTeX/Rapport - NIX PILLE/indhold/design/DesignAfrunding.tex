\section{Afrunding af designdokumentet}
I dette kapitel blev der set på, hvordan problemet skal løses og hvilke komponenter der udgør problemløsningen. Til at starte med, blev der defineret nogle kvalitetsmål som blev brugt til at styre udviklingen. Kvalitetsmålene sikrer at designet bliver tjekket efter og løsningen bliver designet efter de rigtige prioriteter. I dette projekt blev kvalitetsmålene \gaas{Brugbart}, \gaas{Flytbart} og \gaas{Forståeligt} prioriteret som de vigtigste kvalitetsmål. Dette valg er baseret på ideen om systemet skal være nemt at anvende og kan bruges uden introduktion.

Der blev set nærmere på den tekniske platform, altså hvilke tekniske enheder problemløsningen skal udvikles til og fungere på. Her faldt valget på mobile enheder der har adgang til internettet og der har en touch screen til rådighed, som den primære platform. En sekundær platform kan være en mere stationær platform, eksempelvis en computer.
Grunden til dette valg er, at systemet skal bruges på farten, men brugerne skal også have mulighed for at redigere indkøbslisterne på en hurtigere måde.

Ud fra arbejdet i analysedokumentet, kvalitetsmålene og arbejdet omkring den tekniske platform, blev systemets komponenter og deres struktur designet. Under udviklingen blev opfattelsen af problemområdet også afgrænset, idet dele af det blev fundet irrelevante at arbejde med i forhold til dette system.
Med hjælp fra de relevante klassers tilstandsdiagrammer og hændelsestabellen er der udviklet et klassediagram, som indeholder systemets forskellige komponenter og deres forbindelser til hinanden.