I denne sektion vil standarder for arkitektur-design,  og systemets struktur i forhold til komponenter og processer blive beskrevet.

\subsection{Komponentarkitektur}
Arkitekturen i problemløsningen skal indeholde tre komponenter. Det er klient, server og API. Server og API indeholder yderligere tre komponenter hver: Grænseflade, funktioner og model. Klienten er derfor afhængig af serveren, da klienten ikke vil være i stand til at bruge løsningen, hvis serveren ikke er tilgængelig.

For at skabe et overblik vil der først optræde en skabelon af hvordan arkitekturen overordnet vil komme til at se ud. Herefter vil skabelonen blive udbygget i mere detaljerede dele. Den omtalte skabelon kan ses på figur \ref{fig:designkomponent}.

\subsection{Procesarkitektur}
Til at få overblik er der lavet et fordelingsdiagram for at forklare de forskellige processer der er i problemløsningen. 

\figur{1}{DesignFordelingsdiagram.png}{Fordelingsdiagram over de forskellige processor for en brugers system.}{fig:fordellingsdiagram}

Fordelingsdiagrammet, som ses på figur \ref{fig:fordellingsdiagram}, viser processerne for den enkelte brugers system. Det betyder at databasen tilgås af alle aktive brugere der henter tilbud, indkøbslister eller tilføjer/fjerner ting fra indkøbslisterne. Desuden tilgås API'en også når der hentes tilbud.

Hvis to brugere har den samme indkøbsliste, og redigerer i den samme indkøbsliste på samme tid, kan det give konflikter. Konflikten ligger i, hvis den ene bruger fjerner en vare mens den anden fortager ændringer i listen. Dette resulterer i, at begge indkøbslister er forældede og først bliver opdaterede, når brugerne genåbner samme side. Derfor bliver problemløsningen nødt til at håndtere disse processer, således der ikke opstår nogle konflikter og derved sørger for at listerne bliver redigeret korrekt.

\subsection{Standarder}
Designet af problemløsningen skal være mobiloptimeret, således den kan bruges effektivt på mobile enheder. Problemløsningen skulle også kunne bruges på en stationær enhed, derfor skal problemløsningens design være af en form, sådan at den kan bruges på begge platforme.
