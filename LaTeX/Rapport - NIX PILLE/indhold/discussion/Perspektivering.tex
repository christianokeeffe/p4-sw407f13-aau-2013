\section{Perspektivering}
Hensigten med den fremstillede løsning er at hjælpe brugerne med at fremstille og synkronisere indkøbslister brugerne imellem. Baseret på denne forestilling kan systemet være årsag til at brugere går væk fra brugen af papir til indkøbslister.
Synkroniseringen skulle gerne betyde at brugerne undgår miskommunikation med deres partnere og derved undgår fejlindkøb. Dette betyder at systemet på et niveau sikrer forbrugerne mod disse fejlindkøb og derved forebygger mod unødvendige omkostninger ved brugernes handlen.

Systemets tilbudsfunktion betyder, at brugerne ikke længere er afhængig af tilbudsaviserne for at finde tilbudsvarer. Dette betyder at brugerne kan afvise reklamer og stadig have gode muligheder for, at finde de relevante tilbud de søger. Dette resulterer i en mindre affaldsmængde og dermed mindre spild af papir.

Denne egenskab kan både være en fordel og ulempe for selskaberne der ejer dagligvarebutikkerne. Det kan være en fordel i den forstand, at selskaberne kan ønske at stoppe med at producere deres tilbudsavis i papirformat og flytte til det elektroniske medie. Årsagen til denne flytning kan skyldes ønsket om at nedskære omkostningerne på tilbudsaviserne.
Det kan være en ulempe for selskaberne, idet brugerne af systemet kan miste interessen for deres reklamer, da de nemmere kan finde de tilbud de søger via systemet.
Systemet gør det også nemmere for brugerne at få en oversigt over alle de tilbud som er registreret i tilbudsdatabasen. Dette er en fordel for forbrugerne, idet det forhøjer konkurrencen.

I det videre arbejde med systemet er det attraktivt at lave en app til tablets og smartphones. Begrundelsen for denne beslutning er, at det mindsker brugerens internetforbindelse ved at lagre informationerne, som f.eks. typografiarik, på den enhed, hvor app'en er installeret på. På denne måde undgås det at databaserne bliver kaldt så mange gange som en hjemmeside bliver nødt til.
Det videre arbejde vil også ligge i at udarbejde nogle af de funktioner der er blev forslået af testpersonerne men blev anset som værende ikke vigtige i forhold til andre mere essentielle funktioner. Et af forslagene var bl.a. at systemet udnytter brugeres lokation til at forslå tilbud. En anden funktion der blev diskuteret, var at systemet bør sortere varerne på indkøbslisterne i den rækkefølge, som man møder dem gennem butikken. Denne funktion er også relateret til brugerens lokation, idet systemet bliver nød til at kende til, hvilken butik brugeren skal til at handle i.
