\section{Afrunding af problemanalyse}
I dette kapitel er der blevet set på, hvad problemområdet består af og hvilke problematikker, der skal behandles for at kunderne, de potentielle brugere, kan anvende systemet som en løsning på de opstillede problematikker. Der kommer en generalisering af problemområdet, for at få et overblik over, hvad anvendelsesområdet skal være i stand til at kunne.

Et værktøj der blev brugt til at danne overblik over problemområdet, er det rige billede som ses på figur \ref{fig:RigtBillede}, som bruges til at illustrere problemområdet og de konklifter, der kan findes i problemområdet. Gennem analysen af problemområdet, er det blevet gjort klart, at problemet består af tre delproblemer, der dækker over de forskellige konflikter, som kan opstå i problemområdet. Disse tre delproblemer er henholdsvis deling af indkøbslisterne, synkronisering mellem flere indkøbslister og til sidst håndteringen af tilbud i forhold til indkøbslister.

Et andet værktøj, som også blev brugt til at kortlægge problemområdet er et klassediagram, som indholder de forskellige klasser og deres relationer til hinanden, som de kan observeres i den virkelige verden, samt deres hændelser. Til klassediagrammet følger en forklaring af strukturen i klassediagrammet. Selve klasserne er også beskrevet samt deres tilstandsdiagrammer, der illustrerer hvorledes de forskellige klasser skifter tilstande.

Klassediagrammet bliver derefter brugt i arbejdet med anvendelsesområdet. Til arbejdet med anvendelsesområdet bliver der set nærmere på, hvilke redskaber der er egnet til udviklingen af systemet. Der bliver også defineret to aktører, samt brugsmønstre der fortæller om, hvordan disse aktører vil bruge systemet i den givne problemstilling. Ud fra brugsmønstrene er der blevet udviklet funktioner, der understøtter hændelsestabellen.

Til sidst er der blevet udviklet en problemformulering, som afspejler essensen af dette projekt. Det videre arbejde vil ligge i, at få designet de komponenter, som vil udgøre systemet, se på kvalitetsmålene for udviklingen og se nærmere på hvilken platform systemet skal fungere på.