I dette kapitel vil problemet blive analyseret. Problemområdet og anvendelsesområdet vil blive beskrevet. Ved hjælp af klassediagrammer og tilstandsdiagrammer vil problemområdet og anvendelsesområdet blive undersøgt. Analysen fungerer som udgangspunkt for designkapitlet, se kapitel \ref{chap:design}.

\section{Opgaven}
I denne sektion vil formålet med projektet blive beskrevet. Systemdefinition og omgivelserne vil desuden blive beskrevet. Med omgivelser menes problemområdet og anvendelsesområdet for dette projekt.

\subsection{Formål}
Det kan være vanskeligt og tidskrævende at overskue alle de ugentlige tilbud i dagligvarebutikker. Samtidig kan det også være svært at koordinere fælles indkøb mellem eksempelvis bofæller. Derudover er det praktisk hele tiden at have sin indkøbsliste på sig, da der kan opstå situationer, hvor man kunne bruge sin indkøbsliste. Derfor vil det være praktisk med et mobilbaseret indkøbslistesystem, der både kan finde de bedste tilbud ud fra en liste med indkøb, administrere indkøbslister, og som kan synkronisere mellem flere forskellige enheder. 

\subsection{Systemdefinition}
Systemet skal bruges til at administrere og finde tilbud på varer på en indkøbsliste. Systemet skal primært være et indkøbslisteprogram med tilbudsfunktion, som viser de bedste tilbud på de varer der står på indkøbslisten, men sekundært, skal det kunne foreslå tilbud, ud fra brugernes indkøbsvaner.  Systemet skal baseres på en mobil-optimeret hjemmeside, således at løsningen vil virke på gængse mobiloperativsystemer såsom Android, iOS og Windows Phone. Der skal i øvrigt tages hensyn til, at dele af systemet skal kunne betjenes med én hånd, da nogle områder af systemet typisk vil blive brugt i et supermarked, mens man handler. 

\begin{itemize}
\item \textbf{Betingelser:} Systemet skal fungere på forskellige platforme, og i forskellige hverdags-situationer.
\item \textbf{Anvendelsesområde:} Personer, par eller en gruppe, som køber ind, og er interesserede i at handle billigt.
\item \textbf{Teknologi:} Systemet skal være baseret på mobilenheder, således det kan bruges mens man handler.
\item \textbf{Objekter:} Kunder, indkøbslister, varer, tilbud og butikker.
\item \textbf{Funktionalitet:} Støtte til koordinering, administrering og tilbudssøgning.
\item \textbf{Filosofi:} Administrativt værktøj og indkøbsplanlægger.
\end{itemize}

\subsection{Omgivelser}
I dette afsnit gives et overblik over omgivelserne som repræsenterer problemområdet, dette vil blive beskrevet i detaljer i det næste afsnit \gaas{Problemområde}. Herunder på figur \ref{fig:RigtBillede} ses det rige billede, der er blevet lavet for at skabe overblik over problemområdet. 
\figur{1}{RigtBillede.png}{Rigt billede over problemområdet.}{fig:RigtBillede}

\subsubsection{Problemområde}
Problemområdet består af flere forskellige delproblemer, der vil blive set nærmere på. Et af delproblemerne er, at det kan være svært for par eller grupper, at koordinere indkøb indbyrdes, hvilket kan resultere i fejl i de foretagne indkøb. Dette delproblem bunder i kommunikationsproblemer mellem kunderne, som kan være resultatet af flere ting, bl.a. hvis de kommer til at tale forbi hinanden eller at en af kunderne misforstår den anden.


Et andet delproblem er, at det kan være besværligt for kunden, at få et overblik over alle de gældende tilbud på det tidspunkt, hvor indkøbslisten bliver skrevet. Med de mange forskellige tilbudsaviser kan det være tidskrævende og svært at skulle gennemse dem alle sammen, samt holde styr på de forskellige tilbud.

Det tredje delproblem er, at få delt indkøbslisten blandt de relevante personer. Hovedårsagen til dette ligger i papirformatet og det ekstra arbejde der ligger i at få fremstillet en kopi af indkøbslisten, dog findes der også løsninger, der tilbyder et elektronisk format af indkøbslister.

Disse tre delproblemer udgør tilsammen den del af problemområdet, som er i fokus i dette projekt.


\subsubsection{Anvendelsesområde}
Systemet skal lade kunder oprette sig som brugere. Systemet skal administrere brugertilgangen og derved tillade adgang til den pågældende kundes indkøbslister og programmets øvrige funktioner.
Gennem systemet skal kunderne kunne oprette indkøbslister, som kunden kan associere med andre kunder og derved dele den givne indkøbsliste indbyrdes.

Der skal være mulighed for at kunderne kan redigere deres tilknyttede indkøbslisters indhold, som systemet derefter skal gemme. Brugergrænsefladen skal bestå af de nødvendige elementer som gør det muligt for kunderne at tilgå de nødvendige funktioner de skal bruge i deres handlen.

Systemet skal også analysere kundernes indkøbslister og ud fra hvilke varer der er blevet købt, foreslå tilbud der kunne falde i kundens interesse. Til håndtering af tilbud skal systemet hente de informationer, der er nødvendige for, at kunden kan søge i tilbudene og tilføre de ønskede tilbud til den pågældende kundes indkøbsliste.