\section{Problemformulering}
\label{sec:problem}
I denne sektion vil hovedproblemet blive fremsat, hvorefter det vil blive delt op i delproblemer, som problemløsningen vil beskrive en løsning på, med udgangspunkt i analysedokumentet.
Ud fra analysedokumentet kommer hovedproblemet:

\textbf{Hvordan kan et IT-system laves, der kan hjælpe brugeren med at finde de bedste tilbud på de varer der er på en given indkøbsliste, hjælpe brugeren med at koordinere indkøb med andre brugere, samt finde tilbud af interesse ud fra brugerens indkøbsvaner?}

Det kan ses ud fra beskrivelsen af problemområdet og anvendelsesområdet, hvor IT-systemet skal kunne hente og registrere informationer. Informationen skal systemet kunne anvende på brugernes indkøbsliste, således at den finder de billigste varer i de udvalgte butikker.
Delproblemerne er lavet for at hjælpe med at finde løsningen til hovedproblemet.
\begin{itemize}

\item\textbf{Hvordan kan de forskellige klasser og hændelser håndteres i et IT-system?}

IT-systemet skal kunne arbejde med forskellige klasser og hændelser, som er fundet i analysedokumentet. Det er nødvendigt at finde ud af, hvordan systemet kan håndtere klasser og deres hændelser. 

\item\textbf{Hvad kan gøres for at opgaven med at at finde tilbud bliver nemmere og hurtigere?}

Løsningen skal gøre det nemmere og hurtigere at lave en indkøbsliste, ud fra tilbud den pågældende uge.

\item\textbf{Hvordan kan en brugers indkøb analyseres, således at der kan foreslås nye tilbud til brugeren?}

Løsningen skal ud fra brugernes indkøbsvaner foreslå relevante varer, som er på tilbud den pågældende uge.

\item\textbf{Hvordan kan det gøres nemmere for en bruger at koordinere sine indkøb med en eller flere personer?}

Løsningen skal gøre det nemmere for en bruger at koordinere sine indkøb med andre, så brugeren ikke risikerer at ende op med dubletter af en given vare.

\end{itemize}