\section{Anden test}
\label{sec:andentest}
I dette afsnit beskrives alt omkring anden test og hvad konsekvenserne af denne test blev. 

\subsection{Testpersonerne}
Testen er blevet sendt ud til vores seks informanter. Af disse har fire svaret. Alle testpersonerne er unge mellem 18 og 25 år, har en smartphone, og med en begrænset indtægt. Desuden blev testen også gennemgået af projektgruppen.

\subsection{Systemets tilstand}
Testen blev udført da systemets kernefunktioner lige var blevet udviklet. Disse kernefunktioner var: Oprettelse af profil, oprettelse af flere indkøbslister, søgning af tilbud, tilføje og fjerne varer til og fra indkøbslisterne. 

\subsection{Kvalitetsmål i fokus}
I denne test forsøgtes kvalitetsmålene \gaas{brugbar} og \gaas{forståeligt} testet. Årsagen til at netop disse kvalitetsmål blev sat i fokus i den første brugertest var, at det ønskedes testet om informanterne ville benytte systemet til hverdag, og om systemet i det hele taget var forståeligt for dem. 

\subsection{Testmetoden}
I denne test er testpersonerne blevet bedt om at teste visse funktioner på hjemmesiden på en smartphone, og derefter svare på et spørgeskema. De funktioner som skulle testes i denne omgang var: oprettelse af en bruger, søgning af tilbud, oprettelse af flere indkøbslister og at tilføje/slette/redigere disse lister. Hjemmesiden var på daværende tidspunkt en prototype, hvor lige præcis de beskrevne funktioner virkede.
Selve spørgeskemaet indeholdte åbne spørgsmål, hvor der ingen svarmuligheder var, men hvor testpersonen forventedes at beskrive deres observationer og meninger. Spørgsmålene spurgte til testpersonernes meninger omkring funktionaliteten og designet af de funktioner der blev testet. Spørgeskemaet kan findes i bilag \ref{chap:foesteskema}.
Denne testmetode blev valgt, da det skulle undersøges hvorvidt de to funktioner indeholdte hvad testpersonerne mente burde være i sådanne funktioner, om der manglede noget, eller om der kunne laves forbedringer i forhold til kvalitetsmålene i tabel \ref{tab:kriterier}, og deres prioriteringer. Der blev valgt et spørgeskema med åbne spørgsmål, da det dækker behovet for at få den information, der er relevant for det videre forløb i projektet. Der blev valgt et spørgeskema frem for et interview i denne omgang, fordi det sparer tid. 

Begrænsningerne ved denne testform, er at spørgsmålene er prædefinerede, således er det ikke muligt at stille opfølgende spørgsmål, hvis sådanne skulle opstå. Derudover kan man risikere en lav svarprocent ved udsendelse af spørgeskemaer.

\subsection{Testpersonernes svar}
Ud fra testpersonernes svar blev der udledt følgende punkter, som enten skulle rettes eller vurderes om de skulle implementeres.
\begin{itemize}

\item \textbf{Selve søgningen i søge-funktionen:}
Testpersonerne undrer sig over irrelevante søgeresultater. Der bliver desuden foreslået at søgefunktionen skal vise forslag til, hvad der søges på, mens der skrives. 

%Der skulle bedre søgekriterier til så søgningen kun søgte på overskriften og ikke selve beskrivelsen af et tilbud. Desuden vil det undersøges om det er muligt at få søgefunktionen til selv også at søge på synonymer. Derudover ønskedes det at søge-funktionen selv gav forslag til hvad der søges på imens en bruger skriver.

\item \textbf{Markering i indkøbslisten:}
Kun få af testpersonerne kunne finde ud af at markere varerne i indkøbslisten. 


\item \textbf{Systemets hastighed:}
En af testpersonerne påpegede at systemet virkede langsomt over mobilforbindelsen.


\item \textbf{Sorteringen i søgefunktionen:}
Testpersonerne ønskede, at der kunne sorteres bedre for eksempel efter billigste pris eller efter bedste mængdepris.

%dette vil blive set nærmere på hvorvidt det kan løses.

%Dette skal redesignes så det er tydeligere hvordan dette gøres.



\item \textbf{Fejl:}
Under testene blev der fundet tre fejl. Den ene fejl var, at siden gik ned da der forsøgtes at skrive et navn til en indkøbsliste med en apostrof. Derudover fik en af testpersonerne en fejlbesked. Den sidste fejl der blev fundet, var at en af testpersonerne oplevede at tilbudene ikke forsvandt fra søgeresultatet, efter han havde tilføjet dem til en indkøbsliste. Fejlrettelserne blev prioriteret således:  Fejlbesked, indkøbsliste med apostrof, fjerne tilbud fra søgeresultat



\item \textbf{Indkøbslisten:}
Den ene testperson foreslog at billederne fra tilbudene også blev tilføjet på indkøbslisten. Desuden ønskede en af testpersonerne at man kunne tilføje flere varer til listen på én gang. Der var også en enkelt testperson der havde problemer med at tilføje vare direkte til indkøbslisten.


\item \textbf{Navigation:}
Flere testpersoner ønskede en tilbage-knap udover den allerede eksisterende \gaas{forside} knap.


\item \textbf{Prisvisning:}
En af testpersonerne ønskede en bedre visning af priserne.
% og der vurderes på om det skal redesignes.


\item \textbf{Design:}
Nogle af testpersonerne fandt designet af programmet kedeligt. 

%Dette vil blive vurderet om der skal foretages en komplet redesign eller om der skal fortages nogle små ændringer.

%det kan overvejes om det skal designes på en anden måde.


\item \textbf{Enhed:}
Der var en testperson der ønskede at man kunne tilføje vare til indkøbslisten med andre enheder end stk. for eksempel gram. 
%Der vurderes om det skal gøres muligt selv at vælge eller angive enhed.

\item \textbf{Touch screen:}
En af testpersonerne klagede over at da touch screens er yderst følsomme skete det nogle gange at når han/hun ville rulle ned i listen blev noget valgt i stedet. 

%Dette kan muligvis løses ved at man skal holde knappen eller fingeren nede noget længere for at vælge et tilbud.
\end{itemize}

\subsection{Konsekvenser af anden test}
Ud af de 16 forslag til forbedringer er 10 blevet rettet eller integreret, mens de sidste punkter blev fundet irrelevante eller for tidskrævende til at blive rettet.


Problemet med, at der kom mange irrelevante resultater af en søgning er blevet rettet ved at der er blevet lavet en filterret søgning. Desuden kommer der nu forslag ved søgningerne. Ved søgefunktionen er der også blevet lavet en funktion, der giver mulighed for selv at vælge hvad resultatlisten skal sorteres efter. Alle småfejl som for eksempel stavefejl i systemet er også blevet rettet.
Programmet er desuden blevet optimeret, så der skal laves færre eksterne kald og systemet bliver derved hurtigere.


Ved indkøbslisten er der blevet lavet en infoboks, hvor brugeren kan finde hjælp, hvis det er nødvendigt. Stedet man tilføjer vare til indkøbslisten er blevet flyttet, så det nu er over indkøbslisten. Desuden er pilen i listen blevet lavet om til en knap, der åbner et pop-op vindue med indstillingerne for indkøbslisten. På indkøbslisten er der nu et billede af hvert enkelt tilbud på indkøbslisten, og der er blevet rettet flere småfejl. Der er også lavet en SQL-validering der håndterer specialtegn i navne på indkøbslisterne.
Problemstillingen med at forsøgspersonerne synes der manglede en tilbage-knap er blevet løst, så der nu både er en tilbage-knap og en forside-knap. Der er også lavet en bedre prisvisning. Det sidste der er blevet rettet er at programmet i sit hele er blevet optimeret, så hastigheden af hjemmesiden er blevet hurtigere.


De punkter der ikke er blevet rettet er:
Automatisk søgning efter synonymer, at kunne holde fingeren eller musen inde for at markere eller vælge en vare, multitilføjelse, muligheden for andre enheder end stk., og fjernelse af tilbud efter tilføjelse blev alle fravalgt grundet mangel på tid og ressourcer. Designet blev desuden ikke rettet da dette skal gennemtænkes mere.

\subsection{Afrunding anden test}
De vigtigste problemstillinger der blev observeret i anden test var:
\begin{itemize}
\item Mange søgeresultater var irrelevante.
\item Testpersonerne kunne ikke gennemskue, hvordan de skulle markere varerne.
\end{itemize}
De vigtigste problemstillinger der blev rettet i anden test var:
\begin{itemize}
\item Søgninger søger nu kun i titlen, så de fleste af de irrelevante resultater bliver frasorteret.
\item Markeringen blev ændret, så det nu var mere tydeligt hvordan den fungerede.
\end{itemize}

Disse problemstillinger blev prioriteret som de vigtigste, idet de blev fundet essentielle i forbedringen af brugeroplevelsen samt opfyldelsen af de opsatte kriterier som kan ses på tabel \ref{tab:kriterier}.