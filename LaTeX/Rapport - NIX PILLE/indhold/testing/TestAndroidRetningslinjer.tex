\subsection{Femte test - Androids retningslinjer}
\label{sec:femtetest}
I dette afsnit vil de retningslinjer, som Google anbefaler man følger for at lave en Android applikation i særklasse, blive undersøgt.
Informationerne er taget fra Android Developers \citep{AndRet}.

Vejledningen er blevet delt op i tre emner:
\subsubsection*{Enchant me}
Som udgangspunkt, det brugeren ser og føler når de bruger app'en.

Udseende er ikke alt, små effekter der får app'en til at føles mere interaktiv og giver en følelse af respons, giver meget til oplevelsen. Det at flytte eller interagere med objekter frem for knapper kan også være med til at gøre oplevelsen meget sjovere. Ved at kunne ændre på udseende, gives brugeren en følelse af at app'en tilhører dem og at det er dem der bestemmer. Hvis app'en også husker på tidligere valg, så giver det følelsen af at app'en lærer en at kende, hvilket i sidste ende gør brugen af app'en nemmere og hurtigere.

\subsubsection*{Simplify my life}
Hvor nemt og tilfredsstillende det er at bruge app'en.

Simpelt og stilrent er essensen af et godt design. Jo mindre det er nødvendigt at skrive, des bedre, og hvis det kan erstattes af et billede, er det endnu bedre. Det er også meget af tiden bedre bare at komme med et bud, på hvad brugeren vil have, frem for at spørge dem om for meget. For meget information på én gang kan være med til at forvirre, frem for at hjælpe, så vis kun hvad der er nødvendigt, så som hvor i systemet brugerne befinder sig, så de får følelsen af at vide, hvad de laver. Hvis brugeren har brugt tid på at lave noget, så er det vigtigt at det kan gemmes, så de kan tage det med sig videre på f.eks. andre enheder eller hvis app'en skal opgraderes. Hold konsistens over hele app'en, hvis noget ligner hinanden så skal de også gøre det samme ellers skabes der forvirring. Brugere skal kun afbrydes fra det de laver hvis det, der gives besked om er meget vigtig, da brugere godt kan lide at holde fokus på det, de laver.
\subsubsection*{Make me amazing}
Generelt evnen til at få brugeren til at føle sig klog og smart.

Brugere elsker når de finder ud af noget selv, så det bør prøves at implementere funktioner og udseende fra Android apps, derved får de en positiv overraskelse når funktionerne fungerer, som de regner med. Brugere vil ikke føle sig dum når de bruger app'en, så hvis noget går galt, vær sikker på at give dem simple instrukser og spar dem for lange forklaringer. Meget svære eller avancerede opgaver skal helst deles så meget op som de kan for at gøre det nemmere for brugerne. Det er vigtigt at give en form for respons selvom det måske ikke er mere end et svagt lys bag et ikon. Gør det nemt for begyndere at føle sig som eksperter, som f.eks ved at sammensætte nogle filtre og indstillinger til én knap i et fotografiapparat, så de nemt kan tilgås. Vær opmærksom på at fremhæve vigtige knapper og funktioner, så som knappen til at tage et billede med er mere fremhævet end knapperne til at indstille effekter.

\subsubsection{Sammenligning}
I dette afsnit er der vist en tabel, som sammenligner dette projekts produkt og Androids retningslinjer, som kan ses på tabel \ref{tab:RetninglinjerAndriod}. Systemet havde på dette tidspunkt alle de endelige funktionaliteter, med undtagelse af muligheden for at sortere efter varekategori. Det er gjort for at give et indtryk af, hvordan produktet passer på de retningslinjer som det anbefales at Android produkter lever op til for at give den bedste oplevelse. 

\begin{table}[H]
\centering
    \begin{tabular}{|l|c|c|c|}
    \cline{2-4}
\multicolumn{1}{l|}{}       & \rotatebox{90}{Taget højde for} 	& \rotatebox{90}{Kunne gøres bedre}& \rotatebox{90}{Ikke taget højde for~}	\\ 
\hline
Interaktivt design	       	& ~    			  	& x					& ~						\\ \hline
Objekter frem for knapper  	& ~					& ~					& x						\\ \hline
Personificering				& ~					& x					& ~						\\ \hline
Lær mig at kende			& x					& ~					& ~						\\ \hline
Hold det simpelt			& ~					& x					& ~						\\ \hline
Billeder frem for ord		& ~					& ~					& x						\\ \hline
Vælg for mig,(jeg bestemmer)& x					& ~					& ~						\\ \hline
Vis kun det nødvendige		& x					& ~					& ~						\\ \hline
Hvor er jeg på siden?		& x					& ~					& ~						\\ \hline
Gem mine ting				& x					& ~					& ~						\\ \hline
Konsistens					& ~					& x					& ~						\\ \hline
Notifikation, kun vigtigt	& x					& ~					& ~						\\ \hline
Genbrug kendte effekter		& ~					& ~					& x						\\ \hline
Simpel fejlfinding			& ~					& x					& ~						\\ \hline
Enkle opgaver				& x					& ~					& ~						\\ \hline
Gør det tunge arbejde		& x					& ~					& ~						\\ \hline
Fremhæv vigtige ting		& x					& ~					& ~						\\ \hline  

\end{tabular}

\caption{Tabel over sammenligningen mellem produktet og Andriods retningslinjer.}
\label{tab:RetninglinjerAndriod}

\end{table}
%ret herfra
\subsubsection{Konsekvenser for systemet}
\label{sec:andretkon}
Ud fra undersøgelsen af retningslinjerne, er der lavet nogle ændringer ved systemet, for at optimere det til den bedst mulige brugeroplevelse. For at forbedre punktet med at holde det simpelt er det forsøgt at simplificere nogle af de forskellige dele af hjemmesiden, specielt metoden til at søge. Her var der før to søgeknapper: en til filtreret søgning og en anden til fuld søgning. Den filtrerede søgning forsøgte at gøre resultaterne mere relevante i forhold til søgeordet og varer fra fravalgt butikker blev ikke vist. Den fulde søgning tog alt hvad der kom frem. Det blev vurderet, at man altid ville have relevante søgeresultater, så derfor blev det for simpelhedens skyld valgt at fjerne knappen til fuld søgning.