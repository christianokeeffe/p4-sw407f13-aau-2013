\subsection{Sjette test - Retningslinjer for Apples apps}
I dette afsnit vil der blive fokuseret på Apples sæt af retningslinjer og hvorledes dette projekts produkt følger nogle af deres principper og retningslinjer \citep{AppleRetning}. 

\begin{itemize}
\item Æstetikken og den grafiske kvalitet skal være høj, da der bliver fokuseret på brugeroplevelsen.
\begin{itemize}
	\item Der skal være overenstemmelse.
	\item Systemets funktioner skal fremhæves og understøttes af grafik.
	\item Direkte interaktion, og tilbagemelding på brugerens handling.
\end{itemize}

\item Brug metaforer i forbindelse med interaktion.
	
\item Apps skal reagere på forskellige gestus som let berøring, flik, dobbelt berøring og andre gestus.

\item Apps må ikke tage kontrollen fra brugerne.

\item Apps skal kunne tilpasse sig skærmen de bliver vist på, og blive vist rigtigt i forhold til skærmens position.

\item Apps bliver kun vist en ad gangen, dette betyder at en app skal være i stand til at gå i dvale og genoptage det stadie app'en var da den blev sat i dvale.

\item Der skal fokuseres på det primære formål og opfyldelse af det. App'ens indhold skal være relevant i forhold til brugeren.

\item Minimal hjælpetekst og andet, brugergrænsefladen skal være intuitiv. Logisk opbygning af app'en.

\item Undgå spildtid.

\item App'en skal overholde de opstillede mål i forhold til ikoners størrelse. 
\end{itemize}

Denne liste er en sammenkogning af Apple's retningslinjer, hvor der er lagt vægt på hovedemnerne, da de ses som de mest relevante at arbejde med. Alle Apple's retningslinjer kan ses på deres hjemmeside for udviklere, samt tabellen over ikonstørrelser \citep{AppleRetning}.

Apple prioriter brugeroplevelsen højt, da mange af punkterne fokuser netop på, at det skal være nemt og intuitivt for brugerne at bruge forskellige apps. Dette gøres ved, at fokusere på funktioner i app'en og understøtte disse grafisk. Aldrig bør grafikken overskygge funktionerne da det kan gøre app'en mindre overskuelig i forhold dens formål. Hjælpetekst eller lignende er ikke en fordel, idet app'en så ikke er hensigtsmæssigt konstrueret. 
Ved at anvende metaforer og dertil hørende gestusser til interaktionen med app'en vil app'en virke mere intuitiv for brugeren og derved gør det nemmere at bruge app'en. App'en skal altid hjælpe brugeren, dog aldrig tage kontrollen fra dem eller få dem til at føle sig magtesløse.

Disse forskellige tiltag skulle sikre en mere fornuftigt brugeroplevelse i forbindelse med brugen af en app ifølge Apple. Derfor vil der blive sammenlignet mellem disse opstillede punkter og dette projekts produkt for at se nærmere på, om produktet egentlig opfylder nogle af disse krav og om nogle af punkterne er relevante at rette produktet efter. Systemet var på testtidspunktet i samme tilstand som ved femte test, se sektion \ref{sec:femtetest}. Sammenligningen vil foregå via tabel \ref{tab:RetninglinjerApple}, metoden er den samme som ved sammenligningen mellem produktet og Googles retningslinjer.

\begin{table}[H]
\centering
    \begin{tabular}{|l|c|c|c|}
	\cline{2-4}
	\multicolumn{1}{l|}{}              								& \rotatebox{90}{Taget højde for} 	& \rotatebox{90}{Kunne gøres bedre} & \rotatebox{90}{Ikke taget højde for~}\\
	\hline
	Overensstemmelse       										& x   			  	& ~					& ~     			   		\\ \hline
	Funktionalitet understøttet af grafik  						& x					& ~					& ~							\\ \hline
	Brugen af metaforer											& ~					& x					& ~							\\ \hline
	Brugen af gestus											& x					& ~					& ~							\\ \hline
	Fokusere på primær formål									& x					& ~					& ~							\\ \hline
	Brugeren bestemmer											& ~					& x					& ~							\\ \hline
	Undgå spildtid 												& ~					& x					& ~							\\ \hline
	Overholdelse af design krav (ikon størrelse og lignende)	& ~					& ~					& x							\\ \hline
	Undgå hjælpetekst											& ~					& x					& ~							\\ \hline
	Gem brugerens arbejde og valg								& x					& ~					& ~							\\ \hline
	Logisk opbygning											& ~					& x					& ~							\\ \hline
	Tilbagemeldinger på handlinger								& x					& ~					& ~							\\ \hline
	\end{tabular}
	
\caption{Tabel over sammenligningen mellem produktet og Apples retningslinjer.}
\label{tab:RetninglinjerApple}

\end{table}

På tabel \ref{tab:RetninglinjerApple} ses sammenligning mellem produktet og Apples retningslinjer. Disse er dog retningslinjer for en app, mens der i dette projekt bliver arbejde med en hjemmeside. Derfor bliver punkternes mening overført til hjemmesiden, idet dens opbygning meget ligner en app blot det hele foregår på en hjemmeside.
Kolonnen \gaas{Kunne gøres bedre} bruges til at fortælle om, hvilke af disse punkterne der kunne indfries bedre i forhold til produktet.

\subsubsection{Konsekvenser for system}
Ud fra retningslinjerne fra Apple, er der lavet forskellige rettelser. Den største rettelse var bl.a. som følge af rettelserne fra Android retningslinjer (se afsnit \ref{sec:andretkon}): Det er forsøgt at lave et intuitivt design, og derved undgå brugen af hjælpetekster. Systemet havde før denne test forskellige hjælpebobler, hvor man kunne trykke på den og få information. Dette strider imod retningslinjerne, og derfor blev designet af hjemmesiden gennemgået. En konsekvens af rettelserne fra Apple retningslinjer var, at de to hjælpetekstbobler blev fjernet fra søgefunktionen. Herudover blev markeringen af varer nemmere for brugerne, ved at benytte symboler til at demonstrere, hvordan de markerer varer som købt. Hermed kunne hjælpeteksten også fjernes.