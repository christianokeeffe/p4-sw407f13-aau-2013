Dette kapitel omhandler alle de tests der er blevet lavet i hele projektperioden. I kapitlet vil testpersonernes baggrund blive beskrevet. Herefter vil testmetoden blive beskrevet, hvorfor denne er valgt, og i hvilket stadie systemet var, på dette tidspunkt. Til sidst vil testpersonernes svar blive dokumenteret og ud fra disse svar, er det blevet vurderet hvad der skulle ændres. Herunder kommer også en beskrivelse af konsekvenserne, altså hvad der konkret er blevet ændret af de problemstillinger, der er fundet ud fra testpersonernes svar. Der kan ses en oversigt over de forskellige tests i tabel \ref{tab:testoversigt}.

\begin{table}[H]
\begin{adjustwidth}{-50pt}{50pt}

\begin{tabularx}{\textwidth + 100pt}{|l|R|p{1.7cm}|R|R|R|p{1.5cm}|}
\hline
         \rotatebox{90}{\textbf{Test nr.}} & \rotatebox{45}{\textbf{Formål}} &  \rotatebox{45}{\textbf{Metode}} &  \rotatebox{45}{\textbf{Personer}} &  \rotatebox{45}{\textbf{System Tilstand}} &  \rotatebox{45}{\textbf{Vigtigste resultater}} &  \rotatebox{45}{\textbf{Tidspunkt}} \\ \hline
1 & At få en generel idé om design & Lave prototype af systemet &\hspace{0pt}Gruppemedlemmerne reflekterede over designet & Ingen funktioner var lavet på dette tidspunkt kun papirtest & Et udkast til systemets design blev lavet & 19/09/12 \\ \hline

2 & At få brugernes mening omkring funktionerne og design & Åbent spørgeskema & 4 testpersoner gav svar & Funktionen indkøbslister og søgefunktionen var funktionelle &  Af 16 forslag til forbedringer blev 10 implementeret & 16/10/12 til 23/10/12  \\ \hline

3 & At få testet alle funktionerne og derved finde fejl og forbedringsforslag & Åbent spørgeskema & 2 testpersoner svarede sammen og en yderligere testperson svarede & De fire hovedfunktioner var funktionelle &  Fejl ved \gaas{foreslå tilbud} og \gaas{indstillinger} rettet. Tilbage-knap ændret. Stavekontrol af søgeord lavet & 06/11/12 til 13/11/12 \\ \hline

4 & At få testet alle funktionerne og se hvordan testpersonerne bruger programmet & Test i usability lab. som blev filmet og interview  & 2 personer blev testet udover 3 fra gruppen & De fire hovedfunktioner var funktionelle & Indkøbslister er blevet rettet til at være mere brugervenlige. Vigtige knapper er blevet fremhævede & 27/11/12 \\ \hline

5 & Optimere systemet til mobiler & Kvalitativ analyse  & 2 gruppemedlemmer & Alle funktioner på nær varekategorier er udviklede & Simplificeret søgning & 23/11/12 \\ \hline

6 & Optimere systemet til mobiler  & Kvalitativ analyse  & 2 gruppemedlemmer & Alle funktioner pånær varekategorier er udviklede & Hjælpetekst fjernes vha. mere brugervenligt design & 26/11/12 \\ \hline

7 & At få testet programmet i virkeligheden & \hspace{0pt}Observation af testperson og kort interview & 1 testperson blev observeret & De fire hovedfunktioner var funktionelle, men kun indkøbslisten blev testet &  Der er blevet tilføjet varekategorier, som der også kan sorteres efter & 29/11/12 \\ \hline

\end{tabularx}
\end{adjustwidth}
\caption{Tabellen viser en kort oversigt over de forskellige test. \label{tab:testoversigt}}

\end{table}

Ud fra disse syv tests er der blevet lavet en del forbedringer på systemet løbende. På figur \ref{fig:testrettelserdiagram} kan dette ses vist med de vigtigste hændelser.

\figur{1}{TestRettelserDiagram.png}{Figur over de løbende ændringer af systemet ud fra testene.}{fig:testrettelserdiagram}

\section{Første test - Prototype}
Inden løsningens design var lavet, blev der lavet en papirprototype, hvilket skabte et udgangspunkt for designet af løsningen.

\figur{0.5}{Papirprototype.png}{Papirprototype af forsiden.}{fig:papirprototypeforside}

\subsection{Testpersonerne}
Gruppens medlemmer

\subsection{Systemets tilstand}
Prototypen blev lavet ved at brainstorme over, hvad der skulle være i løsningen, og derefter blev det diskuteret i projektgruppen, hvordan de forskellige sider burde se ud. Dette blev til at starte med skitseret på en tavle. Da gruppen var blevet enige om et udgangspunkt, blev tavleskitserne tegnet ind på et stykke papir, med en størrelse der passede hvis problemløsningen skulle vises på en smartphone. Da alle indledende ideer var tegnet på papir, blev prototypen gennemgået. Ved at lave en papirprototype blev der skabt et godt grundlag for, hvordan designet skulle se ud og hvilke funktioner der skulle arbejdes på for at klare opgaven til problemløsningen.
Det første der blev lavet, var hovedmenuen for problemløsningen, hvilket kan ses på figur \ref{fig:papirprototypeforside}. Som det kan ses, ligner det færdige design \ref{fig:GUIFor} for hovedmenuen meget det i prototypen. Måden man navigerer i problemløsningen har nogle forandringer i forhold til prototypen. I problemløsningen er der en menu øverst på siden, uanset hvor man er i problemløsningen, denne menu indeholder link til forsiden, indkøbslister og log ind/log ud (afhængig af om man er logget ind eller ej). Heraf ses det, at har været udvikling siden prototypen blev lavet. Derudover er afstandene mellem billederne og kanterne lavet mindre i forhold til prototypen, for at udnytte pladsen bedst muligt.
På prototypens forside var der fire billedknapper, hvor en endnu ikke var helt gennemtænkt. Der var nogle ideer, om at den skulle indeholde noget med tilbud, enten ugens bedste tilbud, eller tilbud baseret på indkøbsvaner. Ved problemløsningen har alle fire billedknapper en funktion. Det er tænkt, at billedknapperne skulle gå til de sider, der indeholder de forskellige hovedfunktioner: Søg, indkøbslister, tilbudsfunktion og indstillinger. Funktionen \gaas{søg} var tænkt som en side, hvor man ville kunne søge efter tilbud i databasen over tilbud. Indkøbslistefunktionen var tænkt som en side, hvor man kunne oprette indkøbslister og tilføje varer samt tilbud til listerne. Selvfølgelig skulle det også være muligt at fjerne varerne/tilbudene.
Tilbudsfunktionen skulle som tidligere nævnt være en form for afgrænsede tilbud. Indstillingssiden skulle være en simpel side, hvor brugerne kunne indstille forskellige indstillinger relaterede til deres bruger, heriblandt adgangskode, e-mail og til- og fravælge butikker fra søgeresultater. I problemløsningen blev tilbudsfunktionen, til en funktion som ud fra tidligere indkøb, foreslår tilbud til brugeren. 

\subsection{Kvalitetsmål i fokus}
Ved prototypen blev det undersøgt om løsningen var brugbar og forståelig. Dette blev gjort ved at se på hvor hurtig og nemt man kunne navigere i prototypen.

\subsection{Testmetoden}
Gruppens medlemmer prøvede at bruge prototypen, som om det var et reelt produkt. Ved dette kan ideen bag prototypen blive afprøvet og se om de holder i praksis, uden at bruge for meget tid på det. 

\subsection{Konsekvens af første test}
Log ind i prototypen var lavet som en pop op, mens den i problemløsningens har en separat side, hvor der er nogle funktioner såsom: Muligheden for at oprette en bruger til problemløsningen og hvis brugeren har glemt en adgangskode eller brugernavn, kan man få tilsendt en e-mail således brugeren har mulighed for at generhverve adgang til deres konto.

Det var bestemt at funktionen \gaas{søg} skulle være med lige fra starten, men ved prototypen kunne man kun tilgå den via forsiden. Ved problemløsningen kan man derimod også finde tilbud på varer på ens indkøbsliste, ved at gå ind på indkøbslisten, og så vælge den vare man vil finde tilbud på. Man vil så blive præsenteret for et pop op-vindue med indstillinger, hvor en af disse er \gaas{søg}, således behøver man ikke at gå ind på selve siden gaas{søg}, for at finde et tilbud.

\subsection{Afrunding første test}
Prototypen er blevet brugt i gruppen til at skabe en grund-idé til designet på løsningen. Prototypen blev ikke fremvist til eksterne personer, men blev som den første test prøvet på personerne internt i gruppen. Det gav også et godt indblik i hvad der var krævet at systemet.

\section{Anden test}
\label{sec:andentest}
I dette afsnit beskrives alt omkring anden test og hvad konsekvenserne af denne test blev. 

\subsection{Testpersonerne}
Testen er blevet sendt ud til vores seks informanter. Af disse har fire svaret. Alle testpersonerne er unge mellem 18 og 25 år, har en smartphone, og med en begrænset indtægt. Desuden blev testen også gennemgået af projektgruppen.

\subsection{Systemets tilstand}
Testen blev udført da systemets kernefunktioner lige var blevet udviklet. Disse kernefunktioner var: Oprettelse af profil, oprettelse af flere indkøbslister, søgning af tilbud, tilføje og fjerne varer til og fra indkøbslisterne. 

\subsection{Kvalitetsmål i fokus}
I denne test forsøgtes kvalitetsmålene \gaas{brugbar} og \gaas{forståeligt} testet. Årsagen til at netop disse kvalitetsmål blev sat i fokus i den første brugertest var, at det ønskedes testet om informanterne ville benytte systemet til hverdag, og om systemet i det hele taget var forståeligt for dem. 

\subsection{Testmetoden}
I denne test er testpersonerne blevet bedt om at teste visse funktioner på hjemmesiden på en smartphone, og derefter svare på et spørgeskema. De funktioner som skulle testes i denne omgang var: oprettelse af en bruger, søgning af tilbud, oprettelse af flere indkøbslister og at tilføje/slette/redigere disse lister. Hjemmesiden var på daværende tidspunkt en prototype, hvor lige præcis de beskrevne funktioner virkede.
Selve spørgeskemaet indeholdte åbne spørgsmål, hvor der ingen svarmuligheder var, men hvor testpersonen forventedes at beskrive deres observationer og meninger. Spørgsmålene spurgte til testpersonernes meninger omkring funktionaliteten og designet af de funktioner der blev testet. Spørgeskemaet kan findes i bilag \ref{chap:foesteskema}.
Denne testmetode blev valgt, da det skulle undersøges hvorvidt de to funktioner indeholdte hvad testpersonerne mente burde være i sådanne funktioner, om der manglede noget, eller om der kunne laves forbedringer i forhold til kvalitetsmålene i tabel \ref{tab:kriterier}, og deres prioriteringer. Der blev valgt et spørgeskema med åbne spørgsmål, da det dækker behovet for at få den information, der er relevant for det videre forløb i projektet. Der blev valgt et spørgeskema frem for et interview i denne omgang, fordi det sparer tid. 

Begrænsningerne ved denne testform, er at spørgsmålene er prædefinerede, således er det ikke muligt at stille opfølgende spørgsmål, hvis sådanne skulle opstå. Derudover kan man risikere en lav svarprocent ved udsendelse af spørgeskemaer.

\subsection{Testpersonernes svar}
Ud fra testpersonernes svar blev der udledt følgende punkter, som enten skulle rettes eller vurderes om de skulle implementeres.
\begin{itemize}

\item \textbf{Selve søgningen i søge-funktionen:}
Testpersonerne undrer sig over irrelevante søgeresultater. Der bliver desuden foreslået at søgefunktionen skal vise forslag til, hvad der søges på, mens der skrives. 

%Der skulle bedre søgekriterier til så søgningen kun søgte på overskriften og ikke selve beskrivelsen af et tilbud. Desuden vil det undersøges om det er muligt at få søgefunktionen til selv også at søge på synonymer. Derudover ønskedes det at søge-funktionen selv gav forslag til hvad der søges på imens en bruger skriver.

\item \textbf{Markering i indkøbslisten:}
Kun få af testpersonerne kunne finde ud af at markere varerne i indkøbslisten. 


\item \textbf{Systemets hastighed:}
En af testpersonerne påpegede at systemet virkede langsomt over mobilforbindelsen.


\item \textbf{Sorteringen i søgefunktionen:}
Testpersonerne ønskede, at der kunne sorteres bedre for eksempel efter billigste pris eller efter bedste mængdepris.

%dette vil blive set nærmere på hvorvidt det kan løses.

%Dette skal redesignes så det er tydeligere hvordan dette gøres.



\item \textbf{Fejl:}
Under testene blev der fundet tre fejl. Den ene fejl var, at siden gik ned da der forsøgtes at skrive et navn til en indkøbsliste med en apostrof. Derudover fik en af testpersonerne en fejlbesked. Den sidste fejl der blev fundet, var at en af testpersonerne oplevede at tilbudene ikke forsvandt fra søgeresultatet, efter han havde tilføjet dem til en indkøbsliste. Fejlrettelserne blev prioriteret således:  Fejlbesked, indkøbsliste med apostrof, fjerne tilbud fra søgeresultat



\item \textbf{Indkøbslisten:}
Den ene testperson foreslog at billederne fra tilbudene også blev tilføjet på indkøbslisten. Desuden ønskede en af testpersonerne at man kunne tilføje flere varer til listen på én gang. Der var også en enkelt testperson der havde problemer med at tilføje vare direkte til indkøbslisten.


\item \textbf{Navigation:}
Flere testpersoner ønskede en tilbage-knap udover den allerede eksisterende \gaas{forside} knap.


\item \textbf{Prisvisning:}
En af testpersonerne ønskede en bedre visning af priserne.
% og der vurderes på om det skal redesignes.


\item \textbf{Design:}
Nogle af testpersonerne fandt designet af programmet kedeligt. 

%Dette vil blive vurderet om der skal foretages en komplet redesign eller om der skal fortages nogle små ændringer.

%det kan overvejes om det skal designes på en anden måde.


\item \textbf{Enhed:}
Der var en testperson der ønskede at man kunne tilføje vare til indkøbslisten med andre enheder end stk. for eksempel gram. 
%Der vurderes om det skal gøres muligt selv at vælge eller angive enhed.

\item \textbf{Touch screen:}
En af testpersonerne klagede over at da touch screens er yderst følsomme skete det nogle gange at når han/hun ville rulle ned i listen blev noget valgt i stedet. 

%Dette kan muligvis løses ved at man skal holde knappen eller fingeren nede noget længere for at vælge et tilbud.
\end{itemize}

\subsection{Konsekvenser af anden test}
Ud af de 16 forslag til forbedringer er 10 blevet rettet eller integreret, mens de sidste punkter blev fundet irrelevante eller for tidskrævende til at blive rettet.


Problemet med, at der kom mange irrelevante resultater af en søgning er blevet rettet ved at der er blevet lavet en filterret søgning. Desuden kommer der nu forslag ved søgningerne. Ved søgefunktionen er der også blevet lavet en funktion, der giver mulighed for selv at vælge hvad resultatlisten skal sorteres efter. Alle småfejl som for eksempel stavefejl i systemet er også blevet rettet.
Programmet er desuden blevet optimeret, så der skal laves færre eksterne kald og systemet bliver derved hurtigere.


Ved indkøbslisten er der blevet lavet en infoboks, hvor brugeren kan finde hjælp, hvis det er nødvendigt. Stedet man tilføjer vare til indkøbslisten er blevet flyttet, så det nu er over indkøbslisten. Desuden er pilen i listen blevet lavet om til en knap, der åbner et pop-op vindue med indstillingerne for indkøbslisten. På indkøbslisten er der nu et billede af hvert enkelt tilbud på indkøbslisten, og der er blevet rettet flere småfejl. Der er også lavet en SQL-validering der håndterer specialtegn i navne på indkøbslisterne.
Problemstillingen med at forsøgspersonerne synes der manglede en tilbage-knap er blevet løst, så der nu både er en tilbage-knap og en forside-knap. Der er også lavet en bedre prisvisning. Det sidste der er blevet rettet er at programmet i sit hele er blevet optimeret, så hastigheden af hjemmesiden er blevet hurtigere.


De punkter der ikke er blevet rettet er:
Automatisk søgning efter synonymer, at kunne holde fingeren eller musen inde for at markere eller vælge en vare, multitilføjelse, muligheden for andre enheder end stk., og fjernelse af tilbud efter tilføjelse blev alle fravalgt grundet mangel på tid og ressourcer. Designet blev desuden ikke rettet da dette skal gennemtænkes mere.

\subsection{Afrunding anden test}
De vigtigste problemstillinger der blev observeret i anden test var:
\begin{itemize}
\item Mange søgeresultater var irrelevante.
\item Testpersonerne kunne ikke gennemskue, hvordan de skulle markere varerne.
\end{itemize}
De vigtigste problemstillinger der blev rettet i anden test var:
\begin{itemize}
\item Søgninger søger nu kun i titlen, så de fleste af de irrelevante resultater bliver frasorteret.
\item Markeringen blev ændret, så det nu var mere tydeligt hvordan den fungerede.
\end{itemize}

Disse problemstillinger blev prioriteret som de vigtigste, idet de blev fundet essentielle i forbedringen af brugeroplevelsen samt opfyldelsen af de opsatte kriterier som kan ses på tabel \ref{tab:kriterier}.

\section{Tredje test}
Herunder kommer alt omkring tredje test. Testpersonerne er de samme som fra anden test, se sektion \ref{sec:andentest}.

\subsection{Systemets tilstand}
På tidspunktet for tredje test var de fire menupunkter på systemets forside blevet udviklet: Søg, indkøbslister, foreslåede tilbud og indstillinger. Hele systemet var altså på dette tidspunkt funktionelt.  

\subsection{Kvalitetsmål i fokus}
Formålet med denne test var at kontrollere systemets brugergrænseflade, dens funktionalitet og brugernes forståelse af systemet. Selve testen var derfor formet ud fra kvalitetsmålene \gaas{brugbart}, \gaas{forståeligt} og \gaas{effektivt}. Et spørgeskema var blevet udformet så spørgsmålene fokuserede på disse punkter.
Kvalitetsmålene er valgt, fordi de er blandt de meget højt prioriterede, med undtagelse af \gaas{effektivt} som kun var prioriteret \gaas{vigtigt}, hvor de to andre var prioriteret \gaas{meget vigtigt}. 

\subsection{Testmetoden}
Selve testmetoden er den samme som i anden test, men hjemmesiden er blevet udviklet en del siden forrige test. Alle hjemmesidens fire funktioner er nu blevet udviklet. Desuden er de blevet bedt om at fortsætte med at bruge hjemmesiden når de handler, også selvom testen er sat til at slutte efter en uge. Mailen og spørgeskemaet der blev sendt til informanterne kan ses i bilag \ref{chap:andetskema}. Testpersonerne blev bedt om, at lave testen sammen med en anden.

\subsection{Testpersonernes svar}
I denne omgang af test svarede tre ud af seks informanter, og heraf svarede to testpersoner sammen. Alle informanterne meldte desværre at der var fejl, der gjorde det umuligt at udføre dele af den første opstillede opgave. Informanterne var meget positive over de forbedringer og ændringer der er sket siden sidste test. Ud fra svarene kan det konstateres at informanterne har set de fleste af forbedringerne. De varer der er blevet forslået ændret er:
At det er muligt at kunne sortere efter kategorier som for eksempel madvarer og byggemarked. Der blev igen foreslået multitilføjelse af varer. Desuden ønskes der en standardindkøbsliste og standardvarer så det er nemmere at lave en basis indkøbsliste. Ud over dette blev der foreslået en GPS-funktion, der kunne beskrive eller vise hvor de enkelte butikker rent fysisk befandt sig.

\subsection{Konsekvenser af tredje test}
Systemet blev rettet således det blev operationsdygtigt igen og det er muligt at bruge systemets funktioner. Et af ønskerne var at der var en standardliste med standardvarer. Dette blev prioriteret værende mindre vigtigt. Baggrunden for dette er at denne egenskab ikke er nødvendig for systemets funktionalitet.
\gaas{Tilbage-knappen} som blev omtalt i anden test er blevet lavet om til at være en knap, der sender brugeren til oversigten over indkøbslister. Dette gøres fordi \gaas{tilbage-knappen} aldrig kom til at fungere som forventet og derfor passer denne løsning bedre til det tidligere ønske informanterne havde.
Forslaget med GPS-funktion anses som værende for komplekst og tidskrævende, at anvende i den givne løsning, men kunne være relevant at overveje, hvis systemet engang skal eksporters til en app.
Systemet har nu også en fungerende stavekontrol i søgningen således at brugerne får forslag på forkert stavede ord.

\subsection{Afrunding tredje test}
De vigtigste problemstillinger der blev observeret i tredje test var:
\begin{itemize}
\item Der var to fejl, der gjorde at to af funktionerne på siden ikke kunne bruges.
\item Der blev foreslået at varerne kunne sorteres efter varekategori.
\end{itemize}
De vigtigste problemstillinger der blev rettet i tredje test var:
\begin{itemize}
\item Fejlene blev rettet, så siden igen blev funktionsdygtigt.
\end{itemize}
Disse var de vigtigste, da de gjorde dele af siden ubrugelig. 

\section{Fjerde test}
Her beskrives detaljerne omkring fjerde omgang af tests.

\subsection{Testpersonerne}
Denne gang testede tre af gruppemedlemmerne systemet, for at se om testen var klar til testpersonerne. Dagen efter testede to udefrakommende personer systemet. Begge testpersoner var studerende kvinder på 20 år, med en middel teknologisk erfaring.


\subsection{Systemets tilstand}
På tidspunktet for fjerde test var alle kernefunktioner i systemet udviklede, og skulle derfor gennemtestes, så kritiske fejl kunne findes og rettes. 

\subsection{Kvalitetsmål i fokus}
Kvalitetsmålene \gaas{pålideligt}, \gaas{effektiv}, \gaas{korrekt}, \gaas{testbar} og \gaas{flytbar} blev forsøgt testet. Kvalitetsmålet \gaas{flytbart} blev testet da det bl.a. ønskedes testet om en bruger kunne bruge en del af systemet på en almindelig computer, og derefter arbejde videre med en anden del på en mobil platform. \gaas{Effektiv}, \gaas{korrekt} og \gaas{pålidelig} blev testet ved at testpersonerne i usability laboratoriet blev bedt om at lade som om de skulle handle ind i virkeligheden. Dermed blev det opdaget, hvis systemet pludseligt gik ned eller ikke reagerede som forventet. 

\subsection{Testmetoden}
Der blev foretaget tests af hele hjemmesiden i universitets usability laboratorium. Testpersonerne blev filmet mens de udførte seks opgaver med systemet, og fortalte hvad de gjorde og hvorfor. Opgaverne gik ud på at teste de forskellige dele af systemet på computer og smartphone. Desuden gik en af opgaverne ud på, at testpersonerne skulle bruge systemmet mens de gik rundt som om de handlede. Efter opgaverne var gennemført blev der afholdt et semi-struktureret interview med testpersonen inde i testlokalet der også blev filmet. Et af gruppemedlemmerne sad inde hos testpersonen og spurgte ind til det der skete under testen og hjalp hvis en testperson sad fast i en af opgaverne. Desuden var det også denne person der interviewede testpersonerne. Fra et kontrolrum fulgte en tidslogger samt en testobservatør, hvis opgave var at skrive ned, hvis der skete noget interessant og derudover en person til at styre kameraerne. Udgangspunktet for denne testmetode er \citep{DEBUL}.

\subsection{Observationer fra fjerde test}
I dette afsnit beskrives observationerne fra testene. Det er delt op i afsnit efter de forskellige funktioner der blev testet. Dette giver overblik over evalueringen af de forskellige dele af systemet.\newline
\newline
\textbf{Indkøbslister på computeren:}\newline
Alle testpersonerne, inklusiv de tre gruppemedlemmer, glemte at give indkøbslisten et navn, da de ville lave en ny liste. Desuden klagede flere testpersoner over, at det ikke var muligt at se hele teksten i et tilbud, og derfor kom de til at tilføje forkerte tilbud til indkøbslisten. Flere af testpersonerne kom også til at lave fejlen, at de krydsede en vare af i stedet for at åbne indstillingerne flere gange. En af testpersonerne kom til at ramme forsideknappen i stedet for feltet til at skrive en vare ind, da disse felter ligger tæt på hinanden.

Det var et problem for to af testpersonerne at man ikke kunne se at et tilbud man havde valgt rent faktisk blev tilføjet. Den ene ønskede at pop op-vinduet automatisk lukkede. Der skete en underlig sortering af listen der fik en af varerne til at flytte op og ned på listen. Flere at testpersonerne klagede over, at titlerne på tilbuddene kunne være meget intetsigende og ønskede at kunne se en beskrivelse. Flere af testpersonerne ledte efter en \gaas{del}-knap under selve indkøbslisten. Desuden bemærkede en af testpersonerne, at han først ikke kunne se nogen tilknyttede brugere på listen, og da han åbnede den igen stadig ikke kunne se sig selv. Den ene af testpersonerne forsøgte at tilgå indkøbslisterne uden at logge ind/lave en bruger først. En af testpersonerne forsøgte at sortere listen efter butikker uden at have fundet nogen tilbud. Desuden foreslog en af testpersonerne at systemet selv skulle kunne finde tilbud ud fra ens indkøbsliste, så man kunne få det billigst muligt, men nøjes med at handle i én butik.

En af testpersonerne lavede desuden fejlen, at testpersonen kom til at slette en vare, da testpersonen ville se hele titlen på et tilbud og derfor trykkede på pilen ud for \gaas{fjern varen}. En af testpersonerne oplevede, at da et tilbud blev valgte og derefter et andet tilbud skete der ikke noget anden gang. Begge testpersoner var i tvivl om de havde valgt det bedste tilbud. De klagede også over der manglede detaljer i tilbuddet som mængde, mængdepris, type og beskrivelse. Begge testpersoner overser også \gaas{eller} i teksten ved \gaas{del liste} hvilket gør de efterspørger en mail, når de allerede har et brugernavn. Dette gør også at den ene testperson bliver forvirret, over at der er to tekstfelter.

Den ene testperson bemærker at der også i pop op-vinduet er en dårlig sortering af listen over tilbud. En anden testperson følte ikke det var intuitivt, at man skulle oprette en indkøbsliste før man kunne tilføje varer via systemet. Desuden blev der foreslået at markøren automatisk skulle returnere til boksen, hvor man skriver vare ind, efter at have tilføjet en vare og at afkrydsede varer automatisk skulle fjernes efter et døgn. \newline
\newline
\textbf{Indkøbslister på en smartphone:}\newline
En af testpersonerne klagede over at det var svært at skrive varer ind på smartphonen da tastaturet var for småt. Også på smartphonen havde flere testpersoner problemer med at kende forskel på menuen og hvordan man markerede en vare som købt. En af testpersonerne overså blandt andet feltet til at skrive antal i, da testpersonen arbejdede med smartphonen. Her blev der også klaget over man ikke kunne se alle detaljer i pop op-vinduet.

En af testpersonerne ønskede desuden at have mulighed for at klikke på billedet af tilbuddet og så se det i stor udgave. Samme testperson overser sorteringsknappen og kan derfor ikke finde ud af, hvad listen er sorteret efter. Markeringerne blev underlige når to personer arbejdede samtidigt, og det gav fejl i markeringerne. Desuden gik der noget tid før testpersonerne lagde mærke til at en anden havde købt noget, og i de første tre test opdagede de det slet ikke. Der blev klaget over at systemet var langsomt og at der kunne gå op til 12 sekunder inden systemet registrerede, at en vare var blevet markeret, og dette gjorde at en testperson trykkede flere gange uden resultat. Desuden skete der en visningsfejl i den ene af testene. En af testpersonerne glemte helt at krydse varerne af, da testpersonen handlede trods det stod i opgaven. Desuden fandt en af testpersonerne listen uoverskuelig og ønskede at kunne sortere den efter varekategori.\newline
\newline
\textbf{Foreslåede tilbud:}\newline
En fejl er, at der bliver forslået tilbud der allerede findes på indkøbslisten.Testpersonerne fandt hurtigt ud af, hvordan funktionen virkede. De to kvindelige testpersoner leder desuden efter funktionen under indkøbslisterne. Et andet problem var, at der ingen respons var omkring at en vare rent faktisk var tilføjet til indkøbslisten. En af testpersonerne mente stadig der var mange irrelevante tilbud og klagede over det samme tilbud optrådte flere gange, trods det var i forskellige butikker, og foreslog de skulle kombineres under ét hvis det var samme vare.\newline
\newline
\textbf{Indstillinger:}\newline
En af testpersonerne trykkede \gaas{bekræft} i indstillingerne, selv om dette ikke var nødvendigt. Samme person forsøgte at finde \gaas{gem knappen} i toppen af butikslisten, mens den var i bunden. Den ene testperson overser fuldstændig muligheden for at indstille valgte butikker i indstillingerne og må til sidst hjælpes derind. Desuden overser en af testpersonerne knapperne til at slå alle butikker fra/til og sidder derfor og slår alle butikkerne fra enkeltvis. Flere testpersoner klager desuden over at navnene på butikkerne står så langt til venstre at det er svært at læse. En af testpersonerne mente at formuleringen på \gaas{forbliv logget ind}, \gaas{alle til} og \gaas{alle fra} knapperne var misvisende og ikke gav mening. Dette gjorde at testpersonen byttede om på knapperne \gaas{alle til} og \gaas{alle fra}.\newline
\newline
\textbf{Generelt:}\newline
Alle testpersoner mente at man burde blive automatisk logget ind efter at have oprettet en ny bruger. Den ene af de kvindelige testpersoner mente det var nemmest at bruge smartphonen til at skrive en indkøbsliste, mens den anden hellere ville bruge computeren. Begge testpersoner synes designet er en smule kedeligt, trods det er konsistent, og den ene ønskede at designet blev mere computervenligt. Dog mente de designet var pænt på en smartphone.

\subsubsection{Opsummering}
Vurderingskriterierne for vurderinger af brugervenlighedsfejl kan ses på tabel \ref{tab:Brugervejledningsfejlvurdering}.

\begin{table}[H]
\centering
    \begin{tabular}{|l|c|c|c|}
	\hline
	~                           & Forsinkelse 	& Irritation	& Forventet vs. faktisk resultat~ 	\\ \hline
	Kosmetisk       			& < 1 minut   			  		& Lav						 & Lille forskel \\ \hline
	Seriøs       				& Flere minutter 				& Medium					 & Signifikant forskel \\ \hline	
	Kritisk       			    & Total (bruger stopper)  		& Stærk						 & Stor forskel \\ \hline
	
	\end{tabular}
	
\caption{Tabel som viser hvordan burgervenlighedsfejl bliver vurderet. \citep{debslide}}
\label{tab:Brugervejledningsfejlvurdering}

\end{table}

Observationerne er kort opsummeret i tabel \ref{tab:opsummeringaffejlfrausabilitylab}.

\begin{table}[H]
\centering
    \begin{tabular}{|l|c|}
	\hline
	Fejl 																& Type  \\ \hline
	
	Glemte indkøbslistens navn      									& Kosmetisk						   \\ \hline
	Ikke muligt at se hele teksten i et tilbud							& Kosmetisk						   \\ \hline	
	Krydse vare af i stedet for at åbne indstillinger					& Kosmetisk						   \\ \hline
	Ingen feedback når tilbud på vare blev valgt						& Kosmetisk						   \\ \hline
	Varer flyttede ulogisk rundt på listen								& Seriøst						   \\ \hline
	Brugeren optræder ikke selv på listen over tilknyttede brugere		& Kosmetisk						   \\ \hline
	Både muligt at dele indkøbsliste via e-mail og brugernavn			& Kosmetisk						   \\ \hline
	Dårlig sortering af tilbud i pop op									& Kosmetisk						   \\ \hline
	Svært at kende forskel på menu og markering af vare					& Kosmetisk						   \\ \hline
	Systemet kan ikke håndtere at to personer ændrer listen samtidig	& Kritisk						   \\ \hline
	Systemet kan virke langsomt											& Seriøst						   \\ \hline
	Der kan blive foreslået tilbud som allerede er på ens indkøbsliste	& Kosmetisk						   \\ \hline
	Funktionen \gaas{Slå alle butikker fra/til} overses					& Kosmetisk						   \\ \hline
	Man bliver ikke automatisk logget ind efter oprettelse af profil	& Kosmetisk						   \\ \hline	
	Kan ikke finde indstillinger for butikker	& Kritisk						   \\ \hline	
	
	
	\end{tabular}
	
\caption{Opsummering af fejl fundet i usability laboratorium.}
\label{tab:opsummeringaffejlfrausabilitylab}

\end{table}

Resultaterne fra tabel \ref{tab:opsummeringaffejlfrausabilitylab} kan ses talt sammen i tabel \ref{tab:antalaffejlfejlfrausabilitylab}.
\begin{table}[H]
\centering
\begin{tabular}{|c|c|}
\hline
Type fejl & Antal\\ \hline
Kritisk   & 2 \\ \hline
Seriøst   & 2 \\ \hline
Kosmetisk & 11 \\ \hline
\end{tabular}
\caption{Antal af fejl fundet i usability laboratorium.}
\label{tab:antalaffejlfejlfrausabilitylab}
\end{table}

\subsection{Konsekvenser af fjerde test}
Ud af 22 forslag til forbedringer er 15 blevet implementeret, fire er blevet fravalgt fordi de ikke blev fundet muligt eller relevante, og tre blev fravalgt grundet mangel på tid og ressourcer.\newline
Det der er blevet implementeret er:
\begin{itemize}
\item Knappen \gaas{Opdater} under delt liste er blevet omdøbt til \gaas{Del liste}. 
\item Felterne til mail og brugernavn blevet slået sammen og håndterer nu begge inputs i et felt.
\item systemet logger nu brugeren automatisk ind, efter man har lavet en ny bruger.
\item \gaas{Forbliv logget ind} ændret til man ikke logger af automatisk.
\item Inde i \gaas{Tilvalg af butikker} er til og fra knapperne omdøbt til \gaas{Tilvælg alle butikker} og \gaas{Fravalg alle butikker} og knapperne er gjort blå, så de er mere synlige.
\item Der er nu lavet et pop op-vindue der beder om et navn, når man vil oprette en ny indkøbsliste.
\item Pop op-vinduet med indstillinger inde i indkøbslisten lukker nu sig selv efter et tilbud er blevet tilføjet.
\item Der er nu under tilbuddene i indkøbslisten blevet tilføjet flere detaljer, herunder mængdepris.
\item Hvis titlen er for lang på tilbuddene, bliver det opdelt i flere linjer.
\item Der er blevet byttet om på placeringen af indstillingsknappen og knappen der markerer varer som \gaas{købt} på indkøbslisten.
\item Markøren flytter sig nu automatisk til tekstfeltet, hvor man kan indskrive en vare når siden genloades.
\item For at gøre sorteringsfunktionen mere synlig inde under indkøbslisterne, er den nu inde i en grå boks på hjemmesiden, hvilket skaber mere kontrast.
\item Det oprindelige søgeord bliver nu bibeholdt til senere søgninger af tilbud.
\item Pop op-vinduet med indstillinger ude i oversigten over en brugers indkøbslister opdateres efter en liste er blevet delt.
\item Brugeren kan nu bedre se at systemet har registeret en handling, repræsenteret ved prikker der løber rundt i en cirkel.
\end{itemize}
Det der er fravalgt på grund af relevans:
\begin{itemize}
\item At kunne filtrere funktionen foreslåede tilbud mere så der ikke kommer irrelevante tilbud.
\item At systemet selv skulle slette markerede varer efter et døgn.
\item At systemet selv skulle sammenlægge ens tilbud, hvis de har samme butik og pris.
\end{itemize}
Det der er fravalgt på grund af mangel på tid og ressourcer:
\begin{itemize}
\item Indstillingerne af tilvalgte butikker skal være listespecifik. 
\item At kunne sortere varerne efter varekategori og hvilken rækkefølge varerne kommer i butikken.
\item Gøre så det er muligt at vælge, om man vil have mange detaljer på varerne i indkøbslisten eller have mulighed for at se flere på skærmen.
\end{itemize}

\subsection{Afrunding fjerde test}
De vigtigste problemer der blev observeret i fjerde test var:
\begin{itemize}
\item Der var flere problemer under indkøbslisten som blev vurderet som problematiske.
\item Flere knapper blev overset da de ikke var tydelige nok, og flere af formuleringerne på knapperne blev fundet misvisende.
\end{itemize}
De vigtigste problemer der blev rettet i fjerde test var:
\begin{itemize}
\item Flere problemer omkring indkøbslisten blev rettet, så det passede bedre med den metode testpersonerne brugte systemet på.
\item De fleste af knapperne i systemet er nu blevet blå eller fremhævet på anden måde.
\end{itemize}
Disse blev vurderet som vigtigste på baggrund af tabel \ref{tab:opsummeringaffejlfrausabilitylab}

\section{Android og Apple}
Det er interessant at se nærmere på de større selskabers retningslinjer for app udvikling, idet systemet udvikles til at være mobiloptimeret. Derfor er det blevet valgt at se nærmere på Android's og Apple's sæt af retningslinjer for udvikling af apps. Selvom det er en hjemmeside dette projekt omhandler, er det også meningen at produktet skal bruges på en mobil enhed. Derfor kan disse retningslinjer være et godt udgangspunkt at designe websiden ud fra, da det vil give en bedre oplevelse på mobile enheder og gøre det nemmere at lave en decideret app i fremtiden.

\subsection{Kvalitetsmål i fokus}
Ved både test fem og seks blev der lagt fokus på kvalitetsmålene \gaas{brugbart} og \gaas{forståeligt}. Disse kvalitetsmål blev markeret som \gaas{meget vigtigt}, og dermed var der også lagt stor fokus på dette. Dette er blevet gjort ved at bruge nogle allerede udviklede retningslinjer til at teste disse kvalitetsmål.

\subsubsection{Systemets tilstand}
Systemets tilstand under disse to tests, er at alle kernefunktioner i systemet er færdigudviklede, og systemet er funktionsdygtigt. 


\subsection{Femte test - Androids retningslinjer}
\label{sec:femtetest}
I dette afsnit vil de retningslinjer, som Google anbefaler man følger for at lave en Android applikation i særklasse, blive undersøgt.
Informationerne er taget fra Android Developers \citep{AndRet}.

Vejledningen er blevet delt op i tre emner:
\subsubsection*{Enchant me}
Som udgangspunkt, det brugeren ser og føler når de bruger app'en.

Udseende er ikke alt, små effekter der får app'en til at føles mere interaktiv og giver en følelse af respons, giver meget til oplevelsen. Det at flytte eller interagere med objekter frem for knapper kan også være med til at gøre oplevelsen meget sjovere. Ved at kunne ændre på udseende, gives brugeren en følelse af at app'en tilhører dem og at det er dem der bestemmer. Hvis app'en også husker på tidligere valg, så giver det følelsen af at app'en lærer en at kende, hvilket i sidste ende gør brugen af app'en nemmere og hurtigere.

\subsubsection*{Simplify my life}
Hvor nemt og tilfredsstillende det er at bruge app'en.

Simpelt og stilrent er essensen af et godt design. Jo mindre det er nødvendigt at skrive, des bedre, og hvis det kan erstattes af et billede, er det endnu bedre. Det er også meget af tiden bedre bare at komme med et bud, på hvad brugeren vil have, frem for at spørge dem om for meget. For meget information på én gang kan være med til at forvirre, frem for at hjælpe, så vis kun hvad der er nødvendigt, så som hvor i systemet brugerne befinder sig, så de får følelsen af at vide, hvad de laver. Hvis brugeren har brugt tid på at lave noget, så er det vigtigt at det kan gemmes, så de kan tage det med sig videre på f.eks. andre enheder eller hvis app'en skal opgraderes. Hold konsistens over hele app'en, hvis noget ligner hinanden så skal de også gøre det samme ellers skabes der forvirring. Brugere skal kun afbrydes fra det de laver hvis det, der gives besked om er meget vigtig, da brugere godt kan lide at holde fokus på det, de laver.
\subsubsection*{Make me amazing}
Generelt evnen til at få brugeren til at føle sig klog og smart.

Brugere elsker når de finder ud af noget selv, så det bør prøves at implementere funktioner og udseende fra Android apps, derved får de en positiv overraskelse når funktionerne fungerer, som de regner med. Brugere vil ikke føle sig dum når de bruger app'en, så hvis noget går galt, vær sikker på at give dem simple instrukser og spar dem for lange forklaringer. Meget svære eller avancerede opgaver skal helst deles så meget op som de kan for at gøre det nemmere for brugerne. Det er vigtigt at give en form for respons selvom det måske ikke er mere end et svagt lys bag et ikon. Gør det nemt for begyndere at føle sig som eksperter, som f.eks ved at sammensætte nogle filtre og indstillinger til én knap i et fotografiapparat, så de nemt kan tilgås. Vær opmærksom på at fremhæve vigtige knapper og funktioner, så som knappen til at tage et billede med er mere fremhævet end knapperne til at indstille effekter.

\subsubsection{Sammenligning}
I dette afsnit er der vist en tabel, som sammenligner dette projekts produkt og Androids retningslinjer, som kan ses på tabel \ref{tab:RetninglinjerAndriod}. Systemet havde på dette tidspunkt alle de endelige funktionaliteter, med undtagelse af muligheden for at sortere efter varekategori. Det er gjort for at give et indtryk af, hvordan produktet passer på de retningslinjer som det anbefales at Android produkter lever op til for at give den bedste oplevelse. 

\begin{table}[H]
\centering
    \begin{tabular}{|l|c|c|c|}
    \cline{2-4}
\multicolumn{1}{l|}{}       & \rotatebox{90}{Taget højde for} 	& \rotatebox{90}{Kunne gøres bedre}& \rotatebox{90}{Ikke taget højde for~}	\\ 
\hline
Interaktivt design	       	& ~    			  	& x					& ~						\\ \hline
Objekter frem for knapper  	& ~					& ~					& x						\\ \hline
Personificering				& ~					& x					& ~						\\ \hline
Lær mig at kende			& x					& ~					& ~						\\ \hline
Hold det simpelt			& ~					& x					& ~						\\ \hline
Billeder frem for ord		& ~					& ~					& x						\\ \hline
Vælg for mig,(jeg bestemmer)& x					& ~					& ~						\\ \hline
Vis kun det nødvendige		& x					& ~					& ~						\\ \hline
Hvor er jeg på siden?		& x					& ~					& ~						\\ \hline
Gem mine ting				& x					& ~					& ~						\\ \hline
Konsistens					& ~					& x					& ~						\\ \hline
Notifikation, kun vigtigt	& x					& ~					& ~						\\ \hline
Genbrug kendte effekter		& ~					& ~					& x						\\ \hline
Simpel fejlfinding			& ~					& x					& ~						\\ \hline
Enkle opgaver				& x					& ~					& ~						\\ \hline
Gør det tunge arbejde		& x					& ~					& ~						\\ \hline
Fremhæv vigtige ting		& x					& ~					& ~						\\ \hline  

\end{tabular}

\caption{Tabel over sammenligningen mellem produktet og Andriods retningslinjer.}
\label{tab:RetninglinjerAndriod}

\end{table}
%ret herfra
\subsubsection{Konsekvenser for systemet}
\label{sec:andretkon}
Ud fra undersøgelsen af retningslinjerne, er der lavet nogle ændringer ved systemet, for at optimere det til den bedst mulige brugeroplevelse. For at forbedre punktet med at holde det simpelt er det forsøgt at simplificere nogle af de forskellige dele af hjemmesiden, specielt metoden til at søge. Her var der før to søgeknapper: en til filtreret søgning og en anden til fuld søgning. Den filtrerede søgning forsøgte at gøre resultaterne mere relevante i forhold til søgeordet og varer fra fravalgt butikker blev ikke vist. Den fulde søgning tog alt hvad der kom frem. Det blev vurderet, at man altid ville have relevante søgeresultater, så derfor blev det for simpelhedens skyld valgt at fjerne knappen til fuld søgning.
\subsection{Sjette test - Retningslinjer for Apples apps}
I dette afsnit vil der blive fokuseret på Apples sæt af retningslinjer og hvorledes dette projekts produkt følger nogle af deres principper og retningslinjer \citep{AppleRetning}. 

\begin{itemize}
\item Æstetikken og den grafiske kvalitet skal være høj, da der bliver fokuseret på brugeroplevelsen.
\begin{itemize}
	\item Der skal være overenstemmelse.
	\item Systemets funktioner skal fremhæves og understøttes af grafik.
	\item Direkte interaktion, og tilbagemelding på brugerens handling.
\end{itemize}

\item Brug metaforer i forbindelse med interaktion.
	
\item Apps skal reagere på forskellige gestus som let berøring, flik, dobbelt berøring og andre gestus.

\item Apps må ikke tage kontrollen fra brugerne.

\item Apps skal kunne tilpasse sig skærmen de bliver vist på, og blive vist rigtigt i forhold til skærmens position.

\item Apps bliver kun vist en ad gangen, dette betyder at en app skal være i stand til at gå i dvale og genoptage det stadie app'en var da den blev sat i dvale.

\item Der skal fokuseres på det primære formål og opfyldelse af det. App'ens indhold skal være relevant i forhold til brugeren.

\item Minimal hjælpetekst og andet, brugergrænsefladen skal være intuitiv. Logisk opbygning af app'en.

\item Undgå spildtid.

\item App'en skal overholde de opstillede mål i forhold til ikoners størrelse. 
\end{itemize}

Denne liste er en sammenkogning af Apple's retningslinjer, hvor der er lagt vægt på hovedemnerne, da de ses som de mest relevante at arbejde med. Alle Apple's retningslinjer kan ses på deres hjemmeside for udviklere, samt tabellen over ikonstørrelser \citep{AppleRetning}.

Apple prioriter brugeroplevelsen højt, da mange af punkterne fokuser netop på, at det skal være nemt og intuitivt for brugerne at bruge forskellige apps. Dette gøres ved, at fokusere på funktioner i app'en og understøtte disse grafisk. Aldrig bør grafikken overskygge funktionerne da det kan gøre app'en mindre overskuelig i forhold dens formål. Hjælpetekst eller lignende er ikke en fordel, idet app'en så ikke er hensigtsmæssigt konstrueret. 
Ved at anvende metaforer og dertil hørende gestusser til interaktionen med app'en vil app'en virke mere intuitiv for brugeren og derved gør det nemmere at bruge app'en. App'en skal altid hjælpe brugeren, dog aldrig tage kontrollen fra dem eller få dem til at føle sig magtesløse.

Disse forskellige tiltag skulle sikre en mere fornuftigt brugeroplevelse i forbindelse med brugen af en app ifølge Apple. Derfor vil der blive sammenlignet mellem disse opstillede punkter og dette projekts produkt for at se nærmere på, om produktet egentlig opfylder nogle af disse krav og om nogle af punkterne er relevante at rette produktet efter. Systemet var på testtidspunktet i samme tilstand som ved femte test, se sektion \ref{sec:femtetest}. Sammenligningen vil foregå via tabel \ref{tab:RetninglinjerApple}, metoden er den samme som ved sammenligningen mellem produktet og Googles retningslinjer.

\begin{table}[H]
\centering
    \begin{tabular}{|l|c|c|c|}
	\cline{2-4}
	\multicolumn{1}{l|}{}              								& \rotatebox{90}{Taget højde for} 	& \rotatebox{90}{Kunne gøres bedre} & \rotatebox{90}{Ikke taget højde for~}\\
	\hline
	Overensstemmelse       										& x   			  	& ~					& ~     			   		\\ \hline
	Funktionalitet understøttet af grafik  						& x					& ~					& ~							\\ \hline
	Brugen af metaforer											& ~					& x					& ~							\\ \hline
	Brugen af gestus											& x					& ~					& ~							\\ \hline
	Fokusere på primær formål									& x					& ~					& ~							\\ \hline
	Brugeren bestemmer											& ~					& x					& ~							\\ \hline
	Undgå spildtid 												& ~					& x					& ~							\\ \hline
	Overholdelse af design krav (ikon størrelse og lignende)	& ~					& ~					& x							\\ \hline
	Undgå hjælpetekst											& ~					& x					& ~							\\ \hline
	Gem brugerens arbejde og valg								& x					& ~					& ~							\\ \hline
	Logisk opbygning											& ~					& x					& ~							\\ \hline
	Tilbagemeldinger på handlinger								& x					& ~					& ~							\\ \hline
	\end{tabular}
	
\caption{Tabel over sammenligningen mellem produktet og Apples retningslinjer.}
\label{tab:RetninglinjerApple}

\end{table}

På tabel \ref{tab:RetninglinjerApple} ses sammenligning mellem produktet og Apples retningslinjer. Disse er dog retningslinjer for en app, mens der i dette projekt bliver arbejde med en hjemmeside. Derfor bliver punkternes mening overført til hjemmesiden, idet dens opbygning meget ligner en app blot det hele foregår på en hjemmeside.
Kolonnen \gaas{Kunne gøres bedre} bruges til at fortælle om, hvilke af disse punkterne der kunne indfries bedre i forhold til produktet.

\subsubsection{Konsekvenser for system}
Ud fra retningslinjerne fra Apple, er der lavet forskellige rettelser. Den største rettelse var bl.a. som følge af rettelserne fra Android retningslinjer (se afsnit \ref{sec:andretkon}): Det er forsøgt at lave et intuitivt design, og derved undgå brugen af hjælpetekster. Systemet havde før denne test forskellige hjælpebobler, hvor man kunne trykke på den og få information. Dette strider imod retningslinjerne, og derfor blev designet af hjemmesiden gennemgået. En konsekvens af rettelserne fra Apple retningslinjer var, at de to hjælpetekstbobler blev fjernet fra søgefunktionen. Herudover blev markeringen af varer nemmere for brugerne, ved at benytte symboler til at demonstrere, hvordan de markerer varer som købt. Hermed kunne hjælpeteksten også fjernes.

\subsection{Opsamling}
De to selskaber er rimelig enige i, hvordan apps struktur skal designes dog afviger de fra hinanden. Som det kan ses ud fra de to tabeller er det ikke alle punkter som kan blive yderligere optimeret på hjemmesiden, selv med de rettelser der kom ud fra testene. Men de forslag og retningslinjer de to producenter kommer med har stadigvæk givet stof til eftertanke med hensyn til opsætning og valg af effekter på siden, for at gøre den mere mobilvenlig. Hvis produktet senere skal udvikles til også at inkludere en app, er det meste af forarbejdet allerede blevet lavet og det skulle derfor være en nemmere opgave at implementere.

\section{Syvende test}
Her beskrives alt omkring syvende test.

\subsection{Testpersonen}
Testpersonen er en mand på 20 år. Testpersonen har en begrænset økonomi, og er derfor interesseret i at handle billigt. Teknologisk erfaring: Middel. Ejer en smartphone. 

\subsection{Systemets tilstand}
På tidspunktet for syvende test var systemt mere eller mindre færdigudviklet, men manglede stadig at blive testet i en rigtig indkøbssituation. Den syvende test blev netop lavet for at afsløre fejl, som eventuelt kunne opstå når man brugte systemet i et supermarked, hvor forhold for bl.a. data-forbindelsen ikke er optimale.

\subsection{Kvalitetsmål i fokus}
I denne test blev specielt fokuseret på kvalitetsmålene \gaas{brugbart}, \gaas{effektivt} og \gaas{forståeligt}. \gaas{Brugbart} var vigtigt at få testet for at se om programmet kunne bruges i en rigtig handelssituation. Der blev også testet efter om systemet var forståeligt ved at testpersonen ikke kendte systemet, og kun fik en kort introduktion til systemet. Det sidste der blev testet, var om systemet var effektivt at bruge i en rigtig handelssituation.

\subsection{Testmetoden}
Testpersonen fik en kort intro til programmet og en forudlavet indkøbsliste. Derefter gik testpersonen ud og handlede, mens et gruppemedlem observerede hvad testpersonen gjorde og hvad der skete. Derefter blev der holdt et kort interview omkring hvad testpersonen synes om at bruge programmet.

\subsection{Observationer fra syvende test}
Der skete to fejlvisninger i løbet af testen. Ved den ene blev typografiarket ikke loadet ordentligt hvilket gjorde at alt på siden blev vist som tekst. Den anden fejl var at hver gang mobilen blev vendt i landskabstilstand og så vendt til portrættilstand igen kunne siden ikke finde tilbage i de rigtige proportioner igen. Dette er forsøgt løst, men problemet forekommer stadig somme tider på iOS enheder. Bortset fra dette, mente testpersonen at programmet var godt at bruge, når han handlede og at ventetiden på at siden genindlæses efter en vare er blevet markeret, ikke er et problem da han gik et stykke mellem de varer han skulle have, og derfor ikke kom til at vente. Han fandt det dog problematisk hvis siden også skulle genindlæses, hver gang han kom til at vende smartphonen til landskabstilstand. Testpersonen sagde det ville være rart hvis varerne var kategoriseret efter varekategorier, men det ikke var en nødvendighed for programmets brugbarhed.

\subsection{Konsekvenser af syvende test}
Problemstillingen med at programmet ikke kunne finde tilbage til de rigtige proportioner efter at være vendt på siden er forsøgt løst bedst muligt med den tid og ressourcer, der var til rådighed. Den anden visningsfejl skyldes sandsynligvis en kort afbrydelse af data-overførslen af typografiarket og der er derfor ikke blevet rettet noget. Der er desuden blevet lavet så det nu er muligt selv at vælge en varekategori til en given vare og sortere listen efter denne.

\subsection{Afrunding syvende test}
De vigtigste problemstillinger der blev observeret i syvende test var:
\begin{itemize}
\item Ventetiden der blev observeret i test fire var ikke noget problem i en rigtig handelssituation.
\item Systemet virkede efter hensigten med udtagelse af de to visningsfejl.
\end{itemize}
De vigtigste problemstillinger der blev rettet i syvende test var:
\begin{itemize}
\item Den ene visningsfejl blev rettet bedst muligt.
\item Det er nu muligt at vælge en varekategori til varerne og sortere efter denne.
\end{itemize}

\section{Opsummering af testene}

I løbet af projektperioden er der foretaget syv tests. En kort oversigt over de rettelser der blev foretaget på baggrund af disse tests, kan ses på figur \ref{fig:testrettelserdiagram}. Den første var en papirprototype, som blev brugt internt i gruppen, til at lave en skitse til hvordan systemet skulle se ud. Ved anden og tredje test fik testpersonerne sendt et åbent spørgeskema og havde derefter en uge til at teste programmet og besvare spørgeskemaet. Den fjerde test blev foretaget i universitets usability laboratorium og den syvende blev foretaget ude i en rigtig handelssituation. Den femte og sjette test var en gennemgang af systemet for at se om det overholdt Googles og Apples retningslinjer for applikationer til Android og iOS. Alle testene har hjulpet meget med at finde ud af, hvad der var vigtigt for et produkt som det, der er blevet fremstillet og de har givet mange forbedringer til hjemmesiden. Der var flere rettelser efter de to fulde tests af programmet end der var ved den anden og syvende test. Desuden har testene givet et godt indtryk af, hvor tidskrævende de er og hvad udbytte man kan få ud af dem. Her er testen i usability laboratoriet den, der tog længst tid og gav størst udbytte, mens de to tests med åbent spørgeskema brugte mindst og gav et rimeligt udbytte, dog afhængig af hvem der svarede. Femte og sjette test hjalp til at gøre systemet mere mobil-venligt i forhold til Apples og Googles retningslinjer med hensyn til mobil-applikationer. Så alt i alt har testene været utrolig nyttige at udnytte i løbet af udviklingsprocessen for at komme på frem til et godt produkt. Den samlede mængde af test, samt det at der er blevet benyttet forskellige testmetoder, har gjort at programmet er blevet forbedret meget i forhold til hvordan det var stillet uden testene, da det har fået en bedre forståelse af, hvordan brugerne benytter programmet.
