\section{Tredje test}
Herunder kommer alt omkring tredje test. Testpersonerne er de samme som fra anden test, se sektion \ref{sec:andentest}.

\subsection{Systemets tilstand}
På tidspunktet for tredje test var de fire menupunkter på systemets forside blevet udviklet: Søg, indkøbslister, foreslåede tilbud og indstillinger. Hele systemet var altså på dette tidspunkt funktionelt.  

\subsection{Kvalitetsmål i fokus}
Formålet med denne test var at kontrollere systemets brugergrænseflade, dens funktionalitet og brugernes forståelse af systemet. Selve testen var derfor formet ud fra kvalitetsmålene \gaas{brugbart}, \gaas{forståeligt} og \gaas{effektivt}. Et spørgeskema var blevet udformet så spørgsmålene fokuserede på disse punkter.
Kvalitetsmålene er valgt, fordi de er blandt de meget højt prioriterede, med undtagelse af \gaas{effektivt} som kun var prioriteret \gaas{vigtigt}, hvor de to andre var prioriteret \gaas{meget vigtigt}. 

\subsection{Testmetoden}
Selve testmetoden er den samme som i anden test, men hjemmesiden er blevet udviklet en del siden forrige test. Alle hjemmesidens fire funktioner er nu blevet udviklet. Desuden er de blevet bedt om at fortsætte med at bruge hjemmesiden når de handler, også selvom testen er sat til at slutte efter en uge. Mailen og spørgeskemaet der blev sendt til informanterne kan ses i bilag \ref{chap:andetskema}. Testpersonerne blev bedt om, at lave testen sammen med en anden.

\subsection{Testpersonernes svar}
I denne omgang af test svarede tre ud af seks informanter, og heraf svarede to testpersoner sammen. Alle informanterne meldte desværre at der var fejl, der gjorde det umuligt at udføre dele af den første opstillede opgave. Informanterne var meget positive over de forbedringer og ændringer der er sket siden sidste test. Ud fra svarene kan det konstateres at informanterne har set de fleste af forbedringerne. De varer der er blevet forslået ændret er:
At det er muligt at kunne sortere efter kategorier som for eksempel madvarer og byggemarked. Der blev igen foreslået multitilføjelse af varer. Desuden ønskes der en standardindkøbsliste og standardvarer så det er nemmere at lave en basis indkøbsliste. Ud over dette blev der foreslået en GPS-funktion, der kunne beskrive eller vise hvor de enkelte butikker rent fysisk befandt sig.

\subsection{Konsekvenser af tredje test}
Systemet blev rettet således det blev operationsdygtigt igen og det er muligt at bruge systemets funktioner. Et af ønskerne var at der var en standardliste med standardvarer. Dette blev prioriteret værende mindre vigtigt. Baggrunden for dette er at denne egenskab ikke er nødvendig for systemets funktionalitet.
\gaas{Tilbage-knappen} som blev omtalt i anden test er blevet lavet om til at være en knap, der sender brugeren til oversigten over indkøbslister. Dette gøres fordi \gaas{tilbage-knappen} aldrig kom til at fungere som forventet og derfor passer denne løsning bedre til det tidligere ønske informanterne havde.
Forslaget med GPS-funktion anses som værende for komplekst og tidskrævende, at anvende i den givne løsning, men kunne være relevant at overveje, hvis systemet engang skal eksporters til en app.
Systemet har nu også en fungerende stavekontrol i søgningen således at brugerne får forslag på forkert stavede ord.

\subsection{Afrunding tredje test}
De vigtigste problemstillinger der blev observeret i tredje test var:
\begin{itemize}
\item Der var to fejl, der gjorde at to af funktionerne på siden ikke kunne bruges.
\item Der blev foreslået at varerne kunne sorteres efter varekategori.
\end{itemize}
De vigtigste problemstillinger der blev rettet i tredje test var:
\begin{itemize}
\item Fejlene blev rettet, så siden igen blev funktionsdygtigt.
\end{itemize}
Disse var de vigtigste, da de gjorde dele af siden ubrugelig. 