\section{Android og Apple}
Det er interessant at se nærmere på de større selskabers retningslinjer for app udvikling, idet systemet udvikles til at være mobiloptimeret. Derfor er det blevet valgt at se nærmere på Android's og Apple's sæt af retningslinjer for udvikling af apps. Selvom det er en hjemmeside dette projekt omhandler, er det også meningen at produktet skal bruges på en mobil enhed. Derfor kan disse retningslinjer være et godt udgangspunkt at designe websiden ud fra, da det vil give en bedre oplevelse på mobile enheder og gøre det nemmere at lave en decideret app i fremtiden.

\subsection{Kvalitetsmål i fokus}
Ved både test fem og seks blev der lagt fokus på kvalitetsmålene \gaas{brugbart} og \gaas{forståeligt}. Disse kvalitetsmål blev markeret som \gaas{meget vigtigt}, og dermed var der også lagt stor fokus på dette. Dette er blevet gjort ved at bruge nogle allerede udviklede retningslinjer til at teste disse kvalitetsmål.

\subsubsection{Systemets tilstand}
Systemets tilstand under disse to tests, er at alle kernefunktioner i systemet er færdigudviklede, og systemet er funktionsdygtigt. 


\subsection{Femte test - Androids retningslinjer}
\label{sec:femtetest}
I dette afsnit vil de retningslinjer, som Google anbefaler man følger for at lave en Android applikation i særklasse, blive undersøgt.
Informationerne er taget fra Android Developers \citep{AndRet}.

Vejledningen er blevet delt op i tre emner:
\subsubsection*{Enchant me}
Som udgangspunkt, det brugeren ser og føler når de bruger app'en.

Udseende er ikke alt, små effekter der får app'en til at føles mere interaktiv og giver en følelse af respons, giver meget til oplevelsen. Det at flytte eller interagere med objekter frem for knapper kan også være med til at gøre oplevelsen meget sjovere. Ved at kunne ændre på udseende, gives brugeren en følelse af at app'en tilhører dem og at det er dem der bestemmer. Hvis app'en også husker på tidligere valg, så giver det følelsen af at app'en lærer en at kende, hvilket i sidste ende gør brugen af app'en nemmere og hurtigere.

\subsubsection*{Simplify my life}
Hvor nemt og tilfredsstillende det er at bruge app'en.

Simpelt og stilrent er essensen af et godt design. Jo mindre det er nødvendigt at skrive, des bedre, og hvis det kan erstattes af et billede, er det endnu bedre. Det er også meget af tiden bedre bare at komme med et bud, på hvad brugeren vil have, frem for at spørge dem om for meget. For meget information på én gang kan være med til at forvirre, frem for at hjælpe, så vis kun hvad der er nødvendigt, så som hvor i systemet brugerne befinder sig, så de får følelsen af at vide, hvad de laver. Hvis brugeren har brugt tid på at lave noget, så er det vigtigt at det kan gemmes, så de kan tage det med sig videre på f.eks. andre enheder eller hvis app'en skal opgraderes. Hold konsistens over hele app'en, hvis noget ligner hinanden så skal de også gøre det samme ellers skabes der forvirring. Brugere skal kun afbrydes fra det de laver hvis det, der gives besked om er meget vigtig, da brugere godt kan lide at holde fokus på det, de laver.
\subsubsection*{Make me amazing}
Generelt evnen til at få brugeren til at føle sig klog og smart.

Brugere elsker når de finder ud af noget selv, så det bør prøves at implementere funktioner og udseende fra Android apps, derved får de en positiv overraskelse når funktionerne fungerer, som de regner med. Brugere vil ikke føle sig dum når de bruger app'en, så hvis noget går galt, vær sikker på at give dem simple instrukser og spar dem for lange forklaringer. Meget svære eller avancerede opgaver skal helst deles så meget op som de kan for at gøre det nemmere for brugerne. Det er vigtigt at give en form for respons selvom det måske ikke er mere end et svagt lys bag et ikon. Gør det nemt for begyndere at føle sig som eksperter, som f.eks ved at sammensætte nogle filtre og indstillinger til én knap i et fotografiapparat, så de nemt kan tilgås. Vær opmærksom på at fremhæve vigtige knapper og funktioner, så som knappen til at tage et billede med er mere fremhævet end knapperne til at indstille effekter.

\subsubsection{Sammenligning}
I dette afsnit er der vist en tabel, som sammenligner dette projekts produkt og Androids retningslinjer, som kan ses på tabel \ref{tab:RetninglinjerAndriod}. Systemet havde på dette tidspunkt alle de endelige funktionaliteter, med undtagelse af muligheden for at sortere efter varekategori. Det er gjort for at give et indtryk af, hvordan produktet passer på de retningslinjer som det anbefales at Android produkter lever op til for at give den bedste oplevelse. 

\begin{table}[H]
\centering
    \begin{tabular}{|l|c|c|c|}
    \cline{2-4}
\multicolumn{1}{l|}{}       & \rotatebox{90}{Taget højde for} 	& \rotatebox{90}{Kunne gøres bedre}& \rotatebox{90}{Ikke taget højde for~}	\\ 
\hline
Interaktivt design	       	& ~    			  	& x					& ~						\\ \hline
Objekter frem for knapper  	& ~					& ~					& x						\\ \hline
Personificering				& ~					& x					& ~						\\ \hline
Lær mig at kende			& x					& ~					& ~						\\ \hline
Hold det simpelt			& ~					& x					& ~						\\ \hline
Billeder frem for ord		& ~					& ~					& x						\\ \hline
Vælg for mig,(jeg bestemmer)& x					& ~					& ~						\\ \hline
Vis kun det nødvendige		& x					& ~					& ~						\\ \hline
Hvor er jeg på siden?		& x					& ~					& ~						\\ \hline
Gem mine ting				& x					& ~					& ~						\\ \hline
Konsistens					& ~					& x					& ~						\\ \hline
Notifikation, kun vigtigt	& x					& ~					& ~						\\ \hline
Genbrug kendte effekter		& ~					& ~					& x						\\ \hline
Simpel fejlfinding			& ~					& x					& ~						\\ \hline
Enkle opgaver				& x					& ~					& ~						\\ \hline
Gør det tunge arbejde		& x					& ~					& ~						\\ \hline
Fremhæv vigtige ting		& x					& ~					& ~						\\ \hline  

\end{tabular}

\caption{Tabel over sammenligningen mellem produktet og Andriods retningslinjer.}
\label{tab:RetninglinjerAndriod}

\end{table}
%ret herfra
\subsubsection{Konsekvenser for systemet}
\label{sec:andretkon}
Ud fra undersøgelsen af retningslinjerne, er der lavet nogle ændringer ved systemet, for at optimere det til den bedst mulige brugeroplevelse. For at forbedre punktet med at holde det simpelt er det forsøgt at simplificere nogle af de forskellige dele af hjemmesiden, specielt metoden til at søge. Her var der før to søgeknapper: en til filtreret søgning og en anden til fuld søgning. Den filtrerede søgning forsøgte at gøre resultaterne mere relevante i forhold til søgeordet og varer fra fravalgt butikker blev ikke vist. Den fulde søgning tog alt hvad der kom frem. Det blev vurderet, at man altid ville have relevante søgeresultater, så derfor blev det for simpelhedens skyld valgt at fjerne knappen til fuld søgning.
\subsection{Sjette test - Retningslinjer for Apples apps}
I dette afsnit vil der blive fokuseret på Apples sæt af retningslinjer og hvorledes dette projekts produkt følger nogle af deres principper og retningslinjer \citep{AppleRetning}. 

\begin{itemize}
\item Æstetikken og den grafiske kvalitet skal være høj, da der bliver fokuseret på brugeroplevelsen.
\begin{itemize}
	\item Der skal være overenstemmelse.
	\item Systemets funktioner skal fremhæves og understøttes af grafik.
	\item Direkte interaktion, og tilbagemelding på brugerens handling.
\end{itemize}

\item Brug metaforer i forbindelse med interaktion.
	
\item Apps skal reagere på forskellige gestus som let berøring, flik, dobbelt berøring og andre gestus.

\item Apps må ikke tage kontrollen fra brugerne.

\item Apps skal kunne tilpasse sig skærmen de bliver vist på, og blive vist rigtigt i forhold til skærmens position.

\item Apps bliver kun vist en ad gangen, dette betyder at en app skal være i stand til at gå i dvale og genoptage det stadie app'en var da den blev sat i dvale.

\item Der skal fokuseres på det primære formål og opfyldelse af det. App'ens indhold skal være relevant i forhold til brugeren.

\item Minimal hjælpetekst og andet, brugergrænsefladen skal være intuitiv. Logisk opbygning af app'en.

\item Undgå spildtid.

\item App'en skal overholde de opstillede mål i forhold til ikoners størrelse. 
\end{itemize}

Denne liste er en sammenkogning af Apple's retningslinjer, hvor der er lagt vægt på hovedemnerne, da de ses som de mest relevante at arbejde med. Alle Apple's retningslinjer kan ses på deres hjemmeside for udviklere, samt tabellen over ikonstørrelser \citep{AppleRetning}.

Apple prioriter brugeroplevelsen højt, da mange af punkterne fokuser netop på, at det skal være nemt og intuitivt for brugerne at bruge forskellige apps. Dette gøres ved, at fokusere på funktioner i app'en og understøtte disse grafisk. Aldrig bør grafikken overskygge funktionerne da det kan gøre app'en mindre overskuelig i forhold dens formål. Hjælpetekst eller lignende er ikke en fordel, idet app'en så ikke er hensigtsmæssigt konstrueret. 
Ved at anvende metaforer og dertil hørende gestusser til interaktionen med app'en vil app'en virke mere intuitiv for brugeren og derved gør det nemmere at bruge app'en. App'en skal altid hjælpe brugeren, dog aldrig tage kontrollen fra dem eller få dem til at føle sig magtesløse.

Disse forskellige tiltag skulle sikre en mere fornuftigt brugeroplevelse i forbindelse med brugen af en app ifølge Apple. Derfor vil der blive sammenlignet mellem disse opstillede punkter og dette projekts produkt for at se nærmere på, om produktet egentlig opfylder nogle af disse krav og om nogle af punkterne er relevante at rette produktet efter. Systemet var på testtidspunktet i samme tilstand som ved femte test, se sektion \ref{sec:femtetest}. Sammenligningen vil foregå via tabel \ref{tab:RetninglinjerApple}, metoden er den samme som ved sammenligningen mellem produktet og Googles retningslinjer.

\begin{table}[H]
\centering
    \begin{tabular}{|l|c|c|c|}
	\cline{2-4}
	\multicolumn{1}{l|}{}              								& \rotatebox{90}{Taget højde for} 	& \rotatebox{90}{Kunne gøres bedre} & \rotatebox{90}{Ikke taget højde for~}\\
	\hline
	Overensstemmelse       										& x   			  	& ~					& ~     			   		\\ \hline
	Funktionalitet understøttet af grafik  						& x					& ~					& ~							\\ \hline
	Brugen af metaforer											& ~					& x					& ~							\\ \hline
	Brugen af gestus											& x					& ~					& ~							\\ \hline
	Fokusere på primær formål									& x					& ~					& ~							\\ \hline
	Brugeren bestemmer											& ~					& x					& ~							\\ \hline
	Undgå spildtid 												& ~					& x					& ~							\\ \hline
	Overholdelse af design krav (ikon størrelse og lignende)	& ~					& ~					& x							\\ \hline
	Undgå hjælpetekst											& ~					& x					& ~							\\ \hline
	Gem brugerens arbejde og valg								& x					& ~					& ~							\\ \hline
	Logisk opbygning											& ~					& x					& ~							\\ \hline
	Tilbagemeldinger på handlinger								& x					& ~					& ~							\\ \hline
	\end{tabular}
	
\caption{Tabel over sammenligningen mellem produktet og Apples retningslinjer.}
\label{tab:RetninglinjerApple}

\end{table}

På tabel \ref{tab:RetninglinjerApple} ses sammenligning mellem produktet og Apples retningslinjer. Disse er dog retningslinjer for en app, mens der i dette projekt bliver arbejde med en hjemmeside. Derfor bliver punkternes mening overført til hjemmesiden, idet dens opbygning meget ligner en app blot det hele foregår på en hjemmeside.
Kolonnen \gaas{Kunne gøres bedre} bruges til at fortælle om, hvilke af disse punkterne der kunne indfries bedre i forhold til produktet.

\subsubsection{Konsekvenser for system}
Ud fra retningslinjerne fra Apple, er der lavet forskellige rettelser. Den største rettelse var bl.a. som følge af rettelserne fra Android retningslinjer (se afsnit \ref{sec:andretkon}): Det er forsøgt at lave et intuitivt design, og derved undgå brugen af hjælpetekster. Systemet havde før denne test forskellige hjælpebobler, hvor man kunne trykke på den og få information. Dette strider imod retningslinjerne, og derfor blev designet af hjemmesiden gennemgået. En konsekvens af rettelserne fra Apple retningslinjer var, at de to hjælpetekstbobler blev fjernet fra søgefunktionen. Herudover blev markeringen af varer nemmere for brugerne, ved at benytte symboler til at demonstrere, hvordan de markerer varer som købt. Hermed kunne hjælpeteksten også fjernes.

\subsection{Opsamling}
De to selskaber er rimelig enige i, hvordan apps struktur skal designes dog afviger de fra hinanden. Som det kan ses ud fra de to tabeller er det ikke alle punkter som kan blive yderligere optimeret på hjemmesiden, selv med de rettelser der kom ud fra testene. Men de forslag og retningslinjer de to producenter kommer med har stadigvæk givet stof til eftertanke med hensyn til opsætning og valg af effekter på siden, for at gøre den mere mobilvenlig. Hvis produktet senere skal udvikles til også at inkludere en app, er det meste af forarbejdet allerede blevet lavet og det skulle derfor være en nemmere opgave at implementere.