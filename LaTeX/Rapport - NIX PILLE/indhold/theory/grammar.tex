\section{Grammar}
\label{sec:grammar}
A grammar is used to define the syntax of a language. A context-free grammar (CFG) is a 4-tuple $(V, \Sigma, R, S)$ finite language defined by \citep{sipser}:
\begin{enumerate}
\item $V$ is a finite set called the variables
\item $\Sigma$ is a finite set, disjoint from V called the terminals
\item $R$ is a finite set of rules, with each rule being a variable and a string or variables and terminals
\item $S: S \in V$ is a start variable
\end{enumerate}

The most common way of writing a CFG is by using Backus Naur Form (BNF) or Extended Backus Naur Form (EBNF). BNF is named after John Backus who presented the notation, and Peter Naur who modified Backus' method of notation slightly \citep{sebesta}. By using the BNF-notation it is possible to describe a CFG. It is preferred to have a unambiguously grammar. A CFG is ambiguously if a string derived in the grammar has two or more different leftmost derivations \citep{sipser}. An unambiguously grammar will ensure that a program running through a string using CFG can only read the string in one way.

A CFG is a part of the $LL(k)$ grammar classes if it is possible to produce the leftmost derivation of a string by looking at most $k$ tokens ahead in the string. $LL$ algorithms works on the same subset of free grammes which means that $LL$ parsers works on $LL(k)$ grammars. $LL(k)$ means that the grammar needs to be left-recursive free which makes it possible to create a top-down leftmost derivation parser.
The $LL(1)$ have proprieties that makes the grammar attractive for simple compiler construction. A propriety is that $LL(1)$ grammars are fairly easy compare to $LL(k) where k > 1$ to implement because the parser analyser only have to look one element ahead in order to determine what parser action there should be taken. $LL(1)$ is also relatively faster than $LL(k) where k > 1$ based on the same reason, that the parser only have to look one element ahead. A disadvantage of the $LL$ grammars is that the parser finds syntax errors towards the end of parsing process where a $LR$ parser is faster at detecting the syntax errors. $LL$ is also inferior compare to $LR$ in terms of describing a languages based on the idea that $LL$ is a subclass of the bigger grammar class $LR$. That means with a $LR$ grammar it is possible to describe aspects of a language that might not been possible in a $LL$ grammar \citep{CraftingACompiler} \citep{sebesta}.

A CFG is a part of the $LR(k)$ grammar classes if it is possible to produce the rightmost derivation in reverse of a string by looking at most $k$ tokens ahead in the string. $LR$ grammars are a superset for the $LL$ grammars meaning that $LR$ covers a larger variety of programming language that $LL$. $LR$ parser is a bottom-up parser meaning that it start constructing the abstract trees from its leaf and works its way to the root. $LR$ parsers are general harder to implement than $LL$ parsers by hand but there exists tools that automatic can generate $LR$ parsers. $LR(k)$ grammars allows left recursion which means that the $LR$ grammars are a bigger grammar class than $LL$. $LALR$ and $SLAR$ is subclasses of the $LR(k)$ grammars which means that $LR(k)$ describes a larger language at the cost of a bigger parser table in comparison to $SLAR$ and $LALR$. The balance of power and efficiency makes the $LALR(1)$ a popular table building method compare to $LR$ building method \citep{CraftingACompiler} \citep{sebesta}.

Based of these understandings of grammars there will be a section were there will looked into which grammar that will be used in this project.