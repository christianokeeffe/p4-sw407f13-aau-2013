This section contains a debate about the goals of this project, and whether if they have been met. There is also a section which contains the reflections about the project.

The idea of this project is that hobbyist programmers can program their own drink machine based on the Arduino platform. The purpose of the SPLAD language is to provide a simple alternative to the Arudino C/C++ like language, when programming a drinks machine. In the first place this was intended to be used by the owners of small bars, who also have a hobby of programming. Using the SPLAD language, they will be able to build their own drinks machines and program them to their liking.

The reason why bar owners would want this solution is that it could lessen the work load of the bartender, by replacing some of his or her tasks. For certain drinks, the bartender could simply sell a pre-programmed RFID-tag, which could be used on the drinks machine, and the bartender could then serve more customers. This is of course an advantage to both the bartenders and the bar owners. The bartenders can focus on other tasks than mixing the drinks, and for the bar owners, it would be like having an extra bartender without having to pay the extra wages. SPLAD could make this possible to a group of people, who would not have been able to make such a program otherwise.

\section{Characteristic of SPLAD}
This section compares the design criteria set forward in section \ref{sec:DesignCriteria}, and checks if these criteria have actually been met.

"Simplicity" was set to be high for SPLAD. This, as explained in section \ref{sec:simplespladc}, has been acceptably met. 

\section{Simplicity of SPLAD Compared with C}
\label{sec:simplespladc}
This section will compare pieces of code written in SPLAD, with pieces of code written in C. The first code example, seen on listing \ref{lst:helloworldsplad}, is an example of a simple SPLAD program, which prints "HelloWorld" on the LCD.

\begin{code}{helloworldsplad}{Hello world program in SPLAD.}
\begin{lstlisting}
function pour return nothing using(int a, double b)
begin
	return nothing;
end

function RFIDFound return nothing using(int a, int b)
begin
	string message <-- "Hello World!";
	/* Print message on line 1 on LCD */
	call LCDPrint(message, 1);

	return nothing;
end
\end{lstlisting}
\end{code}

Lines 1-4 on listing \ref{lst:helloworldsplad} contain the pour function, which is required by the SPLAD compiler, but is irrelevant in this small example. The string "message" is declared on line 7, which is the string that will be printed to the LCD. On line 9, the function LCDPrint, which is provided by the SPLAD compiler is called, and the message is printed on the display. The message will be printed when an RFID-tag is found by the RFID-reader.

A "Hello World" program written in Arudino C/C++, can be seen on listing \ref{lst:helloworldc}. It should be noted that this is an example provided by the Arduino IDE \citep{LCDtut}. Line 2 of this example includes the LCD Library. On line 5 the LCD display must be initialized with the pins on the Arduino. On line 9, the LCD display is setup, which means that the appropriate number of columns and rows are set depending on the particular model used. On line 11, the message is printed on the LCD.
 
\begin{code}{helloworldc}{"Hello World" program in Arduino C/C++.}
\begin{lstlisting}
// include the library code:
#include <LiquidCrystal.h>

// initialize the library with the numbers of the interface pins
LiquidCrystal lcd(12, 11, 5, 4, 3, 2);

void setup() {
  // set up the LCD's number of columns and rows: 
  lcd.begin(16, 2);
  char message[ ] = "Hello World";
  // Print a message to the LCD.
  lcd.print(message);
}

void loop() {

}
\end{lstlisting}
\end{code}
The difference between the SPLAD program, and the Arduino program is clear: The SPLAD program is more simple, which means that there is an LCD print function provided by default. This is not the case in a normal Arduino program, because Arduino is aimed at a much more general purpose. This means that users of SPLAD do not need to think about which pins the LCD is connected to, while writing their programs. Arduino does not provide a string-type, which means that strings are implemented as char arrays. In SPLAD there is a string type, which might seem more intuitive for novice programmers. The assignment is a bit different in SPLAD compared to C-notation. In SPLAD an assignment is denoted by '<-$\,$-', which makes it completely clear what is assigned to what. In C, the assignment is denoted by '=', which might confuse novice programmers, because '=' generally is used to denote equality in for example mathematics. When using the '=', it can be unclear what is assigned to what, because there are nothing indicating the direction.

\section{Other Criteria}

"Orthogonality" was set to medium for SPLAD, since it is not possible to create classes or custom constructs in our SPLAD. But it is possible to call a function in an expression and to define functions. Therefore, this criteria has also been met to an acceptable extend.

"Data types" was set to medium for SPLAD. Because SPLAD only has five primitive types and two special types, which makes it simple to keep an overview of them. On top of that, it is not possible to make classes. Therefore, the criteria about data types is met.

"Syntax design" was set to be high for SPLAD. Since most of SPLAD has been created in a way to give a better understanding of the different constructs, such as for-loops, and beginning and endings of block statements. Therefore, this criteria has been achieved in SPLAD.

"Support for abstraction" was set to be medium for SPLAD, because SPLAD is an abstraction of the Arduino programming language. Examples of abstraction can be seen in section \ref{sec:simplespladc}. This abstraction makes programming drink machines, based on the Arduino platform, easier.

The criteria "Expressivity" was set to low for SPLAD. Since SPLAD has a focus to be as simple as possible, the expressivity is not in focus for SPLAD. 

"Type checking" was set to high for SPLAD. To make sure that all type errors that could be found, are found when a program is compiled, a lot of work and tests have been made while creating the type checker. There were some problems, such as the build-in functions that an Arduino program has access to. This makes it hard to make a type checker that will find all errors. All types, expressions, parameters which are in SPLAD are type-checked, therefore this criteria is met. 