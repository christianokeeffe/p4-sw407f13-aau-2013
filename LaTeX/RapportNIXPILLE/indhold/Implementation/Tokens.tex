\subsubsection{Tokens}
For a compiler to be able to distinguish between variables names and for example types the compiler will need some rules to describe the difference between them. This is done by reserving words, called keywords, which are used to describe types, the beginnings and endings of blocks, and declaration of statements and more. A variable may not be named the same as any of the keywords, since the compiler cannot distinguish if it is a variable name or a reserved keyword.

\subsubsubsection{Reserved Keywords}
The reserved keywords for SPLAD can be seen on table \ref{tab:keywords}.

\begin{table}[H]
	\begin{tabular}{|l|l|l|}
		\hline
		function	&	switch		&	bool	    \\ \hline
		return 		&	case		&	char	    \\ \hline
		using 		&	default		&	string	    \\ \hline
		begin 		&	break	    &	nothing    	\\ \hline
		end 		&	AND		    &	container	\\ \hline
		call 		&	OR			&	drink	    \\ \hline
		from     	&	true		&	is	        \\ \hline
		to         	&	false		&	as    	    \\ \hline
		step 		&	HIGH		&	but    	    \\ \hline
		while 		&	LOW			&	of		    \\ \hline
		if 	    	&	int			&	add	        \\ \hline
		else 		&	double		&	remove	    \\ \hline
	\end{tabular}
	\caption{The reserved keywords in SPLAD.}
	\label{tab:keywords}
\end{table}

This list is used to keep track of which words are going to be reserved and in that way provide an overview for the programmer. 

\subsubsubsection{Token Specification}
A parser needs a stream of tokens to parse a program correctly. These tokens are generated by a lexer which reads a stream of input symbols and from a given set of rules, and makes the corresponding tokens. A token specification is used to describe the rules the lexer needs in the construction of tokens. Token specifications are expressed in a way related to regular expressions \citep{sebesta}. Regular expressions are strong in describing patterns which is the core of token production \citep{sipser}. The tokens used for this project can be seen on table \ref{tab:tokens}.

\begin{table}[H]
\begin{tabular}{|l|l|}
\hline
STRINGTOKEN & '' .*? '' \\
LETTER & [a - zA - Z]$~^+$\\
DIGIT & [0 - 9]$~^+$\\
NOTZERODIGIT & [1-9][0-9]$~^*$\\
COMMENT& /* .*? */\\
WHITESPACE 		& \textbackslash r | \textbackslash n | \textbackslash t \\
OTHER 			& $\varepsilon$ \\ \hline
\end{tabular}
\caption{The tokens in SPLAD.}
\label{tab:tokens}
\end{table}