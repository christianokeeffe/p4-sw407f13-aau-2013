\subsection{Solution in Bars}
\label{SolutionInBars}
Currently bars and clubs often have multiple bartenders who mix the drinks and serve the customers. The bartender handles both receiving the order, the mixing the drink and the payment for the drink. This process can be done more efficiently. If a bar had a drinks-machine, the bar would require fewer bartenders instead of multiple bartenders. The cashier would handle selling and programming the RFID-tags. The customers themselves then place the RFID-tag on the RFID-reader on the drinks-machine. The machine mixes the appropriate drink, and either decreases the count on the RFID-tag. It then displays an appropriate message to the customer on the display of the machine. The RFID-tags are programmed by the cashier who encodes a drink id and a drink count onto the tag. This allows the customer to buy for example 10 mojitos on the same tag. The machine will simply check if the drink-count on the tag is above zero, and display an error if it is not.

The system can also be used in many other systems, such as an ice-cream machine, juice machine, or even a food dispenser in a restaurant to help with self-service. It can also be used in a cinema for regular customers to get popcorn, coke, or other forms of sweets for the movies. There are many possibilities on how to use such a system, but in this project a drink-mixer will be the focus.