\subsubsection{Grammar - Types}
This section contains the grammar related to types. This can be seen on grammar \ref{gra:Types}.
\begin{grammatik}{Types}{Grammar for types.}
<program> $\ra$ <roots>

<roots> $\ra$ $\eps$
\alt <root> <roots>

<root> $\ra$ <dcl>;
\alt <assign>;
\alt <function>
\alt <COMMENT>
\alt <drinkdcl>

<dcl> $\ra$ <type> <assign>

<type> $\ra$ <constant> <primitivetype>
\alt <specialtype>

<constant> $\ra$ const
\alt $\eps$

<primitivetype> $\ra$ bool
\alt double
\alt int
\alt char
\alt string

<specialtype> $\ra$ drink 
\alt container 

<id> $\ra$ <LETTER>
\end{grammatik}

\subsubsection{Grammar - Drinks}
The grammar for the drinks is described in section \ref{gra:drinkgram}.

\subsubsection{Grammar - Assign}
This section contains the grammar related to assignments. This can be seen on grammar \ref{gra:assign}.
\begin{grammatik}{assign}{Grammar related to assignments.}
<assign> $\ra$ <callid> <assignend>

<assignend> $\ra$ "<""-""-" <expr>
\alt $\eps$

<expr> $\ra$ <term> <exprend>

<term> $\ra$ <comp> <termend>

<comp> $\ra$ <addsub> <compend>

<addsub> $\ra$ <muldiv> <addsubend>

<muldiv> $\ra$ <factor> <muldivend>

<factor> $\ra$ ( <expr> )
	\alt !(<expr>)
	\alt <callid>
	\alt <numeric>
	\alt A <DIGIT>
	\alt <string>
	\alt <functioncall>
	\alt <cast>
	\alt LOW
	\alt HIGH
	\alt true
	\alt false
	\alt INPUT 
	\alt OUTPUT 

<callid> $\ra$ <id> <arrayidend>

<arrayidend> $\ra$ <arraycall> [ <expr> ]
\alt $\eps$

<arraycall> $\ra$   [ <expr> ] <arraycall>
\alt  [ ] <arraycall>
\alt $\eps$

<numeric> $\ra$ <plusminusorempty> <DIGIT> <numericend>
\end{grammatik}

\subsubsection{Grammar - Expressions}
This section contains the grammar related to expressions. This can be seen on grammar \ref{gra:expr}.
\begin{grammatik}{expr}{The grammar related to expressions.}
<addsubend> $\ra$ <plusminus> <addsub>
\alt $\eps$

<muldivend> $\ra$ <timesdivide> <muldiv>
\alt $\eps$

<timesdivide> $\ra$ *
\alt /

<plusminusorempty> $\ra$ $\eps$
\alt <plusminus>

<plusminus> $\ra$ - 
\alt +

<numericend> $\ra$ $\eps$
\alt . <DIGIT>

<string> $\ra$ <STRINGTOKEN>

<functioncall> $\ra$ call <id> (<callexpr>)

<callexpr> $\ra$ <subcallexpr>
\alt$\eps$

<subcallexpr> $\ra$ <expr> <subcallexprend>

<subcallexprend> $\ra$ , <subcallexpr>
\alt$\eps$

<compend> $\ra$ <comparisonoperator> <comp>
\alt$\eps$

<comparisonoperator> $\ra$ ">"
				\alt "<"
				\alt "<="
				\alt ">="
				\alt "!="
				\alt "="

<termend> $\ra$ <termsymbol> <term>
\alt$\eps$

<termsymbol> $\ra$ AND 

<exprend> $\ra$ <exprsymbol> <expr>
\alt$\eps$

<exprsymbol> $\ra$ OR 
\end{grammatik}

\subsubsection{Grammar - Functions}
This section contains the grammar related to functions. This can be seen on grammar \ref{gra:func}.
\begin{grammatik}{func}{The grammar related to functions.}
<cast> $\ra$ <type> (<expr>)

<function> $\ra$ function <id> return <functionmid>

<functionmid> $\ra$ <type> <functionend> <expr>; end
\alt nothing <functionend> nothing; end

<functionend> $\ra$
using (<params>)
begin
	<stmts>
	return

<params> $\ra$ <subparams>
	\alt$\eps$

<subparams> $\ra$ <type> <callid> <subparamsend>

<subparamsend> $\ra$ , <subparams>
\alt$\eps$

<stmts> $\ra$ $\eps$
	\alt <stmt> <stmts>

<stmt> $\ra$ <assign>;
	\alt <nontermif>
	\alt <nontermwhile>
	\alt <from>
	\alt <dcl>;
	\alt <functioncall>;
	\alt <nontermswitch>
	\alt <COMMENT>
\end{grammatik}

\subsubsection{Grammar - Loops}
This section contains the grammar related to loops. This can be seen on grammar \ref{gra:loop}.
\begin{grammatik}{loop}{The grammar related to loops.}
<nontermif> $\ra$ if(<expr>)
	<block>
	<endif>

<endif> $\ra$ 
	else <nontermelse>
	\alt$\eps$

<nontermelse> $\ra$ <nontermif>
	\alt <block>

<nontermwhile> $\ra$ while(<expr>)
		<block>
		
<from> $\ra$ from <assign> to <expr> step <plusminusorempty> <DIGIT>
	<block>

<block> $\ra$ 
	begin
		<stmts>
	end

<nontermswitch> $\ra$ switch (<expr>)
		begin
			<cases>
		end

<cases> $\ra$ case <expr>:
			<stmts>
			break; 
		<endcase>
		
<endcase> $\ra$ <cases>
		\alt default:
			<stmts>
			break;

\end{grammatik}

\subsection{Lexicon}
The lexicon is used by \textit{ANTLR}. In the lexicon tokens are written with capital letters. The tokens are expressed with regular expressions, where "." means any character. This can be seen on \ref{gra:lexicon}
\begin{grammatik}{lexicon}{The grammar for the lexicon.}
<STRINGTOKEN> $\ra$ '' .*? '' 

<LETTER> $\ra$ [a - zA - Z]$~^+$

<DIGIT> $\ra$ [0 - 9]$~^+$

<NOTZERODIGIT> $\ra$ [1-9][0-9]$~^*$

<COMMENT> $\ra$ /* .*? */


\end{grammatik}