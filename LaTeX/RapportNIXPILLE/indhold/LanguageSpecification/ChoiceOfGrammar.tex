\subsection{Choice of Grammar}
\label{sec:grammarchoice}
The programmer, using the language of this project, could be a hobby programmer, who wants to program a custom drinks machine, but does not possess a high level of experience in programming. This is the reason why it was decided that the grammar should have a high level of readability because this in turn will ensure that it is easier for the programmers to read and understand their programs - this is also useful if the code has to be maintained later on. This on the other hand can decrease the level of write-ability because the programs have to be written in a specific way, and will need to contain some overhead in form of extra words and symbols to mimic a language easier for humans to comprehend.

The method to assign a value to a variable is by typing "$variable$ <-- 'value to assign'", without the quotes. This approach has been chosen instead of the more commonly used "$=$" symbol, because a person not accustomed to programming might confuse which side of the "$=$" is assigned to the other. Thus by using the arrow, it is clearly indicated that the value is assigned to the variable, and therefore ensuring readability - especially for the hobby programmer.

When declaring a function it has to be written on the form "function $functionname$ return $type$ using ($parameter(s)$) begin $statements$ return $expresion$ end". Where $function$-$name$ is the name of the function that is about to be declared, $type$ is the type of value that is returned by the function. $Parameter(s)$ are used to parse the function some input values from its call(s). $Statements$ is where the function can call other functions, declare variables, calculate and assign values. $Expresion$ is where the value of the right type is returned or an expression which result is of the correct type.
An example of this can be seen on listing \ref{lst:functiondeclare}.

\begin{code}{functiondeclare}{Example of function declaration in SPLAD.}
	\begin{lstlisting}
		function DoSomething return int using (int x)
		begin
			x <-- x + 1;
			return x;
		end
	\end{lstlisting}
\end{code}

To get a more continuity structure in the code the functions must always return something, but it can return the value "nothing". This will ensure a better understanding and readability of the code because the programmer can see what it returns, even if no value was parsed. To indicate that $return$ is the last thing that will be executed in a function, the $return$ must always be at the end of a function. To indicate that a function is called "call $functionname$" must be written.
Words are used instead of symbols, when suitable, to improve the understanding of the program(compared to most other programming languages).
"begin" and "end" are used to indicate a block (eg. an "if" statement). To combine logical operators the words "AND" and "OR" are used. The ";" symbol is used to improve readability by making it easier to see when the end of a statement has been reached.

It would be appropriate to design a grammar that is a subset of $LL(1)$ grammars. This is based on the idea that it easier to implement a parser for $LL(1)$ grammars by hand compared to $LR$ grammars \citep{CraftingACompiler}. This approach means it would be possible to both implement a parser by hand or use some of the already existing tools. This way both approaches are possible which is a suitable solution for the project because it allows the project group to later go back and make the parser by hand instead of using a parser generator if so desired.

If the purpose was to create an efficient compiler it would be more appropriate to design the grammar as a subset of the $LALR$ grammar class. A parser for $LALR$ is balanced between power and efficiency which makes it more desirable than $LL$ and other $LR$ grammars, see section \ref{sec:grammar} for more on the grammars. %$LR$ parsers can be made by hand but it is much more difficult than the $LL$ parsers. 
%den sætning der er udkommenteret, bliver det ikke allerede skrevet i forrige afsnit?