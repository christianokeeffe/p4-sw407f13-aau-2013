\subsubsection{Design Criteria in This Project}
To determine how a programming language should be syntactically described, the trade-offs of designing a programming language must be taken into account. The different characteristics of a programming language, that will be used to evaluate trade-offs can be seen on table \ref{tab:langCharacteristics}.

\begin{table}[H]
	\begin{tabular}{|l|l|}
		\hline
		Readability & How easy it is to understand and comprehend a computation \\ \hline
		Write-ability & How easy it is for the programmer to write a computation clearly, correctly, \\
		~ & concisely and quickly \\ \hline
		Reliability & Assures a program behaves the way it is suppose to \\ \hline
		Orthogonality & A relatively small set of primitive constructs can be \\
		~ & combined legally in a relatively small number of ways \\ \hline		
		Uniformity & If some features are similar they should look and behave similar \\ \hline
		Maintainability & Errors can be found and corrected and new features can be added easily \\ \hline
		Generality & Avoid special cases in the availability or use of constructs and by \\ 
		~ & combining closely related constructs into a single more general one \\ \hline
		Extensibility & Provide some general mechanism for the programmer to \\
		~ & add new constructs to a language \\ \hline
		Standardability & Allow programs to be transported from one computer \\
		~ & to another without significant change in language structure \\ \hline
		Implementability & Ensure a translator or interpreter can be written \\
		\hline
	\end{tabular}
	\caption{Brief explanation of language characteristics \citep{sebesta}}
	\label{tab:langCharacteristics}
\end{table}

These characteristics are used to evaluate the the trade-offs of programming language. An overview of these can be seen on table \ref{tab:langTradeOffs}.

\begin{table}[H]
	\begin{tabular}{l|c|c|c|}
\textbf{Characteristic} & \rotatebox{90}{Readability} &\rotatebox{90}{Writability} & \rotatebox{90}{Reliability} \\ \hline
		Simplicity & $\bullet{•}$ & $\bullet{•}$ & $\bullet{•}$ \\ \hline
		Orthogonality & $\bullet{•}$ & $\bullet{•}$ & $\bullet{•}$ \\ \hline
		Data types & $\bullet{•}$ & $\bullet{•}$ & $\bullet{•}$ \\ \hline
		Syntax design & $\bullet{•}$ & $\bullet{•}$ & $\bullet{•}$ \\ \hline
		Support for abstraction & ~ & $\bullet{•}$ & $\bullet{•}$ \\ \hline
		Expressivity & ~ & $\bullet{•}$ & $\bullet{•}$ \\ \hline
		Type checking & ~ & ~ & $\bullet{•}$ \\ \hline
		Exception handling & ~ & ~ & $\bullet{•}$ \\ \hline
		Restricted aliasing & ~ & ~ & $\bullet{•}$ \\ \hline
	\end{tabular}
	\caption{Overview of trade-offs \citep{sebesta}}
	\label{tab:langTradeOffs}
\end{table}

In table \ref{tab:langTradeOffs} $\bullet$ means that the characteristic affects the feature of the programming language where the $\bullet$ is placed. If there is no $\bullet$ in front of a feature, it means that this particular characteristic is not affected by the feature. 

Based on these trade-offs, it is clear that having a simple programming language affects both readability, writability and reliability. This is because having a very simple-to-understand language, might not make it very writable. On the other hand, having a simple-to-write programming language, might not make it very readable. An example of this is the if-statement in C, which can be written both with the 'if'-keyword, or more compact. This can be seen by comparing listing \ref{lst:ifstmtnormal} with listing \ref{lst:ifstmtcompact}, which both yield the same result. It is then clear, that the compact if-statement might be faster to write, but slower to read and understand, and opposite with the if-statement.

\begin{code}{ifstmtnormal}{Simple example of if-statement in C using the 'if'-keyword}
	\begin{lstlisting}
		if (x > y)
		{
    		res = 1;
		}
		else
		{
    		res = 0;
		}
	\end{lstlisting}
\end{code}

\begin{code}{ifstmtcompact}{Simple example of if-statement in C without using the 'if'-keyword}
	\begin{lstlisting}
		res = x > y ? 1 : 0;
	\end{lstlisting}
\end{code}

When defining the syntax of a programming language, it should balance these characteristics to achieve the right amount of trade-offs for that particular language. For the language of this project, it is important that the language is simple to read and understand, because the target group is the hobbyist-programmer, who might not have much experience in programming.