\subsection{Symbol Tables}
Symbol tables are used to store information such as type and attributes about names in the program to be compiled. Generally, there are two approaches to symbol tables: One symbol table for each scope, or one global symbol table \citep{sebesta}. 
\subsubsection{Multiple Symbol Tables}
In each scope, a symbol table exists, which is an ADT (Abstract Data Type), which stores the name of the identifiers and relates each identifier to its attributes. The general operations of a symbol table is: Empty the table, add entry, find entry, open and close scope \citep{sebesta}. 

It can be useful to think of the structure of static scoping and nested symbol tables as a kind of tree structure. Then when the compiler analyzes the tree, only one branch/path is available at a time. This exactly creates these features of e.g. local variables.

A stack might intuitively make sense because of the way scopes are defined by "begin" and "end". When a scope begins a symbol table is simply pushed onto the stack, and when the scope ends, the symbol table is popped from the stack. This also accounts for nested scopes. Searching for a non-local variable would require searching the entire stack \citep{sebesta}. 

\subsubsection{One Symbol Table}
To maintain one symbol table for a whole program, each name will be in the same table. The names must therefore be named appropriately by the compiler, so that each name also contains information about nesting level. Various approaches to maintain one symbol table exist, for example maintaining a binary search tree might seem like a good idea, because it is generally searchable in $O(lg(n))$. The fact that programmers generally do not name variables and functions at random, causes the search to take as long as linear search. Therefore hash-tables are generally used. This is because hash-tables perform excellently, with insertion and searching in $O(1)$, if a good hash function and a good collision-handling technique is used \citep{sebesta}.