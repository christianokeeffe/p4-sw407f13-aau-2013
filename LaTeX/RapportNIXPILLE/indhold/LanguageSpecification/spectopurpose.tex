\subsection{Specification of the language compared to the purpose}
\label{sec:specific}
To specify the language, so it will make the program as suitable as possible for writing code to be used in a drink machine, we looked at what the central aspects of a drink machine are:
\begin{inddes}
\item[A drink:] A drink is central to this machine. A drink should be the product made by the machine, defined by a number of ingredients. A drink should be like a recipe.
\item[An ingredient:] An ingredient is an element of a drink. They will be stored in a container within the machine. 
\item[A RFID-tag:] An RFID-tag will contain a drink ID and an amount of how many drinks there are left for the holder of the tag.
\item[A RFID-RW:] An RFID-reader and writer for Arduino, to write and read the content of an RFID-tag. See section \ref{SolutionInBars} for details on how an RFID works.
\item[A LCD:] For communicating with the user, an LCD is in most situations preferable.
\item[Buttons:] For getting input from users, buttons are a possibility like a button to allow the user to confirm the making of a drink. 
\item[Mechanism for pouring ingredients:] A mechanism for pouring the right amount of ingredients into the drink.
\end{inddes}
These are by our assessment the most central aspects of the drink machine system. We will now make a judgment of each of the listed aspects, and see if it is possible to make a structure in the language which will support the programmer or in any other way make it easier to implement this aspect in the system.
\subsubsection{A drink}
The concept of a drink is one of the most central aspects of a drink machine system. A drink should contain a recipe as a list of ingredients, and how much of the ingredients to pour. A drink also has a name, and should have some sort of ID which can be stored on an RFID-tag. We should make a structure which can implement a drink type and assign the recipe to the drink. The structure should be following the same design criteria as the rest of the language, and should be inspired by the other structures in the language. To fulfill these requirements, the declaration of a drink should have a block with "begin" and "end", and in the block have the "recipe". An example can be seen on listing \ref{lst:drink1}.

\begin{code}{drink1}{The first example of a declaration of a drink}
\begin{lstlisting}
drink [drinkname] is
begin
	[recipe]
end
\end{lstlisting}
\end{code}

We have now decided how to define an element of the type drink. We must now look at the body of the block in the declaration. To make it as readable as possible, it should be written as close to a normal recipe as possible. Because of that, we have decided to state it in the form "add [number] of [ingredient]", where [number] is a number representing the amount of the ingredient to add, though this will be up to the programmer to decide what type of measure it actually is. The declaration of a drink will now be defined as seen on listing \ref{lst:drink2}.

\begin{code}{drink2}{The final structure of how to declare a drink}
\begin{lstlisting}[mathescape]
drink [drinkname] is
begin
	add [number] of [ingredient];
	add [number] of [ingredient];
	$\vdots$
	add [number] of [ingredient];
end
\end{lstlisting}
\end{code}

In some situations, a drink could be similar to another drink, with only a few changes. An example could be a drink with a double shot of alcohol. In this case, it would be the exactly same drink, but with more of one ingredient. It could also be an ingredient that should be removed, for example alcohol to make it non-alcohol or if someone is allergic to some of the elements in a drink. Because of these situations and many more, it could be preferable to have a way to inherit the recipe from another drink, and then modify it. The declaration of a drink which inherits from another drink should be very much like the normal drink-declaration as seen on listing \ref{lst:drink2}, but also with some significant modifications, so it is easy to see that this drink inherits from another drink. The block structure should be the same, and the way to add an amount of an ingredient should be the same. We have to add a new command to the block statement: "remove". With the remove statement, it should be possible to completely remove an ingredient from a drink. It should be easy to read, and therefore, we have decided that the declaration statement before the block should be "drink [drinkname] as [drinkname of drink to inherit] but" to say that the drink is as the other drink, but with changes. The final structure will therefore be as seen on listing \ref{lst:drinkinherit}.

\begin{code}{drinkinherit}{The structure of how to declare a drink which inherits the recipe from another drink}
\begin{lstlisting}[mathescape]
drink [drinkname] as [drinkname of drink to inherit] but
begin
	add [number] of [ingredient];
	remove [ingredient];
	$\vdots$
	add [number] of [ingredient];
end
\end{lstlisting}
\end{code}

We have now defined how the structure of the type drink should be. To formally define the syntax, we use BNF which can be seen in grammar \ref{gra:drinkgram}.

\begin{grammatik}{drinkgram}{The grammar for the drink declaration}
<drinkdcl> $\ra$ drink <id> is begin <drinkstmts> end
\alt drink <id> as <id> but begin <changedrinkstmts> end

<drinkstmts> $\ra$ <drinkstmt> <drinkstmtsend>

<drinkstmt> $\ra$ add <numeric> of <id>

<drinkstmtsend> $\ra$ ; <drinkstmts>
\alt ;

<changedrinkstmts> $\ra$ <changedrinkstmt> <changedrinkstmtsend>

<changedrinkstmt> $\ra$ <drinkstmt>
\alt remove <id>

<changedrinkstmtsend> $\ra$ ; <changedrinkstmts>
\alt ;
\end{grammatik}

\subsubsection{Ingredients}
With the drink type defined above, we should now focus on how to handle ingredients. In the physical machine, an ingredient will in most cases be a liquid in a container. It could also be leaves or fruit slices, also in a container. When pouring an ingredient, no matter what ingredient it is, the drink machine should in some way (different for each type of ingredient) send a signal to the container to open and pour the wanted amount of the ingredient. When sending a signal in Arduino, you change the output voltage on a pin, so to make it easier for the programmer, we could implement a type container which hold the pinnumber for the signal to the container. The structure of this can be seen on listing \ref{lst:container}.

\begin{code}{container}{The structure of how to declare a container}
\begin{lstlisting}[mathescape]
container [containername] <-- [containerpin]
\end{lstlisting}
\end{code}

\subsubsection{An RFID-tag}
\label{sec:RFIDtag}
For a customer to use the drink machine, they should buy an RFID-tag with information of which drink and how many of the drink they have bought. A solution for these RFID-tags, is to take care of this for the programmer, so they should not deal with how to store the information and read it. It should therefore be provided by functions built into the language. This can be seen in section \ref{sec:RFIDRW}.

\subsubsection{An RFID-RW}
\label{sec:RFIDRW}
To read and write the RFID-tags, an RFID reader and writer is required. To simplify the communication with the RFID-RW, the language could, as mentioned in section \ref{sec:RFIDtag}, take care of it. When writing information to a tag, you should call a function with two parameters: The drink ID and the number of drinks. A call to the function RFIDWrite could be as the example seen on listing \ref{lst:RFIDWrite}.

\begin{code}{RFIDWrite}{An example of the call to RFIDWrite.}
\begin{lstlisting}[mathescape]
call RFIDWrite([drinkID], [Amount]);
\end{lstlisting}
\end{code}

\subsubsection{An LCD}
For communicating with the user of the drink machine, an LCD would be preferable like the one seen in figure \ref{fig:lcd}.

\figur{0.5}{lcd.jpg}{An example of a LCD to use in a drink machine. \citep{letlcd}}{fig:lcd}

Like the function provided to the RFID-RW (see section \ref{sec:RFIDRW}), it could simplify the programmers job if the language provided a function to print text on the LCD. The function should take two parameters: the string to print on the display and an integer which indicates which line to print the string on. This function will be called LCDPrint. It should also be possible to clear the LCD. The function to do this will be called LCDClear. An example of a call to the functions can be seen on listing \ref{lst:lcd}.

\begin{code}{lcd}{An example of how to call the LCD functions.}
\begin{lstlisting}[mathescape]
call LCDPrint("[text to print]", [linenumber]);
call LCDClear();
\end{lstlisting}
\end{code}

\subsubsection{Buttons}
To get input from either the staff or the customer, buttons could be used. It could be possible to make a button type in the language, and make some operations for the button easier. This could be making a listener, which will call a function when a button is pressed, or other buttons functions to be build into the language. This is considered as an good idea, but because it is not essential for the project it will not be implemented, but it could be made if the language was developed further.

\subsubsection{Mechanism for pouring ingredients}
The mechanism to pour the ingredient would be very different. A liquid will for example not be put into the drink the same way as leaves would be put into a it. For this reason, it has been decided not to support the pouring mechanism in any special way in the language, because there are too many possible ways to make a pouring mechanism.