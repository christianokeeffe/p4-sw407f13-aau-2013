In this section the semantics of SPLAD will be described. 

\subsection{Scoping}
The scope of a variable is the block of the program in which it is accessible. A variable is local to a block, if it is declared in that block. A variable is non-local to a block if it is not declared in that block, but is still visible in that block (ex. global variables) \citep{sebesta}.

In SPLAD static scoping is used. This means that scopes are computed at compile time, based on the program text input. The main reason for this, is that programs for the Arduino platform is mainly written in C, which also uses static scoping. This makes the compilation from SPLAD to C simpler for the compiler \citep{arduinobuild}. Static scoping means that a hierarchy of scopes are maintained during compilation. To determine the name of used variables, the compiler must first check if the variable is in the current scope. If it is, the value of the variable is found, and the compiler can proceed. Else it must recursively search the scope hierarchy for the variable. When done, if the variable is still not found, the compiler returns an error, because an undeclared variable is used \citep{sebesta}.
\subsubsection{Symbol Tables}
Generally there are two approaches to symbol tables: One symbol table for each scope, or one global symbol table \citep{sebesta}. 
\subsubsection*{Multiple Symbol Tables}
In each scope, a symbol table exists, which is an ADT (Abstract Data Type), that stores identifier names and relate each identifier to its attributes. The general operations of a symbol table is: Empty the table, add entry, find entry, open and close scope \citep{sebesta}. 

It can be useful to think of this structure of static scoping and nested symbol tables as a kind of tree structure. Then when the compiler analyzes the tree, only one branch/path is available at a time. This exactly creates these features of e.g. local variables.

A stack might intuitively make sense because of the way scopes are defined by begin and end. A begin scope would simply push a symbol table scope to the stack, and when the scope ends, the symbol table is popped from the stack. This also accounts for nested scopes. But searching for a non-local variable would require searching the entire stack \citep{sebesta}. 

\subsubsection*{One Symbol Table}
To maintain one symbol table for a whole program, each name will be in the same table. The names must therefore be named appropriately by the compiler, so that each name also contain information about nesting level. Various approaches to maintain one symbol table exists, for example maintaining a binary search tree might seem like a good idea, because it is generally searchable in $O(lg(n))$. But the fact that programmers generally does not name variables and functions at random, causes the search to take as long as linear search. Therefore hash-tables are generally used. This is because of hash-tables perform excellent, with insertion and searching in $O(1)$, if a good hash function and a good collision-handling technique is used \citep{sebesta}.