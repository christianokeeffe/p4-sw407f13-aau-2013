\section{Design Criteria}
When designing a language certain criteria should be kept in mind:
\begin{itemize}
\item Readability
- How easy it is to understand and comprehend a computation
\item Write-ability
- How easy it is for the programmer to write a computation clearly, correctly, concisely and quickly. 
\item Reliability
- Assures a program behaves the way it is suppose to.
\item Orthogonality
- A relatively small set of primitive constructs can be combined legally in a relatively small number of ways.
\item Uniformity
- If some features are similar they should look and behave similar.
\item Maintainability
- Errors can be found and corrected and new features can be added easily.
\item Generality
- Avoid special cases in the availability or use of constructs and by combining closely related constructs into a single more general one	
\item Extensibility
- Provide some general mechanism for the programmer to add new constructs to a language.
\item Standardability
- Allow programs to be transported from one computer to another without significant change in language structure.
\item Implementability
- Ensure a translator or interpreter can be written
\end{itemize}

\begin{tabular}{l|c|c|c|}
\textbf{Characteristic} & \rotatebox{90}{Readability} &\rotatebox{90}{Writability} & \rotatebox{90}{Reliability} \\ \hline
Simplicity & $\bullet{•}$ & $\bullet{•}$ & $\bullet{•}$ \\ \hline
Orthogonality & $\bullet{•}$ & $\bullet{•}$ & $\bullet{•}$ \\ \hline
Data types & $\bullet{•}$ & $\bullet{•}$ & $\bullet{•}$ \\ \hline
Syntax design & $\bullet{•}$ & $\bullet{•}$ & $\bullet{•}$ \\ \hline
Support for abstraction & ~ & $\bullet{•}$ & $\bullet{•}$ \\ \hline
Expressivity & ~ & $\bullet{•}$ & $\bullet{•}$ \\ \hline
Type checking & ~ & ~ & $\bullet{•}$ \\ \hline
Exception handling & ~ & ~ & $\bullet{•}$ \\ \hline
Restricted aliasing & ~ & ~ & $\bullet{•}$ \\ \hline
\end{tabular}

