\section{Parser}
A parser takes the tokens from the scanner and use them to create an abstract syntax tree. It also checks if the stream of tokens conforms to the syntax specification, usually written formal using context-free grammar (CFG).

The main purpose of the parser is to analyze the tokens and check if the source program is written in the correct syntax. If this is not the case the parser should show a message describing the error. The parser will at the end create an abstract syntax tree. 

Generally there are two different approaches to parsing: top-down and bottom-up.

\subsection{Top-down parsers}
Then top-down parser starts at the root and works its way to the leaves in a depth-first manner, doing a pre-order traversal of the parse tree. This is done by reading tokens from left to right using a leftmost derivation. Furthermore top-down parsers can be split into table-driven LL and recursive descent parse algorithms.

\subsubsection*{Table-driven LL Parsers}
Uses a parse table to determine what to do next. The entries in the parse table is determined by the particular LL(k) grammar. The parser then searches the table to see what to do.

\subsubsection*{Recursive-descent parser}
The recursive-descent parsers consists of mutually recursive parsing routines. Each of the non-terminals in the grammar has a parsing procedure that determines if the token stream contains a sequence of tokens derivable from that non-terminal.

\subsection{Bottom-Up Parsers}

A bottom-up parser has to do a postorder traversal of the parse tree, meaning that it starts from the leaves and works towards the root.
A bottom-up parser is more powerful and efficient than a top-down parser, but not as simple.

\subsubsection*{LR}
A LR parser reads from left to right and because it is a bottom-up parser it uses a reversed rightmost derivation which means it takes terminals and turn them into non-terminals. It is as the LL parser driven from a parse table. The biggest difference is how it is derived and how the parse table is handled.  

\subsubsection*{LALR}
A LALR(Lookahead Ahead LR) parser is one of the most commonly used algorithms today, because it is a powerful algorithm but do not need a very large parse table. It works like the LR parser.

\subsection{Abstract Syntax Trees}
The parser generates an abstract syntax tree (AST) \citep{CraftingACompiler}, which is an abstract data type describing the structure of the source program. This means that the AST contains information about which constructs the source program contains. More specifically, each node in the AST represent a construct in the source language, for example an 'if'-block.

When the AST has been generated, it is decorated with types by the type checker. The type checker traverses the AST, and checks the static semantics of each node, which means that it verifies that the node represent valid constructs. If each node is correct it is returned to the translator \citep{CraftingACompiler}. The translator then uses the AST to an intermediate representation (IR code), which is used in the later phases of the compiler.