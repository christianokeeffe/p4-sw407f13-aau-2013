\subsection{Overview of the Compiler}
\label{sec:OverviewCompiler}
Figure \ref{fig:OverviewCompiler} shows an abstract overview of each of the different phases in a compiler, what each phase requires as input, and what each step returns to the next phase.
\figur{0.6}{OverviewCompiler.PNG}{This is an abstract overview of how the compiler is structured. The figure is from \citep{OverviewCompiler2}.}{fig:OverviewCompiler} 

A compiler is a fundamental part of modern computing. Its purpose is to translate a programming language into a lower level programming language. A compiler can also compile source code into virtual instructions, which can be executed by virtual machines, as is the case for the Java-compiler. This makes the compiled program portable across different computers \citep{CraftingACompiler}.

A compiler consists mainly of three different phases. The different phases roughly correspond to the different parts in a language specification which can be seen on figure \ref{fig:OverviewCompiler}. The syntax analysis correspond to the syntax, the contextual analysis to the contextual constraints and the code generation phase roughly corresponds to the semantics.

These three main phases are even in simple compilers implemented through more phases. This can be seen on figure \ref{fig:OverviewCompiler2}. In the syntax analysis phase the compiler consists of a scanner and a parser. The scanner takes the source program and transforms it into a stream of tokens. The parser then uses the tokens to create an abstract syntax tree (AST). In the contextual analysis a symbol table is created from the abstract syntax tree. At the end, the semantic analysis decorate the AST, and translates this into the targeted language.

The syntactic analysis is described in detail in section \ref{sec:syntactic}, the contextual analysis is described in section \ref{sec:contextual}, and the code generation is described in section \ref{sec:codegeneration}.

\figur{0.8}{OverviewCompiler2.PNG}{This is an more detailed overview of how the compiler is structured. The figure is from \citep{CraftingACompiler}.}{fig:OverviewCompiler2}