\subsection{Scope Rules}
The scope of a variable is the range of statements where variables is visible. The variable is visible if it can be referenced in the statement.
The languages scope rules determine how a variable name is associated with a variable in a particular occurrence or then working with a functional language, it need to know how a name is associated with an expression then a variable is declared in a program unit or block. When a variable declared in a program unit or block it is local for that part. Then the non-local variables are visible within the program unit or block if they are not declared there. The last is global variables that is a special category of non-local variables.

\textbf{Static Scope}
Static scoping method was introduced in ALGOL 60, which is the method of binding names to non-local variables. There are two categories of static-scoped languages. First one is which sub-programs can be nested, which creates nested static scopes. And the other is static scopes is also created by sub-programs but nested scopes are created only by nested class definitions and blocks.

\textbf{Blocks}
Blocks are use to defined new static scopes in many languages. The idea is that it allows a section of code to have its own local variables.

An example on the use of block can be seen in the code \ref{BlockCode}. Before the block $\{ \}$ the variable $x$ is made and set to integer 5, in the block $x$ is set to 10 and a extra variable $y$ is made and is set to integer 15, $y$ is only visible inside block, thereby it can not be call out side the block, but after the block $x$ still have the value 10 that given inside the block.

\begin{code}{BlockCode}{a simple code with use of blocks}
\begin{lstlisting}
int x = 5
\{
    int y = 15
    x = 10
\} 
\end{lstlisting}
\end{code}

\textbf{Declaration Order}
The main think about declaration order is how the data declarations are made, do they need to be before functions, like C89, before they are used, like C#, or can they be anywhere in the code, like C99, C++, Java and JavaScript.

\textbf{Global Scope}
In some languages that allow a program structure to be a sequence of function definitions, like C, C++, PHP. Definitions outside functions in a file create global variables, which make it  visible to those functions.

\textit{Dynamic Scope}
With dynamic scope, the scope can only be determined at run time, because it is based on the calling sequence of sub-programs and not there their spatial relationship to one a other.

\textit{The difference between dynamic and static scope}
To help better understand the difference between dynamic and static Scope a larger code example is being shown here \ref{code:StaticDynamicScoping}.
\begin{code}{code:StaticDynamicScoping}{A simple code showing the difference between static and dynamic scoping.}
\begin{listing}
const int b = 5;
int foo()
{
	int a = b + 5;
	return a;
}

int bar()
{
	int b = 2;
	return foo();
}

int main()
{
	foo();
	bar();
	return 0;
}
\end{listing}
\end{code}
The code \ref{code:StaticDynamicScoping} will in both cases return 10 in the foo() function, but in bar() the result will differ. With static scoping the bar() function will return 10 while with dynamic scoping it will return 7.













