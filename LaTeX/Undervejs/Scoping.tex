\subsection{Scopecheck}
\label{sec:scopecheck}
In our project we have chosen to use static scoping. Because the program languages that is made in the project group is a imperative program languages, where it make more sense that it have static scoping. And static scoping are used by well use program languages like C, C\# and Java \citep{ProgrammingCommunityIndex}.
\fxfatal{hele intro skrives om. Vi SKAL have bedre argumentation end "det gør de andre, så det er smart"}

\fxfatal{der mangler et helt afsnit omkring formålet med scopecheckeren. Vi begynder bare at snakke om hvordan den virker, men ikke hvad vi vil med den. Altså det vi vil er, at forsøge at finde ud af om man tilgår en variabel uden for et scope}

When scope checking is started, a lists called "scopecontrol" is made which can hold others lists. Each list in scopecontrol is a "scope". The scope lists are used to store variable names. An example of this setup can be seen on figure \ref{fig:scopediagram}. Scope checking is break up in six different places in the tree visitor: visitProgram, visitBlock, visitFunction, visitDcl, visitSubparams and visitCallid.

\begin{figure}[H]

\centering
\begin{tikzpicture}

\node [style=mynodestyle] (v1) at (0,2.5) {Scopcontrole};
\node [style=mynodestyle] (v2) at (-2,1) {Scope 1};
\node [style=mynodestyle] (v3) at (2,1) {Scope 2};
\node [style=mynodestyle] (v4) at (-3.5,-0.5) {Variable a};
\node [style=mynodestyle] (v5) at (-0.5,-0.5) {Variable b};
\node [style=mynodestyle] (v6) at (2,-0.5) {Varibale c};
\draw [-latex] (v1) edge (v2);
\draw [-latex] (v1) edge (v3);
\draw [-latex] (v2) edge (v4);
\draw [-latex] (v2) edge (v5);
\draw [-latex] (v3) edge (v6);
\end{tikzpicture}
\caption{A visual diagram of the structure of scopecontrole}
\label{fig:scopediagram}
\end{figure}

For making sure that global variables is saved into scopecontrol a global scope is needed to be made in visitProgram. This scope will be removed when we are finished with visiting the program.
\fxfatal{indsæt kodeeksempel her}

There are a two ways to make blocks in our project languages, not counting the one that make the global scope. First one is in visitBlock, one get to visitBlock from if, while and fromStatement. In visitBlock a new list of strings is madem and then added to scopecontrol. Thereafter all statements in the block is visited. When this is done it will remove the list that it have made from scopecontrol and return.
\fxfatal{Hvad med blocken ved switch?}
\fxfatal{Indsæt kodeeksempel}

The second way to make new blocks is by making a function, and thereby visit the visitor visitFunction. A function name is not taken care of in the scope checker, because a function can only be declared in the outer scope. The visitBlock cannot be used to make scopes for functions, since there can be sent parameters to the function from where it is called. It must therefore make the list and add it to scopecontrol before the parameter are declared. After the parameters are declared, all statements in the function will be visited and it will then remove the list from scopecontrol before returning.

VisitDcl and visitSubparams is where variables names is inserted into the innermost scope, that have the last index in scopecontrol and is a list of strings.

The main part of scope checker is in visitCallid. Here it needs to look through all known scopes to see if a variable name exist. This is done by using two for-loops, which can be seen in code example \ref{lst:scopecallid}. The first one goes through scopecontrol and the second one is for each element in the list that are in scopecontrol. In this for-loops there is a if-statement that look for the variable name that is being called at the moment with the lists variable names. If it is in a list it will return from the visitCallid, but if it is not found in the lists it will put a error message into the list of errors and return from visitCallid.
\fxfatal{errorlisten er slet ikke nævn før. den bør nævnes noget tidligere}

\kode{code:scopecallid}{code snippet of where the name of a variable is look through all scopes}{scopecallid.txt} \fxfatal{ref ikke lavet i kode}

After the scopecheck is done, it will print all the errors in the list of errors.

\fxfatal{ret scopecheckeren, så den tjekker indefra og ud}