\section{Scoping}
%\begin{code}{code:StaticDynamicScoping}{A simple code showing the difference between static and dynamic scoping.}
%\begin{listing}
%const int b = 5;
%int foo()
%{
%	int a = b + 5;
%	return a;
%}
%
%int bar()
%{
%	int b = 2;
%	return foo();
%}
%
%int main()
%{
%	foo();
%	bar();
%	return 0;
%}
%\end{listing}
%\end{code}

%The code \ref{code:StaticDynamicScoping} will in both cases return 10 in the foo() function, but in bar() the result will differ. With static scoping the bar() function will return 10 while with dynamic scoping it will return 7. 


\subsection{Scoping in our Project}
In our project we have chosen to use static scoping. Because the program languages that is made in the project group is a imperative program languages, where it make more sense that it have static scoping. And static scoping is used by well use program languages like C, C# and Java. \ref{Programming_Community_Index}

Data that is used for scoping, is save in a list (scopecontrol) that is made of lists and these lists hold variable names. The scoping is taken care of in 6 different places in the parse tree visitor, in visit-program, visit-block, visit-from, visit-DCL, visit-Subparams and visit-callid. In visit-program it is needed to make a scope, This is because there are global variables and they need a scope to be places in. In visit-block a new list is added to the list of lists, after all the element is visit the scope 


