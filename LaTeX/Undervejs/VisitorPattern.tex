\subsection{Visitor pattern}
\label{sec:VisitorPattern}
Visitor pattern is for traversing a AST or a parse tree during semantic analysis and code generation, to help manage the large numbers phase and node interactions. A phase made by writing \textit{visit} methods in the phase's class, there is a \textit{visit} for each kind node for action must be performed.

here is an example on how Generic visit look like:

$f.visit(\textbf{AbstractNode \; n})
\;\; n.Accept(this)
end$

% næsten det samme som står i bogen
Single dispatch is used by most object-oriented languages to determine which $visit$ must be use. The method is based on the type of the object $f$. But there is a few problem with single dispatch, such as that it find a match based on the declared type of its parameters when it is called. For example if a phase contained a method $visit(IfNode$ $n)$, it will not invoked an a actual $IfNode$, this is because the is based on the declared type $(AbstractNode)$ of the supplied parameter.

Another option is to use double dispatch, this is use to allows to invoke $visit$ the compiler's phase $f$ and the supplied node $n$ actual type.