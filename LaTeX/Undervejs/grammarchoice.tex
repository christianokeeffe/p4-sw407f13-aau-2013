\section{Choice of grammar}
It was decided that the grammar for this project should have a high level of readability. This was decided because the programmer could be a hobby programmer, who want to make a drink machine, but does not have a high level of education in programming. A high readability will ensure that this person easily can read and understand their program - also if they have to edit it later on. The method to assign a value to a variable is by typing "$variable$ <-- $value to assign$". By using the arrow, it is clearly indicated that the value is assign to the variable, and therefore ensuring readability - especially for the hobby programmer.
The functions must always return something, but it can return the value "nothing". This will ensure the understanding and readability, when the programmer can see that it returns, but no value i parsed. To indicate that $return$ is the last thing which will be executed in a function, the $return$ must always be at the end the function. To indicate that a program is called there must be written "call $functionname$".
There are used words instead of symbols (compared to most other programming languages) were suitable to improve the understanding of the program. To indicate a block (eg. a if statement), there are used "begin" and "end". To combine logical operators the words "AND" and "OR" is used. To end a line ";" is used, also to improve readability.

It would be appropriate to design a grammar that is a subset of $LL(1)$ grammars. This is based on the idea that it easy to implement a parser for $LL(1)$ grammars by hand compared to $LR$ grammars. This approach means it would possible to both implement a parser by hand or use some of the already existing tools. This way both approaches are possible which are a suited solution for the project because it allows the project group to later go back and make the parser by hand instead of using a tool if that was the intention.

If the purpose was to create an efficient compiler it would be more appropriate to design the grammar as a subset of the $LALR$ grammar class. A parser for $LALR$ is balanced between power and efficiency which makes it more attractive than $LL$ and other $LR$ grammars, see section \ref{sec:grammar} for more on the grammars. $LR$ parsers can be made by hand but it is a lot more difficult that the $LL$ parsers.