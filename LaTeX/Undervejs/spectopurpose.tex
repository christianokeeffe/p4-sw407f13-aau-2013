\section{Specification of the language to the purpose}
To specify the language, so it will make the program as suitable as possible for writing drink machines, we looked at what the central aspects of a drink machine is:
\begin{description}
\item[A drink:] A drink is central to this machine. A drink should be the product made by the machine, defined by a number of ingredients. A drink should be like a recipe.
\item[An ingredient:] An ingredient is the elements of a drink. It will in the machine be contained in a container. 
\item[A RFID-tag:] A RFID-tag with a drink ID and an amount of how many drinks there are left.
\item[A RFID-RW:] A RFID-reader and writer for Arduino, to write and read the content of a RFID-tag.
\item[A LCD:] For communicating with the user, a LCD is in most situations preferable.
\item[Buttons:] For getting input from users, buttons are a possibility. 
\item[Mechanism for pouring ingredients:] A mechanism for pouring the right amount of ingredients into the drink.
\end{description}
These are by our assessment the most central aspects of the drink machine system. We will now make a judgment of each of the listed aspects, and see if it is possible to make a structure in the language which will support the programmer on in any other way make it easier implement this aspect in the system.
\subsection{A drink:}
The concept of a drink is one of the most central aspects of a drink machine system. A drink should contain a recipe as a list of ingredients, and how much of the ingredients to pour. A drink do also have a name, and should have a form for ID which can be stored on a RFID-tag. We should make a structure which can implement a drink type and assign the recipe to the drink. The structure should be following the same design criterias as the rest of the language, and should be inspirited by the other structures in the language. To fulfill these requirements, 