Opgave 1
1.1
Physical memory er den reelle hukommelse maskinen har, f.eks. harddisken.
Virtual memory er en abstraktion over den fysiske hukommelse.

1.2
Thrasing er når processoren bruger mest tid på page fault.

1.3
Principle of locality er væsenlig fordi det bruger page replacement for at mindske page faults.

1.4
Prepaging er når den prøver at forudse hvad der skal bruges og så henter det ind i forvejen.

1.5
Page demanding loader en page hvis der sker en page fault.

1.6
Page buffering er når operativ systemet permanent har et antal ledige frames, hvis en frame ikke skal bruges lægges den i en buffer. Hvis der sker en page fault kigger den i bufferen før den kigger på disken og kan loade buffer frames hurtigere.

Opgave 2

Opgave 3

