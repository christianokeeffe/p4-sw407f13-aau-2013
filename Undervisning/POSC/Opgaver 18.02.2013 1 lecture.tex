1. Discuss what resources an operating system can/should manage.
2. Discuss pros and cons of having the GUI be part of the operating system.
3. Discuss if an operating system should manage the filesystem, i.e., how files are named and organised on a hardddisk.
4. Explain how race conditions can result in program errors.
5. Discuss if it is possible to implement protection against race conditions in software or if hardware support is needed.
6. Discuss if race conditions induced by parallelism are similar to, or worse than, race conditions induced by concurrency.
7.Find out if your favourite operating system has a shell and, if so, how to use it.

1. Et operativ system skal være en bro mellem hardwaren og softwaren, håndtere hukommelse (Harddisk RAM etc.), processer, navne og koordinere hændelser og aktiviteter. 

2. cons: Det bliver mere låst.
   pros: Der er en standart.
   
3. Ja, operativ systemet skal håndtere filsystemet for at undgå at andre programmer fx kan gå ind og overskrive/ødelægge andres filer. Operativ systemet skulle også forhindre diskfragmentering idet den sørge for filerne ligger sammen.

4. Hvis fx et program skal bruge et resultat fra et sub program, hvor sub programmet så går ned, går main programmet også ned.

5. Ja det er muligt men det kræver både software og hardware support

6. Det er værst ved parallelisme, fordi de 2 paralle processor kan tilgå den sammen variable, hvor de begge 2 så kan finde på at lave om på variablen hvorefter dette kan forsage et forkert resultat. I samtidige sker det i sekvens så først den ene delproces og derefter den anden og derved tilgås variablen ikke på sammentid og derfor kan der kun være fejl i den ene af processerne.

7. Windows har en shell.