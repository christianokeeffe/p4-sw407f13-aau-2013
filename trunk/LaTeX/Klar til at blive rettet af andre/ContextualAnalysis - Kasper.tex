\section{Contextual Analysis}
%Til retter - læs hvad der står under contextual analysis i deres rapport, og tjek om det ikke er det rigtige
Because we are using ANTLR, we get a complete parse tree with a basic visitor to work on. With the basic visitor to expand, three modified visitor have been made. First visitor is for checking if the scopes rules are in order, the errors that are found is put into a error list which will be shown to the programmer to show that is going wrong with the code. The second the visitor is for type checking, it will visit the\fxfatal{skal det være "a"} parse tree to see if all the types that are used us use correct together, again will errors be put into a list and shown to the programmer. The last visitor will generate code from our language into language that can be used by an Arduino.

For making a parse tree with ANTLR generate functions over a given code,This can be seen on the code snippet \fxfatal{kodestykke fra main filen}.

When making a parse tree with the ANTLR generate functions with first one is the lexer, can be seen on line \fxfatal{linje}, where it can be seen that the lexer need the program code. Then finding the tokens the command that be seen on line \fxfatal{linje} is used, here it can be seen that the token generator need the lexer output for making the tokens. And for making the parser the tokens are needed, the command that be seen on line \fxfatal{linje}.

With the parse tree it is possible to use the scope checker, type checker and the code generator. The scope checker is call in line \fxfatal{linje}. After scope checking comes types checking with is called in line \fxfatal{linje}. When scope checking and types checking is complete and there is no errors it will start generating.