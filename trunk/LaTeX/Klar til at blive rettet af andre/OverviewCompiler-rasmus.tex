\section{Overview of the Compiler}

\figur{0.8}{OverviewCompiler2.PNG}{This is an overview of how the compiler is structured. The picture is from \citep{CraftingACompiler}.}{fig:OverviewCompiler2}

On figur \ref{CraftingACompiler2} can be seen an overview of each of the different steps in the compiler and what it takes in and parses on the the next step.

A compiler is a fundamental part of modern computing. There job is to translate programming language into machine language. A compiler allows programs to virtually all computers to ignore the machine-dependent details of machine language and therefor be portable across a lot of different computers. \citep{CraftingACompiler}

\figur{0.8}{OverviewCompiler.PNG}{This is another overview of how the compiler is structured.}{fig:OverviewCompiler}

A compiler consists of 3 different phases. The different phases roughly correspond to the different parts in a language specification, which can be seen on figur \ref{OverviewCompiler} . The syntax analysis correspond to the syntax, the contextual analysis to the contextual constrains and the code generation correspond to the semantics.

If you take a simple compiler it will go thorough a bit more than three phases, as seen on figur \ref{OverviewCompiler2}. Under the syntax analysis the compiler consists of a scanner and a parser. The scanner takes the source program and makes a stream of tokens, the parser then uses them to make an abstract syntax tree. In the contextual analysis a symbol table is made from the abstract syntax tree. Last comes the semantic analysis where the AST is decorated and then translated into the target language.

The different phases will be described more thorough later in the rapport. \fxfatal{Få lavet konkret henvendelse når opbygningen på rapporten er fastsat.}