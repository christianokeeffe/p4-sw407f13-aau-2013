\section{Parse Tree} \fxfatal{skal den være i section, subsection eller slet ikke i en?}
Parse tree are nearly the same as a abstract syntax tree, but here all internal nodes are labeled with a non-terminal symbol and all leaves are labeled with a terminal symbol. A sub-tree describes one instance of an abstraction of a sentence.

To help understand the different between abstract syntax tree \ref{fig:abstract-syntax-tree} and parse tree \ref{ParseTreeEKS}, two trees are made, one of each, over the same code for a variable declaration \ref{VariableDeclarationEKS} in the projects language.

\begin{code}{VariableDeclarationEKS}{A simple variable declaration in the project language.}
	\begin{lstlisting}
		int x <-- 3+2;
	\end{lstlisting}
\end{code}

\begin{figure}[H]
\Tree[.program [.<-- [.x
]
                    [.+ [.3
]
                        [.2
                    ]]]]
\caption{A abstract syntax tree}
\label{fig:abstract-syntax-tree}
\end{figure}


\figur{0.7}{parsetree.png}{A parse tree that is made by using ANTLR.}{fig:ParseTreeEKS}