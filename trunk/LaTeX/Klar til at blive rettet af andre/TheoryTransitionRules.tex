\section{Transition Rules}
In this section some of the transition rules in SPLAD will be explained. The complete list of all the rules can be seen in appendix \ref{app:TransitionRules}.
In the following text we use the following names to represent different syntactic categories.

\begin{itemize}
\item $n \in \textbf{Num}$ - Numerals
\item $v$ - Values
\item $x \in \textbf{Var}$ - Variables 
\item $r \in \textbf{Arrays}$ - Array names
\item $a \in A{exp}$ - Arithmetic expression
\item $b \in B{exp}$ - Boolean expression
\item $e$ \fxfatel{hvad er det her ? mit bud er $e \in A_{exp} \cup B_{exp}$}
\item $C \in \textbf{Com}$ - Commands
\end{itemize}

\subsection{Environment-Store Model}
In our project we use the \textit{environment-store model} to represent how a variable is bound to a storage cell (called a \textit{location}), in the computer, and that the value of the variable is the content of the bound location. All the possible locations are denoted by \textbf{Loc} and a single location as $l \in \textbf{Loc}$. We assume all locations are integer, and therefore $\textbf{Loc} = \mathbb{Z}$. Since all locations are integers er can define a function to find the next location: $\textbf{Loc} \rightarrow \textbf{Loc}$, where $l = l + 1$

We define the set of stores to be the mappings from locations to values $\textbf{Sto } = \textbf{ Loc } \rightharpoonup \mathbb{Z}$, where $sto$ is an single element in $\textbf{Sto}$.

The following names represent the different environments.
\begin{itemize}
\item $env_V \in Env_V$ - Variable declarations
\item $env_A \in Env_A$ - Array declarations
\item $env_C \in Env_C$ - Constant declarations
\item $env_E$ \fxfatal{What is this?}
\end{itemize}

beskriv l side 79
\item $sto$

\subsection{Arithmetic Expressions}
The transition rule for multiplication in SPLAD can be seen on table \ref{tab:MultExp}. The rule states, that if $a_1$ evaluates to $v_1$ and $a_2$ evaluates to $v_2$, using any of the rules from $A_{exp}$, then $a_1 * a_2$ evaluates to $v$ where $v = v_1 * v_2$.

\begin{table}
\begin{tabular}{l l}
[MUL] & \[\frac{env_E, \: sto \vdash a_1 \rightarrow_a v_1 \;\; env_E, \: sto \vdash a_2 \rightarrow_a v_2}{env_E, \: sto \vdash a_1 * a_2 \rightarrow_a v}\] \\
~ & ~ \\
~ & \indent\indent where $v=v_1 * v_2$ \\ 
~ & ~ \\
\end{tabular}
\caption{The transition rule for the arithmetic multiplication expression.}
\label{tab:MultExp}
\end{table}

\subsection{Boolean Expressions}
The transition rule for logical-or in SPLAD can be seen on table \ref{tab:OrExp}. The rules have to parts: [OR-TRUE] and [OR-FALSE]. The [OR-TRUE] rule states that either $b_1$ or $b_2$ evaluates to \textit{true}, using any of the rules from $B_{exp}$ then the expression $b_1 OR b_2$ evaluates to \textit{true}. [OR-FALSE] states that if both $b_1$ and $b_2$ evaluates to \textit{false} then the expression $b_1 OR b_2$ evaluates to \textit{false}.

\begin{longtable}{l l}
\longtablesetting{2}
[OR-TRUE] & \[\frac{env_E, \: sto \vdash b_1 \vee b_2 \rightarrow_b \text{true}}{env_E, \: sto \vdash b_1  \; \text{OR} \; b_2 \rightarrow_b \text{true}}\] \\
~ & ~ \\

[OR-FALSE] & \[\frac{env_E, \: sto \vdash b_1 \wedge b_2 \rightarrow_b \text{false}}{env_E, \: sto \vdash b_1  \; \text{OR} \; b_2 \rightarrow_b \text{false}}\] \\
~ & ~ \\
\caption{Transition rule for the boolean expression logical-or.}
\label{tab:OrExp}
\end{longtable}

\subsection{Declarations}
\fxfatal{Kan ikke laves siden at transition rules til dette punkt ikke er lavet endnu}

\subsection{Assignments}
The transition rule for variable assignment in SPLAD can be seen on table \ref{tab:VarAssign}. When a variable is assigned the contents of $l$ is updated to $v$, where $l$ is the location of $x$ found in the $env_V$ and $v$ is the result of evaluation $e$.

\begin{longtable}{l l}
\longtablesetting{2}
[VAR-ASS] & \[env_C, \: \vdash \langle x=e, \; sto \rangle \rightarrow sto[l \mapsto v]\] \\
~ & ~ \\
~ & \indent\indent where $env_C, \; sto \vdash e \rightarrow_e v$ \\
~ & \indent\indent and $env_V \; x=l$ \\
~ & ~ \\
\caption{Transition rule for variable assignment.}
\label{tab:VarAssign}
\end{longtable}

\subsection{Commands}
The transition rule for the while statement in SPLAD can be seen on table \ref{tab:WhileStatement}. The rule have to parts: [WHL-TRUE] and [WHL-FALSE]. If the condition $b$ evaluates to \textit{true} then the [WHL-TRUE] states that $C$ will be executed which will update the \textit{store} (sto) and again call the expression and evaluate the new $b$. If the condition $b$ evaluates to \textit{false} then $C$ is \underline{not} executed and the \textit{store} is not updated. The program exits the while statement.

\begin{longtable}{l l}
\longtablesetting{2}
[WHL-TRUE] & \[\frac{env_C \: \vdash \langle C, \: sto \rangle \rightarrow sto'' \; env_C \: \vdash \langle \text{\textbf{while}}(b)\;\text{begin}\;C\; \text{end}, \: sto'' \rangle \rightarrow sto'}{env_C \: \vdash \: \langle \text{\textbf{while}}(b) \: \text{begin}\;C\;\text{end}, \: sto \rangle \rightarrow sto' }\] \\
~ & ~ \\
~ & \indent\indent if $env_C, \; sto \vdash b \rightarrow_b \text{true}$ \\
~ & ~ \\

[WHL-FALSE] & \[env_C \: \vdash \langle \text{\textbf{while}}(b) \: \text{begin} \: C \: \text{end}, \: sto \rangle \rightarrow sto\] \\
~ & ~ \\
~ & \indent\indent if $env_C, \; sto \vdash b \rightarrow_b \text{false}$ \\
~ & ~ \\
\caption{Transition rules for the while statement.}
\label{tab:WhileStatement}
\end{longtable}