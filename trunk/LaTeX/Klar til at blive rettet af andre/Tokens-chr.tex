\subsection{Tokens}
For a compiler to able to distinguish between variables names and types the compiler will need some rules to describe the difference between them. This is done by reserving the words, called keywords, which are used to describe types, the beginnings and endings of blocks, and declaration of statements. A variable may not be named the same as any of the keywords since the compiler can not distinguish if it is a variable name or a reserved keyword.

\subsubsection{Reserved Keywords}
The reserved keywords for SPLAD can be seen on table \ref{tab:keywords}.

\begin{table}[H]
	\begin{tabular}{|l|l|l|}
		\hline
		AND			&	end			&	OR		\\ \hline
		begin 		&	false		&	return	\\ \hline
		bool 		&	from		&	step	\\ \hline
		break 		&	function	&	string	\\ \hline
		case 		&	HIGH		&	switch	\\ \hline
		char 		&	if			&	to		\\ \hline
		container 	&	int			&	true	\\ \hline
		default 	&	LOW			&	using	\\ \hline
		double 		&	nothing		&	while	\\ \hline
		else 		&	~			&	~		\\ \hline
	\end{tabular}
	\caption{The reserved keywords in SPLAD.}
	\label{tab:keywords}
\end{table}

This list is used to keep track of which words are going to be reserved and in that way provide an overview for the programmer. 

\subsubsection{Token Specification}
A parser needs a stream of tokens to parse a program correctly. These tokens are generated by a lexer which reads a stream of input symbols and from a given set of rules, makes the corresponding tokens. A token specification is used to describe the rules the lexer need in the construction of tokens. Token specification are expressed in way related to regular expressions \citep{sebesta}. Regular expressions are strong in describing patterns which is the core of token production \citep{sipser}. The tokens used for this project can be seen on table \ref{tab:tokens}.

\begin{table}[H]
\begin{tabular}{|l|l|}
\hline
PRIMITIVETYPE	& 'int' | 'double' | 'bool' | 'char' | 'container' | 'string' \\
STRINGTOKEN		& " $\dots$ " \\
DIGIT 			& $[0 - 9]^+$ \\
NOTZERODIGIT 	& $[1-9][0-9]^*$ \\
LETTER 			& $[A-Za-z]^+$\\
COMMENT 		& /* $\dots$ */ \\
WHITESPACE 		& \textbackslash r | \textbackslash n | \textbackslash t \\
OTHER 			& $\varepsilon$ \\ \hline
\end{tabular}
\caption{The tokens in SPLAD.}
\label{tab:tokens}
\end{table}

Further work would be making a lexer to generate a token for the parser. Another options was to find a suited tool for generating a lexer for the given rules. This is a valid option because making a lexer can be automated and therefore already exists a lot of good lexer generators that can be used, see section \ref{sec:KnownLexersAndParsers}.