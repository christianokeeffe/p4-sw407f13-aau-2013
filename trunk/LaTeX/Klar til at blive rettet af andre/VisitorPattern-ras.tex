\subsection{Visitor pattern}
\label{sec:VisitorPattern}
Visitor pattern is for traversing an AST or a parse tree during the semantic analysis and code generation, to help manage the large number of phase and node interactions. A phase is started by visiting the first node in the abstract syntax tree, to reach every node there must be a \textit{visit} in the preceding node.
Single dispatch is used by most object-oriented languages to determine which $visit$ must be used. The method is based on the type of the object $f$. But there are a few problems with single dispatch, such as that it finds a match based on the declared type of its parameters when it is called. This is where the visitor pattern comes in with a form of double dispatch, the idea is to make use of the abstract class like the one in listing \ref{lst:GenericVisitor} that all nodes implement. If the visit method is called without the accept method, it will try to visit the abstract node. For example if a phase contained a method visit(\textit{IfNode n}) like in listing \ref{lst:ConcreteVisitor}, it will not invoked the actual \textit{IfNode}, this is because the supplied parameter is based on the declared type (\textit{AbstractNode}). Therefore the specific node accepts a visitor as can be seen in listing \ref{lst:ConcreteNode} and thus determines the type of the node which allows it to be visited because it now knows the actual \textit{IfNode} and the code it contains can be executed.

Here is an example of a basic visitor model:

\begin{code}{GenericVisitor}{A generic Visitor}
\begin{lstlisting}
class Visitor
	procedure visit(AbstractNode n)
		n.ACCEPT(this)
	end
end
\end{lstlisting}
\end{code}

\begin{code}{ConcreteVisitor}{A concrete Visitor}
\begin{lstlisting}
class TypeChecking extends Visitor
	procedure VISIT(IfNode i)
	end
end
\end{lstlisting}
\end{code}

\begin{code}{ConcreteNode}{A concrete Node}
\begin{lstlisting}
class IfNode extends AbstractNode
	procedure ACCEPT(Visitor v)
		v.VISIT(this)
	end
	...
end
\end{lstlisting}
\end{code}