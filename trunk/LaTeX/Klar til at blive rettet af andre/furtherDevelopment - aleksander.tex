\subsection{Further Development}
This section outlines which areas of the language, compiler and the Arduino component would be natural to develop further. It would be interesting to actually build the drink machine, with appropriate containers with ingredients, and hoses and valves, so that drinks actually could be mixed. It has not been possible to build a complete drink machine because of limited funds. It could be interesting to pitch the idea of a drink machine to an actual bar, to see if it would be useful in a real context. On the compiler side, it would be natural to further develop this into actually producing Arduino machine-code, and thereby skipping the step of compiling in the Arduino compiler. The reason why it was decided to compile into Arduino C/C++, is that it was decided that it would be too time consuming to fully understand the Arduino machine-code, and then compile into machine-code. When debugging code, it is important to have correct error messages. While this already is the case for the SPLAD compiler, it does not provide somewhat important information such as the line number the error happened. The reason why this was not in focus in the initial development of the compiler, is that it was simply deemed too time consuming, and not as important as getting the compiler working in the first place. The language could be extended with a construct to handle arrays of drinks, which would enable a simpler way to work with multiple drinks. Again the reason why this was not implemented in the first place, is that is would be too time consuming, compared to the deadline of this project. 