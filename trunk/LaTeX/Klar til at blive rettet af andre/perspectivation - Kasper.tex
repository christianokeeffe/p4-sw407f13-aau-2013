%ret kun angivne 
\subsection{Putting the Project into Perspective}
The idea of this project is, that any hobby programmer can program their own drink machine based on the Arduino platform. The purpose of the SPLAD language is to provide an easy alternative to the Arudino C/C++ like language, when programming a drinks machine. In the first place this was intended to be used by the owners of small bars, which also has a hobby of programming. Using the SPLAD language, they will be able to build their own drinks machines and program them to their liking.

The reason why bar owners would want this solution is that it could lessen the work load of the bartender, by replacing some of his or her tasks. For certain drinks, the bartender could simply sell a pre-programmed RFID-tag, which could be used on the drinks machine, and the bartender could then serve more customers. This is of course an advantage to both the bartenders and the bar owners. The bartenders can focus on other tasks than mixing the drinks, and for the bar owners, it would be like having an extra bartender without having to pay the extra wage. 
%herfra
SPLAD could make it this possible to a group of people, who wouldn't have achieved to make such a program otherwise.
%hertil

\subsubsection{Further Development}
This section outlines which areas of the language, compiler and the Arduino component would be natural to develop further.

It would be interesting to actually build the drink machine, with appropriate containers with ingredients, hoses and valves, so it actually could mix drinks. It has not been possible to build a complete drink machine because of limited funds. The construction of the drink machine would also require time to build which either must be taken from the actual project or some of the non-project time must be dedicate to the built. It could be interesting to pitch the idea of a drink machine to an actual bar, to see if it would be useful in a real context.

On the compiler side, it would be natural to develop the compiler so it do the producing of the Arduino code itself, and thereby skip the step for the user of compiling in the Arduino compiler. The reason why it was decided to compile into Arduino C/C++ is, that it was decided that it would be too time consuming to fully understand the Arduino machine-code, and then compile into machine-code.

When debugging code, it is important to have correct error messages. While this already is the case for the SPLAD compiler, it does not provide somewhat important information such as the line number the error happened. The reason why this was not in focus in the initial development of the compiler, is that it was simply deemed too time consuming, and not as important as getting the compiler working in the first place. The language could be extended with a construction to handle arrays of drinks, which would enable a simpler way to work with multiple drinks. Again the reason why this was not implemented in the first place, is that is would be too time consuming, compared to the deadline of this project.
%herfra
At last, SPLAD should also handle different operations on the drink, as stir, shake, cool down, heat and so on. This was not supported in the project, because there are too many ways of implementing such a feature. In a future version of SPLAD, this should be supported.
%hertil