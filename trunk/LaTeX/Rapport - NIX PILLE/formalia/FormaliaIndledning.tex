Dette projekt fokuserer på at analysere et senarie i virkeligheden, som ifølge den objektorienterede arbejdsmetode kaldes problemområdet og få oprettet et objekt orienteret billede af problemområdet. Konflikterne i dette område skal derefter analyseres. Dette fører til en definition af anvendelsesområdet, hvor mulige aktører og brugsmønstre bliver fundet. Arbejdet leder så til, at det er muligt at designe et systemet, der skal fungere som en løsning på konflikterne i problemområdet.
Gruppen har valgt at anvende den iterative model frem for vandfalds-modellen. Den objektorienterede arbejdsmetode er beskrevet i \citep{ooaogd}.
Den iterative metode betyder, at alle disse aktiviteter foregår sideløbende med hinanden og i flere omgange.

I dette projekt har gruppen besluttet at arbejde med kundens handlen og den indbyrdes håndteringen af indkøbslister mellem kunderne.