\chapter{Prolog}
Dette projekt er skrevet af Aleksander Sørensen Nilsson, Christian Jødal O'Keeffe, Kasper Plejdrup, Mette Thomsen Pedersen, Niels Brøndum Pedersen og Rasmus Fischer Gadensgaard. Projektet blev påbegyndt i september 2012 og var færdigt den 20. december 2012 på Aalborg universitet, hvor alle i gruppen studerer til Software Ingeniør på 3. semester.
Alle deltagerne valgte oprindeligt Aalborg universitet, fordi de kan lide PBL-metoden, der bliver brugt på Aalborg universitet. PBL står for \gaas{problem baseret læring} og som i Danmark kun bliver brugt på Aalborg universitet og Roskilde Universitet \citep{pblaau}.

Målet med dette projekt er at udvikle et produkt ved hjælp af gruppens viden og evner inden for objektorienteret programmering, algoritmer og datastrukturer samt design og evaluering af brugergrænseflader. Ud fra den samlede proces er målet at tilegne og forbedre færdigheder inden for de forskellige områder, herunder hele den objektorienterede arbejdsmetode.

Den objektorienterede metode er benyttet til at anskue problem- og anvendelsesområdet, og på den måde definere problemet og udvikle en løsning \citep{ooaogd}.
Kurserne i algoritmer og datastrukturer bruges til at udvikle en effektiv og hurtig algoritme, der kan løse problemet.
Kurset \gaas{evaluering af brugergrænseflader} bruges til at lave en brugergrænseflade og senere udføre tests for at tjekke designet. Gruppen vil gerne takke eTilbudsavisen (www.etilbudsavisen.dk) for at låne deres API ud til gruppen i dette projekt.

Produktet for dette projekt er en hjemmeside, som skal agere indkøbsassistent og kunne foreslå tilbud til brugeren ud fra dennes indkøbsvaner.