\chapter{Tests}
I kurset \gaas{Design og evaluering af brugergrænseflader} er der blevet undervist i forskellige typer af tests \citep{DIS}.

\section{Generelt om test}
Tests bruges til at undersøge eksempelvis systemer og koncepter i en virkelighedsnær kontekst. Resultaterne fra en sådan test kan efterfølgende bruges til at evaluere systemet eller konceptet, og finde svagheder og styrker i disse. På baggrund af dette, kan man foretage ændringer for at mindske svagheder samt udbedre fejl. 

\section{Forskellige typer tests}
Der findes forskellige metoder til at evaluere et system. En sammenligning af nogle af disse kan ses på tabel \ref{tab:Testsammenligning}. 

\begin{table}[H]
\centering

    \begin{tabularx}{\textwidth}{|R|R|R|}
	\hline
		Type    				& Fordele	& Ulemper
		\\ \hline
		Usability laboratorium
		&
		\begin{noindlist}
		\item Grundig gennemgang af brugers reaktioner på system 
		\item Effektivt til at finde uklarheder i systemet
		\end{noindlist}
		&
		\begin{noindlist} 
		\item Tidsmæssigt dyrt
		\item Kontekst kan mangle da det er i scene sat
		\end{noindlist}
		\\ \hline
		Prototype  				
		&
		\begin{noindlist}
		\item Nem måde at få hurtigt feedback
		\end{noindlist}
		&
		\begin{noindlist}
		\item Begrænset feedback da brugeren ikke selv sidder med det i hånden
		\end{noindlist}
		\\ \hline
		Spørgeskema     		
		&
		\begin{noindlist}
		\item Tidsmæssigt billig, Ikke krævende for testpersoner
		\item Hurtig og struktureret feedback
		\end{noindlist}		 
		&
		\begin{noindlist}
		\item Svar begrænset til spørgeskemaets spørgsmål
		\item Svar kvalitet afhænger udelukkende af testpersonens svar evne
		\end{noindlist}
		\\ \hline
		Interview     			
		&
		\begin{noindlist}
		\item Kvalitativ feedback
		\item Subjektiv feedback 
		\end{noindlist}
		&
		\begin{noindlist}
		\item Kræver at både testpersoner og interviewer er til stede samtidig
		\item Kontekst kan mangle/være mangelfuld 
		\end{noindlist}
		\\ \hline
		Felt observation		
		&
		\begin{noindlist}
		\item Subjektiv feedback i rigtig kontekst
		\end{noindlist}
		&
		\begin{noindlist}
		\item Situations bestemte fejl/mangler kan være svære at finde
		\end{noindlist}
		\\ \hline
\end{tabularx}
	
\caption{Tabel med sammenligning af forskellige testmetoder}
\label{tab:Testsammenligning}
\end{table}

\section{Tests i dette projekt}
I dette projekt er der i alt blevet brugt fem forskellige former for tests. I begyndelsen af projektperioden blev der udformet en papir-prototype for at illustrere gruppens idé til et system. Da udviklingen af systemet var i gang, blev der lavet to spørgeskemaundersøgelser med informanter, som evaluerede systemet. Derefter blev der af to gruppemedlemmer testet om systemet levede op til Google og Apples designkrav for en mobilapplikation til henholdsvis Android og iOS.  

\subsection{Papirprotoype} % her
Projektgruppen brugte i begyndelsen en papirprototype til hurtigt at udvikle en skitse af hvordan systemet skulle se ud, og hvilke funktioner det skulle have. Dette skabte et grundlag for selve udviklingen af systemet, da gruppen så vidste hvilke funktioner der skulle laves, og nogenlunde hvilken rækkefølge de skulle laves i.

\subsection{Spørgeskemaundersøgelser}
Den første brugertest i dette projekt blev lavet som en spørgeskemaundersøgelser, projektets informanter blev bedt om, at udføre en række opgaver ved hjælp af projektets system. Når disse opgaver var løst, skulle informanterne udfylde et spørgeskema, som tog udgangspunkt i de netop udførte opgaver. I dette projekt er der blevet anvendt åbne spørgeskemaer ved disse undersøgelser, da det gav informanterne mulighed for at uddybe deres svar. Resultaterne fra disse spørgeskemaer blev brugt til at rette fejl og implementere forbedringer i systemet.

\subsection{Brugertest i usability laboratorium}
I dette projekt er der blevet brugt en del tid på brugertest i et usability laboratorium. Her blev det undersøgt hvordan to testpersoner der ikke havde set programmet før, ville løse nogle stillede opgaver. Metoden er baseret på to forelæsninger om brug af usability labbet, og ideen var at finde ud af, hvordan en ny bruger af hjemmesiden ville agere med den. Derfor var det vigtigt at opgaverne var stillet korrekt så testpersonerne ville komme ind omkring alle de funktioner der var ønsket testet, og at opgaverne mindede så meget som muligt om realistisk brug \citep{DEBUL}.

Resultaterne fra denne test blev brugt til at finde forslag til forbedringer og få en bedre forståelse af, hvordan potentielle brugere ville bruge systemet. Efter selve testen var afsluttet blev der holdt et interview med testpersonerne. Til testen var der tilknyttet en tidslogger og en testobservatør, som skrev alt, der blev observeret i testen, ned. Dette gjorde at materialet fra testen var detaljeret nok til at yderligere analyse af filmene var unødvendig. Bagefter blev rettelserne samlet og de forskellige forslag til rettelser og problemer blev analyseret og der blev vurderet på, om de skulle implementeres, om de ikke var relevante, eller om der ikke var ressourcer til det.

\subsection{Brugertest i virkelig indkøbssituation}
Der blev foretaget en enkelt brugertest i en virkelig indkøbssituation, hvor en person blev overvåget mens denne handlede ved hjælp af projektets system. Denne test blev foretaget for at undersøge, hvordan systemet virkede i virkeligheden, og for at finde svagheder og fejl, som måske ikke kunne findes i et laboratorium eller via et spørgeskema.
