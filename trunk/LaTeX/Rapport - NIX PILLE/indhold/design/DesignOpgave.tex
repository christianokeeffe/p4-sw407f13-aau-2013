%tjek op imod bogen, og det passer det der er skrevet. altså om de forskellige kriterier er beskrevet ud fra de de rent faktisk betyder.
I dette afsnit vil der blive set nærmere på de forskellige kvalitetsmål og hvordan de er prioriteret i dette projekt.
Prioriteringer vil blive illustreret i en tabel, hvorefter de individuelle prioriteringer vil blive beskrevet og begrundet.

\subsection{Formål}
Produktløsningen har til formål at hjælpe med koordineringen af indkøb, tilbudssøgning på varer og forslå varer der ofte bliver købt ved at analysere den pågældende kundes indkøbsvaner.
Gennem disse funktionaliteter skal produktløsningen hjælpe kunderne i deres handel og være et alternativ til en almindelige indkøbsliste eller andre løsninger.

\subsection{Rettelser til analysen}
\label{sec:rettelsertilanalyse}
Der er foretaget omfattende ændringer i klassediagrammet, se figur \ref{fig:Klassediagram}, fra analysedokumentet, da nogle af klasserne blev fundet trivielle for produktløsningen. Her er der tale om at ti ud af de femten klasser blev fundet ubetydelige for problemløsningen. De klasser der er taget med er \gaas{Kunde}, \gaas{Indkøbsliste}, \gaas{Vare}, \gaas{Tilbud} og \gaas{Butik}. De fravalgte klasser er blevet frasorteret, da der ikke skal tages specielt hensyn til disse i problemløsningen. De fem valgte klasser dækker alle de informationer, der vil blive nødvendige i problemløsningen.

\subsection{Kvalitetsmål}
\label{sec:kvalitetsmaal}
For at få en mere retningsbestemt udvikling, er der blevet specificeret og prioriteret nogle kvalitetsmål. Kvalitetsmålene fungerer som retningslinjer for hvilke områder, der vil blive fokuseret på og illustrerer hvilke områder, der er vigtige for systemet. Disse kvalitetsmål og deres prioriteringer indsættes i en tabel for at danne en sammenhæng mellem dem. Et overblik over, hvad der er prioriteret kan ses på tabel \ref{tab:kriterier}. Kvalitetsbogen er baseret på definitionerne i \citep{ooaogd}.

\begin{table}[H]
\centering
\begin{tabular}{|c | c | c | c | c | c | c|}
\hline
\textbf{Kvalitetsmålene} & \rotatebox{90}{Meget vigtigt} & \rotatebox{90}{Vigtigt} & \rotatebox{90}{Mindre vigtigt} & \rotatebox{90}{Irrelevant} & \rotatebox{90}{Trivielt opfyldt~}\\ \hline
Brugbart        & x &   &   &   &      \\ \hline
Forståeligt     & x &   &   &   &      \\ \hline
Pålideligt      &   & x &   &   &      \\ \hline
Effektivt       &   & x &   &   &      \\ \hline
Korrekt         &   & x &   &   &      \\ \hline
Testbart        &   & x &   &   &      \\ \hline
Vedligeholdbart &   &   & x &   &      \\ \hline
Flytbart        & x &   &   &   &      \\ \hline
Genbrugbart     &   &   &   & x &      \\ \hline
Sikkert         &   &   & x &   &      \\ \hline
Fleksibelt      &   &   & x &   &      \\ \hline
Integrerbart    &   &   &   & x &      \\ \hline

\end{tabular}
\caption{Denne tabel fungerer som en oversigt over, hvordan de forskellige kriterier er prioriteret i forhold til hinanden.}
\label{tab:kriterier}
\end{table}
\textbf{Brugbart:}
Systemet skal være brugbart, da brugerne anvender det i deres dagligdag. Heri ligger det, at systemet skal være let anvendelig både stående, gående eller siddende da brugerne befinder sig i en af disse tre tilstande, når de handler ind. Og da det skal bruges til en bred vifte af brugere, skal det være let og simpelt at bruge, så der ikke er for mange funktioner og lignende man skal sættes ind i. Dette vil sige at systemet skal være designet mod mobile enheder, da det er den platform der egner sig bedst til kundernes handel. Derfor er \gaas{brugbart} prioriteret \gaas{meget vigtigt}.

\textbf{Forståeligt:}
For at gøre det lettere for brugerne skal systemet også være let forståeligt. Dette gøres ved at gøre det overskueligt. Derfor er \gaas{forståeligt} også prioriteret \gaas{meget højt} ligesom \gaas{brugbart}.

\textbf{Pålideligt:}
Udover at systemet skal være brugbart, skal det også være pålideligt, da brugerne altid skal kunne stole på systemet. Hvis brugerne får givet forkerte oplysninger kan det lede til, at brugerne ikke får koordineret indkøbene efter hensigten, og derved mister problemløsningen et af sine primære formål. Det nytter heller ikke, at systemet tilføjer ting til indkøbslisten som ingen af brugerne har skrevet ind. Derfor er det vigtigt at systemet er pålideligt, da dette sikrer en meget bedre brugeroplevelse af problemløsningen. Det er naturligvis ikke muligt at sikre pålideligheden i alle situationer, eksempelvis kan brugerens internetforbindelse blive tabt. Der menes derfor at den del af systemet brugerne kan påvirke skal være pålideligt.

\textbf{Effektivt:}
Kvalitetsmålet \gaas{effektivt} betyder at problemløsningen skal fungere i en rimelig hastighed således brugeren ikke skal vente længe på, at problemløsningen henter eller sender data. Dette kvalitetsmål er sat til \gaas{vigtigt}, da systemet gerne skulle fungere så hurtigt som det nu engang er muligt for det givne system. Samtidig skal det udføre så få database-kald og API-kald som muligt, netop for at mindske ventetiden.

\textbf{Korrekt:}
Systemet skal også følge de mål, der er blevet sat i projektet så godt som muligt, så systemet rent faktisk opfylder de krav, der er blevet stillet. \gaas{Korrekt} er derfor prioriteret som værende \gaas{vigtigt}. Derudover skal der være konsistens i systemet, således at brugerne kan forvente at kunne navigere ensartet gennem hele systemet. 

\textbf{Testbart:}
Systemet skal være testbart, således at det er muligt at teste de forskellige kvalitetsmål og derved bedømme om den ønskede korrekthed er opnået. Dette prioriteres som \gaas{vigtigt} da det i forhold til \gaas{brugbart} eller \gaas{forståeligt} ikke ses lige så vigtigt, men det er lige så vigtigt som \gaas{effektivt} da det kan gøre systemet nemmere for brugeren at bruge.

\textbf{Vedligeholdbart:}
For at have mulighed for at vedligeholde systemet, skal systemet udvikles og udformes på en måde, der gør det nemmere at udskifte dele som ikke længere virker i sammenhæng med resten af systemet. Derfor er \gaas{vedligeholdbart} blevet prioriteret som værende mindre vigtigt, da systemets informationskilde ikke kommer til at ændre sig meget.

\textbf{Flytbart:}
\gaas{Flytbart} er prioriteret som \gaas{meget vigtigt}, idet det er vigtigt at løsningen kan fungere på forskellige enheder, da det er sandsynligt at eksempelvis par, som vil benytte løsningen sammen, ikke har ens enheder, eller at den enhed de bruger i hjemmet er en anden enhed, end den de tager med sig når de skal handle. Derfor bør løsningen fungere på forskellige platforme, og med forskellige browsere. 

\textbf{Genbrugbart:}
Det anses ikke for nødvendigt at systemet skal være \gaas{genbrugbart}. Derfor er dette kvalitetsmål også prioriteret som værende \gaas{irrelevant}.

\textbf{Sikkert:}
Systemet skal også i et vist omfang være sikkert for brugerne, da de bliver nødt til at logge ind i systemet for få adgang til deres indkøbslister og nogle bruger det samme kodeord på flere forskellige sider. Derfor er det blevet vurderet, at det er nødvendigt at have kryptering af \gaas{log ind} informationer. Dog er sikkerhed ikke fokuspunktet for dette system, idet systemet ikke behandler personfølsomme informationer. Grundet dette er sikkerhed blevet prioriteret som \gaas{mindre vigtigt} for dette system.

\textbf{Fleksibelt:}
For at sikre at systemet ikke går ned under brug, skal systemet være fleksibelt nok til at kunne håndtere forskellige fejl uden at gå ned. Derfor er \gaas{fleksibelt} kvalitetsmålet blevet prioriteret \gaas{mindre vigtigt} idet det ikke er ligeså vigtigt som de andre vigtige kvalitetsmål, men samtidig er relevant for det givne system.

\textbf{Integrerbart:}
Da systemet i dette projekt i denne omgang kun er tiltænkt integreret på webserveren, er integrerbart ikke vurderet til at være vigtigt, sammenlignet med de andre kriterier. Det er derfor vurderet til at være \gaas{irrelevant}.