\section{Oversigt over butikker} 
Denne del af hjemmesiden er lavet, så brugeren kan bestemme hvilke butikker, som søgningen skal finde tilbud i. Der bliver vist en sorteret liste over butikkernes forbogstaver, hvori butikkerne ligger.
Når man trykker på et bogstav, vil listen åbne sig se figur \ref{fig:GUIButikker}, således at butikker der starter med det bestemte bogstav vil blive vist. Butikkerne vil blive vist med deres navne og en skydeknap ved siden af navnet. Skydeknappens tilstand fortæller om søgningen skal tage den bestemte butik med eller ej. Skydeknappen har to tilstande den kan være i: \gaas{Til} og \gaas{Fra}. Første gang en butik kommer ind på denne liste, vil den være i tilstanden \gaas{Til}, som er den værdi, man giver butikker hvis søgningen skal tage butikken med.

Listen med butikker bliver lavet ved, at der først hentes den eksisterende liste af butikker i databasen. Derefter bliver der lavet en liste med butikker fra API'en. Her anvendes et API-kald, som henter butikkernes navne og ID'er. Derefter vil brugerens liste med butikker blive sammenlignet med listen med butikker fra API'en. Sammenligningen bruges til at finde ud, af om der findes butikker som ikke er med på brugerens liste over butikker. Hvis dette er tilfældet vil den/de manglende butikker blive tilføjet til listen af butikker med en startværdi \gaas{Til}. Listen indeholder også en parameter der gør det muligt at sætte skydeknappens startposition.

\gaas{Gem-knappen}, nederst på hjemmesiden, gør at butikkerne og deres tilstande gemmes i databasen. Første skridt i denne proces er at tjekke, om butikken allerede er eller ikke er med i databasen. Hvis butikken allerede eksisterer i databasen vil den blot redigere i den eksisterende butik. Hvis butikken ikke eksisterer i databasen vil systemet oprette en ny butik i databasen.
Når systemet enten skal oprette eller redigere en butik, vil den kigge på den pågældende butiks tilstand i listen over butikker. Tilstanden vil derefter blive afspejlet ved hjælp af en boolean variabel, \gaas{1} repræsenter at butikker skal tages med i søgningen og \gaas{0} at den ikke kan tages med.
På figur \ref{fig:GUIButikker} ses GUI'en for denne funktionalitet og hvordan de forskellige informationer angående butikker og deres tilstande bliver repræsenteret for brugeren.