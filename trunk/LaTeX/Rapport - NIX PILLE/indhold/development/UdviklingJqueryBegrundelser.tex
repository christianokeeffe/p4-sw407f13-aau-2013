\section{Mobiloptimering af side}
For at gøre løsningen mobil-optimeret bruges et touch-optimeret web-framework kaldet jQuery Mobile \citep{jquerymobile}. jQuery Mobile blev valgt, fordi det var nemt at komme i gang med, anerkendt, veldokumenteret samt gav et godt mobiloptimeret resultat. For at bruge jQuery skulle der inkluderes tre linjer kode i løsningens SiteMaster. Dette gøres for at hjemmesiden kan generere det tema samt funktioner jQuery Mobile tilbyder. De tre linjer kode inkluderer et typografiark, jQuery Mobiles JavaScript som giver dens funktioner og jQuery framework som kan ses på kodeudsnit \ref{lst:JQuery}.

\kode{Inkludering af jQueryMobile.}{JQuery}{JqueryMobile.txt}
SiteMasteren er hovedfilen som er en skabelon for hele hjemmesiden, som bestemmer f.eks hjemmesidens skrifttype. Derefter ændrede jQuery Mobile automatisk de fleste elementer til at være mobiloptimerede. Dette gav dog nogle indledende problemer, da siden i forvejen var en smule mobiloptimeret, hvor jQuery Mobile så tog styringen, hvilket gjorde at n oget kode skulle tilpasses, så det passede bedre med jQuery Mobiles måde at håndterevisse elementer på.
For eksempel hvis der laves en knap, skal der ikke bekymres om bl.a. skriftstørrelsen er stor nok til at kunne læses på en mobiltelefon, det klarer jQuery Mobile selv. jQuery Mobile ændrer nemlig knappens udseende til at være en stor knap med stor skriftstørrelse, og tilpasser knappens bredde efter mobilens skærm.

HTML-elementer kan tildeles properties således at jQuery Mobile ikke ændrer på det pågældende element. I denne løsning er der enkelte elementer, som ikke bruger jQuery Mobile. Heriblandt det pop op-vindue, som vises når der trykkes på \gaas{Indstillinger} ved et tilbud. jQuery Mobile kunne i princippet godt være brugt, men til lige netop dette formål viste det sig at jQuery Mobiles håndtering af iframe-popops var dårlig. Hvis jQuery Mobiles løsning var blevet brugt, ville den pre-loade siderne i pop op-vinduerne, mens den loadede den forespurgte side.
Det vil sige at hvis indkøbslisten indeholder 100 varer og de skal vises, ville den finde tilbud på 100 forskellige varer, ved at loade indkøbslisten. Med den løsningen som er valgt, kaldet Colorbox \citep{colorbox}, bliver siden først indlæst, når der trykkes på knappen der aktiverer pop op-vinduet. Dette reducerer load-tiden af en indkøbsliste betydeligt! Hvilket fremmer kvalitetsmålet \gaas{effektivt}. Det skal dog nævnes at jQuery Mobiles sidenavigering bliver brugt i de pop-ops, som optræder ved indkøbslisterne. 