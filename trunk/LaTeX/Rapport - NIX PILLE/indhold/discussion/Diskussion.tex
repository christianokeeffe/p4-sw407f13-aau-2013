I det følgende afsnit vil projektets mål blive diskuteret og efterfølgende vil der blive lavet en perspektivering.
\section{Kvalitetsmålenes opfyldelse}
Hvis man ser på kvalitetsmålene for systemet i sektion \ref{sec:kvalitetsmaal}, vurderes det at alle er blevet opfyldt.

%Brugbart
\gaas{Brugbart} er opfyldt ved, at problemløsningen er lavet til at kunne bruges på en mobil enhed og er enkelt i designet. For at gøre indkøbslisterne mere overskuelige ved mange varer, kan man bl.a. vælge at få sin indkøbsliste sorteret efter varekategori. Som det blev set ud fra testene i kapitel \ref{chap:testing} sagde testpersonerne undervejs at løsningen er meget brugbar, og dermed understøttes kvalitetsmålet også af testpersonerne.

%Forståeligt
\gaas{Forståeligt} er også opfyldt i problemløsningen. Testene viste at testpersonerne ikke behøvede en introduktion til problemløsningen, dog var der småproblemer undervejs, som blev udbedret, og dermed kan det dokumenteres at løsningen er forståelig.

%Flytbar
Da problemløsningen er en hjemmeside, er den i sig selv flytbar. Løsningen kan altså i princippet bruges på alle enheder som har en moderne browser og internetopkobling, heriblandt smartphones. Dog er siden optimeret til enheder med mindre skærme, såsom smartphones. \gaas{Flytbart}, som er prioriteret som \gaas{meget vigtig} er altså opfyldt. Dette har testpersonerne også understøttet, da de under testene sagde at de ingen problemer havde med at skulle skifte platform fra computer til mobil.

%Effektiv
\gaas{Effektivt} er delvist opfyldt. Problemløsningen skal kontakte både SQL-databasen og API'en, og nogle gange mange gange, afhængigt af hvilken, side der bliver forespurgt. Der er dog blevet arbejdet på at lave så få kald som muligt, for at reducere load-tider. For at opfylde dette kriterium fuldt ud, skulle systemet lave færre eksterne kald, til at forbedre hastigheden. Det menes dog at det ligger på et acceptabelt niveau, hvilket testpersonerne også har sagt. Det er acceptabelt, men kunne gøres bedre.

%Pålidelig
Problemløsningen er pålidelig, da den ikke går ned ved almindeligt brug. Derudover er der blevet arbejdet på at reducere API- og SQL-kald, således at mængden af data som skal sendes og modtages mindskes, da data-signalet på en smartphone kan variere meget i forskellige supermarkeder. Testpersonerne havde i starten nogle problemer med fejl, men i den endelige version af systemet er dette rettet.

%da den viser alle de tilgængelige tilbud, deres start og slut dato, prisen samt mængdeprisen, hvis den er med i information.

%Testbar
\gaas{Testbart} er opfyldt, fordi problemløsningen er en hjemmeside, hvilket gør at problemløsningen er nem og billig at teste, for både testpersoner og projektgruppen. Samtidig kan tests foretages mens testpersonerne køber ind - altså under normalt brug.

%Korrekt
Løsningen er korrekt, da den er konsistent. Designet af løsningen er konsistens da der bruges jQuery Mobile, som sørger for designet, hvilket gør det ensartet gennem hele løsningen. Selve koden er også relativt konsistent, da den er skrevet af få personer, og dermed i ensartet stil. Samtidig er løsningen skrevet både i C\# i den kodebaserede fil (code-behind fil), og HTML, CSS og JavaScript i selve layout-delene. Testpersonerne understøttede at løsningen var \gaas{korrekt}, da det var begrænset hvilke problemer der havde på de forskellige sider, i forhold til at forstå hvad de skulle. De sagde også at systemet var konsistent i forhold til når man skiftede platform. Kvalitetsmålet \gaas{korrekt}, som er prioriteret som \gaas{vigtigt}, er derfor også opfyldt.

%Fleksibel
Løsningen er fleksibel i den forstand at den er forholdsvis modulært opbygget, med forskellige sider, som relativt let kan ombygges. Samtidig er løsningen let at udbygge, da man nemt kan tilføje nye sider og dermed ekstra funktionalitet. Kriteriet \gaas{fleksibelt} er dermed også opfyldt. 

%Vedligeholdbart
På trods af at \gaas{vedligeholdbart} er blevet prioriteret som \gaas{mindre vigtigt}, er løsningen stadig vedligeholdbar. For det første er der gennem hele løsningen blevet skrevet kodekommentarer. Derudover er løsningen delt op i undersider, således man kan ændre en underside og blot offentliggøre den, frem for at skulle uploade hele siden til en server hver gang man har lavet en mindre ændring. 

%Genbrugbar
Problemløsningen er ikke \gaas{genbrugbart}. Systemet er ikke pakket ind i komponenter og de enkelte dele, kan derfor ikke nemt genbruges i et andet system. Kriteriet \gaas{genbrugbart} som er prioriteret som værende \gaas{irrelevant}, og dermed anses det ikke som noget problem at det ikke er blevet opfyldt.

%Sikker
Der er blevet gjort noget ud af at problemløsningens sikkerhed. Når brugeren opretter en profil på siden, bliver kodeordet krypteret med \gaas{bcrypt}. Men udover password-krypteringen er der ikke gjort noget yderligere for at sikkerhed-optimere løsningen. \gaas{Sikkert}, som er prioriteret \gaas{mindre vigtigt}, er altså også opfyldt.

%Integrerbart
Løsningen er ikke umiddelbart integrerbar, da den ikke nemt kan integreres med andre systemer. Men \gaas{integrerbart} er ifølge kvalitetsmålene ved afsnit \ref{sec:kvalitetsmaal} også \gaas{irrelevant} i forhold til dette projekt. 


Dermed opfylder løsningen altså alle de fremsatte krav. Løsningen gør det muligt at oprette indkøbslister, finde tilbud på varer, og kan samtidig bruges til at krydse varer af på indkøbslisten løbende mens man handler. At bruge løsningen som erstatning til den store mængde ugentlige tilbudsaviser gavner på længere sigt miljøet, da mindre papir bliver brugt. Løsningen muliggør også at eksempelvis par kan handle i flere butikker samtidig med den samme indkøbsliste. Indkøbslisten bliver således opdateret løbende, mens man bruger den, også selvom flere brugere bruger den på samme tid. Indbygget i indkøbslisten er også start- og sluttidspunkter for tilbud. Man bliver altså advaret hvis et tilbud ikke er begyndt endnu, eller hvis et tilbud er udløbet. Det gør man f.eks. ikke på en håndskrevet indkøbsliste, medmindre man har angivet tilbudenes tidspunkter på den. 

I den følgende perspektivering, vil løsningen blive perspektiveret i forhold til eksempler i den virkelige verden. 