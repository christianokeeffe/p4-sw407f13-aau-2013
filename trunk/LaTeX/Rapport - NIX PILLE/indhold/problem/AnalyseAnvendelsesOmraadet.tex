\section{Anvendelsesområde}
Dette afsnit vil omhandle anvendelsesområdet, hvor der vil blive set nærmere på aktørerne og brugsmønstrene. Ud fra brugsmønstrene vil der blive lavet funktioner der vil håndtere brugsmønstrene og give den ønskede funktionalitet til systemet.
Brugergrænsefladen vil også blive gennemgået (se sektion \ref{sec:brugerflade}) og vist gennem nogle eksempler, der illustrerer en mulig brugergrænseflade for systemet.

\subsection{Brug}
For at opnå en bedre forståelse af, hvorledes systemet skal hjælpe kunderne, vil der blive set nærmere på aktørerne og deres brugsmønstre.

\subsubsection{Oversigt}
Der er blevet identificeret to aktører: En indkøber og en planlægger og i alt seks brugsmønstre, der involverer disse to aktører. Det kan dog godt være en person, som har begge aktørroller. Disse kan ses i tabel \ref{tab:Aktoertabel}.

\begin{table}[H]
\centering
    \begin{tabular}{|l|c|c|}
\hline
        ~                                       & Indkøberen & Planlæggeren  \\ \hline
        Tilføj til indkøbsliste                 & ~          & x             \\ \hline 
        Slet fra indkøbsliste                   & ~          & x             \\ \hline
        Markér vare                             & x          & ~             \\ \hline
        Del liste                               & ~          & x             \\ \hline
        Opret indkøbsliste                      & ~          & x             \\ \hline
        Slet indkøbsliste                       & ~          & x             \\ \hline
        Find tilbud                             & ~          & x             \\ \hline
    \end{tabular}

\caption{Aktørtabellen over de fundne aktører samt deres brugsmønstre.}
\label{tab:Aktoertabel}
\end{table}

\subsubsection{Aktører}
\begin{table}[H]
    \begin{tabularx}{\textwidth}{R}
    	\hline
        \textbf{Indkøberen} \\ \hline
        \textbf{Formål:} Er en person som handler ind ud fra en indkøbsliste. \\
        \textbf{Karakteristik:} Arbejdet udføres af personer med varierende teknologisk erfaring.   \\
        \textbf{Eksempler:} Økonomisk bevidst forbruger, som lever på et stramt budget, besidder en 
        basal teknologisk erfaring og har en smartphone.\\
        Et ungt par med begrænsede midler, middel teknologisk erfaring, begge har en smartphone.
    \end{tabularx}
\end{table}

\begin{table}[H]
    \begin{tabularx}{\textwidth}{R}
    	\hline
        \textbf{Planlæggeren} \\ \hline
        \textbf{Formål:} En person, som benytter en indkøbsliste til at planlægge sine daglige indkøb ved hjælp af en indkøbsliste,
        og som samtidig interesserer sig for at købe sine dagligvarer billigst muligt.\\
        \textbf{Karakteristik:} Arbejdet udføres af personer med varierende teknologisk erfaring og med sans for overblik. \\
        \textbf{Eksempler:} Økonomisk bevidst forbruger, som lever på et stramt budget, besidder en basal teknologisk erfaring og har en smartphone.\\
        Et ungt par med begrænsede midler, middel teknologisk erfaring og begge har en smartphone.
    \end{tabularx}
\end{table}

\subsubsection{Brugsmønstre}

De syv brugsmønstre kan kategoriseres i grupperne \gaas{Administrering af varer på indkøbslisten}, \gaas{Administrering af indkøbslisterne} og \gaas{Find tilbud}.

\textbf{Administrering af varer på indkøbslisten}

\begin{itemize}
	\item Tilføj til indkøbsliste
	\item Slet fra indkøbsliste
	\item Markér vare 
\end{itemize}

Mønster: En aktør ønsker at indsætte en vare på en indkøbsliste, og derfor skal systemet kunne håndtere dette. Desuden vil aktøren gerne kunne slette varer fra indkøbslisten igen, når de er købt. En aktør skal kunne markere de varer han/hun har lagt i indkøbsvognen, således at aktøren er klar over, hvilke varer han/hun mangler at finde. Derfor skal der være en funktion i systemet, der kan holde styr på, hvad aktøren har \gaas{købt} på det bestemte tidspunkt.

\textbf{Administrering af indkøbslisterne}

\begin{itemize}
	\item Del liste
	\item Opret indkøbsliste
	\item Slet indkøbsliste
\end{itemize}

Mønster: Det skal være muligt for en aktør at kunne oprette en indkøbsliste og tilføje varer og tilbud til denne. Desuden skal det være muligt at slette indkøbslisten igen når den har tjent sit formål. Hvis en aktør ønsker at dele en indkøbsliste med en anden aktør, skal systemet kunne håndtere flere aktører der deles om en indkøbsliste. Således en vare eller tilbud ikke bliver købt flere gange af de forskellige aktører der deler indkøbslisten.

\textbf{Find tilbud}

\begin{itemize}
	\item Find tilbud
\end{itemize}

Mønster: Det skal være muligt for aktører at bruge systemet til at finde tilbud. Altså skal der være en funktion til at søge efter tilbud i løsningen.

\subsection{Funktioner}
De nødvendige funktioner for systemet der skal udvikles i dette projekt kan ses på tabel \ref{tab:funktionstabel}. Funktionerne er designet så de understøtter hændelserne angivet i problemområdet, som kan ses i tabel \ref{tab:haendelsestabel}.
\begin{table}[H]
\centering
    \begin{tabular}{|l|c|c|}
        \hline
    Funktioner                                          & Kompleksitet & Type        \\ \hline
	Find tilbud                                         & Medium       & Aflæsning   \\ \hline
	Opret indkøbsliste                                  & Simpel       & Opdatering  \\ \hline
	Knyt person til en indkøbsliste                     & Simpel       & Opdatering  \\ \hline
	Slet indkøbsliste                                   & Simpel       & Opdatering  \\ \hline
	Tilføj vare til indkøbsliste                        & Simpel       & Opdatering  \\ \hline
	Slet vare fra indkøbsliste                          & Simpel       & Opdatering  \\ \hline
	%Find ofte købte varer                               & Kompleks     & Beregning og aflæsning  \\ \hline
	%Find tilbud ud fra ofte købte varer                 & Medium       & Beregning   \\ \hline
	Foreslå tilbud                                      & Kompleks     & Signal, Beregning og aflæsning    \\ \hline
    \end{tabular}
\caption{Tabel over funktioner baseret på hændelsestabellen.}
\label{tab:funktionstabel}

\end{table}
På tabel \ref{tab:funktionstabel} kan det ses, at der for systemet findes en række simple funktioner. Derudover findes der en kompleks funktion samt en medium. \gaas{Find tilbud}-funktionen inkluderer at finde et tilbud til kunden, ud fra nogle givne kriterier som fx en søgestreng. Funktionen der er en aflæsnings-funktion, skal have en kilde med tilbud den kan aflæse. For at finde fremtidige tilbud skal systemet kunne foreslå tilbud til kunden ud fra tidligere køb. Dette er en kompleks funktion, fordi en person der bruger et sådan system naturligt vil skrive sine indkøb på mange forskellige måder, samt tilføje tilbud med forskellige titler. Denne komplekse funktion \gaas{Foreslå tilbud} skal altså gå igennem kundens indkøbslister og finde de varer der ofte bliver købt. Derefter vil funktionen ud fra resultaterne søge efter tilbud og vise forslagene til kunden.

\subsection{Brugergrænsefladen}
\label{sec:brugerflade}
I dette afsnit vil dialogformen blive beskrevet. Der vil desuden blive vist skitser af brugergrænsefladen, og tanken bag disse vil blive beskrevet.
\subsubsection{Dialogform}
For at aktørerne kan navigere rundt er der blevet lavet en menu der gør navigationen nemmere. Når aktører kommer med input vil der enten være en tekstboks eller en knap således at aktøren klart kan se formålet med inputtet og hvad systemet skal bruge inputtet til.
%Hvis aktøren skulle bruge kommandosprog, ville det tage lang blive lære at bruge systemet og det kunne være vanskelig hele tiden at skrive kommandoer mens man køber ind. Især på en mobilenhed hvor man som regel ikke har et keyboard til at taste ind med.
Brugergrænsefladen vil have et vindue med søgefunktionen, indkøbslisterne og de tilbud, som systemet foreslår ud fra brugerens indkøbsvaner. Ud over dem, vil der være indstillinger, hvor aktøren vil kunne indstille forskellige indstillinger ved systemet, såsom hvilke butikker aktøren vil have tilbud fra. Nedenstående liste viser en oversigt over de forskellige vinduer.

\begin{itemize}
	\item Hovedvindue
	\item Log ind
	\item Søge-funktion
	\item Indkøbslister
	\item Foreslåede tilbud
	\item Indstillinger
\end{itemize}

%Vi vælger en store menu-knapper til at navigere rundt på siden, og skemaudfyldelse til indtastning af data. Brugerne kan direkte manipulere dele af hjemmesiden, såsom at krydse elementer på indkøbslisten af. 
%Har vi lavet så man kan scanne stregkoder? Så bør der nok stå noget om det her.. ;)
%Det ville ikke give mening at bruge kommandoer på en mobil-optimeret hjemmeside. Brugergrænsefladen har separate sider, som er forbundet af links på de individuelle sider. Således kan brugeren blive guidet til de relevante funktioner ud fra den side brugeren ser på nuværende tidspunkt.

\subsubsection{Oversigt}
\figur{1.0}{AnalyseOversigtOverGUI.png}{Diagram over strukturen mellem de forskellige vinduer.}{fig:OversigtOverGUI} %billede nu sat ind men det skal huskes at billede også skal rettes
På figur \ref{fig:OversigtOverGUI} kan der ses et navigerings-diagram, der giver et overblik over, hvordan aktører kommer rundt i systemet. Vinduerne er baseret på skabelonerne og giver en generel idé om hvordan løsningen kommer til at %er det det rigtige ord? 
se ud på en mobilenhed. %Alle funktionerne i systemet som aktøren kan bruge bliver vist i vinduerne, hvilket gør at der skabes et vist overblik over systemet. 

\subsubsection{Eksempler}
Der er blevet lavet nogle skabeloner for at give et generelt indblik i, hvorledes systemet grænsefladen mellem aktøren og systemet kan se ud.

\figur{0.4}{StartskaermEndelig.png}{Her ses en skitse over hvordan det var planlagt, startskærmen skulle se ud.}{fig:Startskaerm}
Figur \ref{fig:Startskaerm} viser startskærmen, som giver adgang til alle hjemmesidens andre funktioner. Via tryk på billederne var det meningen, at en bruger skulle sendes videre til den valgte funktion, og ved at trykke på \gaas{log ind}-knappen skulle det være muligt at logge ind.

\figur{0.4}{LogIndSkaermEndelig.png}{Her ses en skitse over, hvordan man skulle logge ind.}{fig:LogInd}
Figur \ref{fig:LogInd} viser, hvordan \gaas{Log ind}-boksen oprindeligt var designet.

\figur{0.4}{SoegeFunktionEndelig.png}{Her ses en skitse over, hvordan det var planlagt søgefunktionen skulle se ud.}{fig:Soegefunktion}
Figur \ref{fig:Soegefunktion} viser søgefunktionen, hvor brugeren ville have mulighed for at søge efter tilbud, og derefter kunne tilføje tilbuddet til en indkøbsliste.

\figur{0.4}{IndkoebslisteSorteretEfterButikker1Endelig.png}{Her ses en skitse over, hvordan udseendet til indkøbsliste blev tænkt, sorteret efter butikker.}{fig:ListeButikker}
Figur \ref{fig:ListeButikker} viser, hvordan indkøbslisten ville se ud hvis den var sorteret efter butikker først og varekategori derefter.

\figur{0.4}{IndkoebslisteSorteretEfterKategori2Endelig.png}{Her ses en skitse over hvordan indkøbslisten skulle se ud sorteret efter varekategori og derefter inddelt efter butikker.}{fig:ListeKategorier}
Figur \ref{fig:ListeKategorier} viser hvordan det oprindeligt var tænkt at indkøbslisten skulle se ud. Sorteret først efter kategori og derefter inddelt efter butik.

\figur{0.4}{ForslaaedeTilbudEndelig.png}{Her ses en skitse over, hvordan funktionen til at forslå tilbud, skulle se ud.}{fig:ForslaaedeTilbud}
Figur \ref{fig:ForslaaedeTilbud} viser funktionen til at foreslå tilbud. Funktionen skulle vise de tilbud en funktion havde analyseret sig frem til, som vil kunne interessere den enkelte bruger.

\subsection{Den tekniske platform}
Systemet skal udvikles som en webbaseret løsning, hovedsageligt tilpasset mobile enheder, men samtidig også tilgængeligt på en mere stationær enhed. Systemet skal programmeres i C\#, som også fungerer som et passende sprog i forhold til den objektorienteret model, der også er fortaget analyse ud fra. Systemet er rettet mod mobile enheder da det typisk vil være en sådan enhed, der vil blive benyttet mens man handler, frem for eksempelvis en bærbar computer.
Til systemet skal der bruges en database, hvor det er muligt at definere passende tabeller, således det er muligt at konstruere en relevant databasestruktur. Databasen skal også være tilgængelig for de eventuelle kommende brugere. 
Systemet skal opbygges efter almindelig hjemmesidestandarter og stadig være optimeret til mobile enheder.
Systemet skal betjenes via berøringsnavigation for de mobile enheder, og for de stationere enheder skal der bruges mus og tastetur.
Optimeringen rettet imod mobile enheder, skal sikre at systemet bliver mere brugervenligt angående berøringsnavigation. 