De nødvendige funktioner for systemet der skal udvikles i dette projekt kan ses på tabel \ref{tab:funktionstabel}. Funktionerne er designet så de understøtter hændelserne angivet i problemområdet, som kan ses i tabel \ref{tab:haendelsestabel}.
\begin{table}[H]
\centering
    \begin{tabular}{|l|c|c|}
        \hline
    Funktioner                                          & Kompleksitet & Type        \\ \hline
	Find tilbud                                         & Medium       & Aflæsning   \\ \hline
	Opret indkøbsliste                                  & Simpel       & Opdatering  \\ \hline
	Knyt person til en indkøbsliste                     & Simpel       & Opdatering  \\ \hline
	Slet indkøbsliste                                   & Simpel       & Opdatering  \\ \hline
	Tilføj vare til indkøbsliste                        & Simpel       & Opdatering  \\ \hline
	Slet vare fra indkøbsliste                          & Simpel       & Opdatering  \\ \hline
	%Find ofte købte varer                               & Kompleks     & Beregning og aflæsning  \\ \hline
	%Find tilbud ud fra ofte købte varer                 & Medium       & Beregning   \\ \hline
	Foreslå tilbud                                      & Kompleks     & Signal, Beregning og aflæsning    \\ \hline
    \end{tabular}
\caption{Tabel over funktioner baseret på hændelsestabellen.}
\label{tab:funktionstabel}

\end{table}
På tabel \ref{tab:funktionstabel} kan det ses, at der for systemet findes en række simple funktioner. Derudover findes der en kompleks funktion samt en medium. \gaas{Find tilbud}-funktionen inkluderer at finde et tilbud til kunden, ud fra nogle givne kriterier som fx en søgestreng. Funktionen der er en aflæsnings-funktion, skal have en kilde med tilbud den kan aflæse. For at finde fremtidige tilbud skal systemet kunne foreslå tilbud til kunden ud fra tidligere køb. Dette er en kompleks funktion, fordi en person der bruger et sådan system naturligt vil skrive sine indkøb på mange forskellige måder, samt tilføje tilbud med forskellige titler. Denne komplekse funktion \gaas{Foreslå tilbud} skal altså gå igennem kundens indkøbslister og finde de varer der ofte bliver købt. Derefter vil funktionen ud fra resultaterne søge efter tilbud og vise forslagene til kunden.