\section{Problemområdet}
I denne sektion vil der blive set nærmere på problemområdet for dette projekt, fra en objektorienteret vinkel. Til at få et indblik i problemet bliver der udviklet et diagram, et så kaldt klassediagram. Et klassediagram anvender UML (Unified Modeling Language) notation og bruges til at illustrerer de forskellige relationer mellem de forskellige klasser i problemområdet. Klasserne beskriver objekter ved, at fungere som skabelon for grupper af objekter, der har fælles træk. Klassediagrammet er bleven fremstillet ud fra viden fra OOAD (Objektorienteret analyse og design) bogen \citep{ooaogd} og selve diagrammet kan ses på figur \ref{fig:Klassediagram}.

\begin{landscape}
\figur{1.3}{AnalyseKlassediagram.png}{Oversigt over Klassediagram.}{fig:Klassediagram}
\end{landscape}

\subsection{Struktur}
For at få en bedre forståelse af strukturen i klassediagrammet på figur \ref{fig:Klassediagram} vil dette afsnit behandle de forskellige strukturer i klassediagrammet og beskrive dem.

\subsubsection{Klassen Kunde}
Klassen \gaas{Kunde} er en abstraktion af kunder fra den virkelige verden der handler ind, som enten kan være et medlem eller ikke-medlem af et selskab, derfor er klassen \gaas{Medlemskab} associeret med klassen \gaas{Kunde}.

Til klassen \gaas{Kunde} er klassen \gaas{Indkøbsliste} associeret, da en kunde kan have nul til mange indkøbslister. En indkøbsliste skal dog altid tilhøre mindst én kunde før den bliver relevant, da en indkøbsliste uden en kunde tilknyttet, vil betyde, at den ikke indeholder nogle varer der skal huskes.
Klassen \gaas{Kunde} er associeret med klassen \gaas{Begivenhed}. Grunden til associationen mellem klassen \gaas{Kunde} og \gaas{Begivenhed} er, at en person kan oprette nul til mange begivenheder og en begivenhed kan have en til flere personer tilknyttet angående arrangeringen af begivenheden.

\subsubsection{Klassen Indkøbsliste}
\gaas{Indkøbsliste} klassen udgør en væsentlig del af problemområdet og illustrerer netop den fysiske indkøbsliste, som der kan være problemer med at få delt og koordineret mellem de relevante kunder.

\subsubsection{Klassen Begivenhed}
Klassen \gaas{Begivenhed} er en abstraktion af alle de forskellige arrangementer en kunde kunne være interesseret i at oprette. Denne klasse er taget med fordi en kunde kan være interesseret i at oprette en indkøbsliste specifikt til et bestemt arrangement. Dette er grunden til at klassen \gaas{Indkøbsliste} er associeret til \gaas{Begivenhed} klassen.

\subsubsection{Klassen Medlemskab}
Klassen \gaas{Medlemskab} er associeret til \gaas{Selskab} i et forhold 1. Grunden til forholdet er, at et selskab ikke behøves at have nogle medlemmer men samtidig kan have flere end et medlem og der kan kun være ét selskab tilknyttet til en type medlemskab. 
En kunde kan også have flere medlemskaber af forskellige selskaber.

\subsubsection{Klassen Vare}
Den abstrakte klasse \gaas{Vare} er en abstraktion over de forskellige elementer en kunde kunne tilføje til deres indkøbslister. Denne abstraktion generaliseres til to underklasser \gaas{Almindelige Vare} og \gaas{Tilbud}. 

\gaas{Vare}-klassen bliver associeret med \gaas{Indkøbsliste}, således at der i modellen bliver dannet en relation mellem vare og indkøbsliste. Denne relation dækker over at en indkøbsliste indeholder nul til flere objekter af \gaas{Vare}, mens der stadig kan eksistere varer selvom de ikke er tilknyttet en indkøbsliste. Dette gør det muligt for varer at optræde på flere kunders indkøbslister samtidigt.

\gaas{Vare}-klassen bliver også associeret med \gaas{Tilbudsavis}, så der også bliver dannet en relation mellem vare og tilbudsavisen. Denne relation går ud på, at en tilbudsavis indeholder en til mange varer og et vareobjekt kun er tilknyttet én tilbudsavis af gangen. Dette gør at tilbud kun optræder én gang i en given tilbudsavis og at ét selskabs tilbud ikke fremkommer i et andet selskabs tilbudsavis.

\subsubsection{Klassen Tilbud}
\gaas{Tilbud} generaliseres derimod ud i endnu fire underklasser: \gaas{Pakketilbud}, \gaas{Mængdetilbud}, \gaas{Enkelttilbud} og \gaas{Medlemstilbud}. Dette er gjort, fordi der findes forskellige tilbud og alle disse skal håndteres i modellen. 

Klassen \gaas{Almindelig vare} er aggregeret til \gaas{Pakketilbud}, således at \gaas{Pakketilbud} består af to til flere \gaas{Almindelig vare}. Dette er gjort fordi pakketilbud består af flere almindelige varer, som i en kombination udløser rabatten. \gaas{Mængdetilbud} og \gaas{Tilbud} er med i modellen, for at kunderne kan anvende disse typer af tilbud. \gaas{Medlemstilbud} klassen er tilbud, som kun gælder for medlemmer af den specifikke butik, derfor har den attributten \gaas{betingelser}.

\subsubsection{Klassen Tilbudsavis}
For at sikre at kunderne i denne model kan anskaffe sig informationer om butikkernes tilbud, er der oprettet en klasse \gaas{Tilbudsavis}. \gaas{Tilbudsavis} er associeret med \gaas{Vare} i et forhold, der beskriver, hvordan en tilbudsavis kan have flere varer. \gaas{Tilbudsavis} er også associeret med \gaas{Selskaber} i forholdet en til mange, for at illustrere, hvordan forskellige butikskæder i samme selskab har deres udgave af tilbudsavisen ude på markedet, på samme tid.

\subsubsection{Klassen Selskab}
For at kunne håndtere de forskellige butikker og deres forskellige tilbud, er der blevet genereret en klasse \gaas{Selskab}, som abstraherer over dette område. Selskaber har så underklasserne \gaas{Butikskæder} og \gaas{Selvstændig butik}, hvor den selvstændige butik forstås som en privatejet butik.
Til \gaas{Butikskæder} er der aggregeret en \gaas{Butik}-klasse på, i et forhold, som fortæller, hvordan en butikskæde kan indeholde flere butikker. \gaas{Butik}-klassen illustrerer de individuelle butikker i en butikskæde.

\subsection{Klasser}
I denne sektion vil de forskellige klasser, samt deres tilstandsdiagram kort blive beskrevet.

\subsubsection{Kunde}
\figur{0.75}{AnalyseKundeTilstandsdiagram.PNG}{Viser tilstandsdiagrammet for klassen \gaas{Kunde}.}{fig:Kunde-tilstandsdiagram}
Klassen indeholder oplysninger om kunderne, således der er en måde at identificere kunder og kæde dem sammen med hinanden. Se figur \ref{fig:Kunde-tilstandsdiagram}.

Attributter: Navn, Forhold.

\subsubsection{Medlemskab}
\figur{0.75}{AnalyseMedlemTilstandsdiagram.PNG}{Viser tilstandsdiagrammet for klassen \gaas{Medlemskab}.}{fig:Medlem-tilstandsdiagram}
Klasse der beskriver en kundes mulighed for at have et medlemskab i et selskab, der så udløser en rabat, forskellige goder eller en form for præmie. Se figur \ref{fig:Medlem-tilstandsdiagram}.

Attributter: Rabat, Medlems ID.

\subsubsection{Indkøbsliste}
\figur{0.75}{AnalyseListeTilstandsdiagram.PNG}{Viser tilstandsdiagrammet for klassen \gaas{Indkøbsliste}.}{fig:Liste-tilstandsdiagram}
Klassen omfatter selve indkøbslisten, som er en liste af varer som kunden skal handle ind. Se figur \ref{fig:Liste-tilstandsdiagram}.

Attributter: Kunde.

\subsubsection{Begivenhed}
\figur{0.75}{AnalyseBegivenhedTilstandsdiagram.PNG}{Viser tilstandsdiagrammet for klassen \gaas{Begivenhed}.}{fig:Begivenhed-tilstandsdiagram}
Klassen der beskriver et scenarie, hvor kunden kan oprette en indkøbsliste i anledning af en begivenhed. Se figur \ref{fig:Begivenhed-tilstandsdiagram}.

Attributter: Tidspunkt, Navn.

\subsubsection{Vare}
\figur{0.75}{AnalyseVareTilstandsdiagram.PNG}{Viser tilstandsdiagrammet for klassen \gaas{Vare}.}{fig:Vare-tilstandsdiagram}
På figur \ref{fig:Vare-tilstandsdiagram} ses tilstandsdiagrammet for klassen \gaas{Vare}-Klassen sammenfatter de forskellige typer af varer. Her menes enten tilbud eller almindelige varer, her forstås almindelige varer som varer hvor der til normalpris. Tilbud indeholder fire generaliseringer, som bliver beskrevet i tilstandsdiagrammet for tilbud som set på figur \ref{fig:Tilbud-tilstandsdiagram}.
 
Attributter: Navn, Antal, mængdepris, pris.

\subsubsection{Tilbud}
\figur{1}{AnalyseTilbudTilstandsdiagram.PNG}{Viser tilstandsdiagrammet for klassen \gaas{Tilbud}.}{fig:Tilbud-tilstandsdiagram}
Denne klasse sammenfatter de forskellige slags tilbud som eksisterer. Se figur \ref{fig:Tilbud-tilstandsdiagram}. Disse klasser er beskrevet i klassediagrammet der kan ses på figur \ref{fig:Klassediagram}.

Attributter: Butik, Start dato, Slut dato.


%skal umildbart ikke tages med
%\subsection{Pakke tilbud}
%Klasse for gruppe af forskellige almindelige vare som til sammen udløser en rabat.

%Attributter:

%\subsection{Mængde tilbud}
%Klassen for mængde køb af en vare som udløser en rabat.

%Attributter: Min antal, Max antal per kunde.

%\subsection{Enkelt tilbud}
%Klasse for et almindeligt tilbud, hvor nuværende pris er mindre end før prisen.

%Attributter: 

\subsubsection{Selskab}
\figur{0.75}{AnalyseSelskabTilstandsdiagram.PNG}{Viser tilstandsdiagrammet for klassen \gaas{Selskab}.}{fig:Selskab-tilstandsdiagram}
Klasse for et selskab der kan eje forskellige butikskæder eller være en sammenslutning for private butiksejere. Se figur \ref{fig:Selskab-tilstandsdiagram}.

Attributter: Navn.

%Virker heller ikke til at skulle bruges
%\subsection{Kæde}
%Klasse for sammenfatningen af forskellige butikker af samme butikskæde, der eksister på markedet. Attributten antal beskriver antallet af butikker i den %individuel kæder.

%Attributter: Antal.

%\subsection{Butik}
%Klasse for de individuelle butikker der kan findes i en butikskæde.

%Attributter: Navn, Placering.

%\subsection{Enkelt butik}
%Klasse der beskriver de private butikker som ikke tilhøre en butikskæde.

%Attributter: Placering.

\subsection{Hændelser}

Tabel \ref{tab:haendelsestabel} viser en oversigt over hændelserne for de mest essentielle klasser. Ikke alle klasserne fra klassediagrammet (se figur \ref{fig:Klassediagram}) er taget med fordi de enten kan indgå som en underklasse eller som ses mindre væsentlige.

\begin{table}[H]
\centering
    \begin{tabular}{|l|c|c|c|c|}
\hline
        ~                                  & Vare & Kunde & Indkøbsliste & Selskab \\ \hline
        Indkøbsliste oprettet              & ~    & x     & x            & ~       \\ \hline 
        Indkøbsliste slettet               & ~    & x     & x            & ~       \\ \hline
        Indkøbsliste delt med andre        & ~    & x     & x            & ~       \\ \hline
        Vare tilføjet / slettet fra liste  & x    & ~     & x            & ~       \\ \hline
        Tilbud søgt                        & x    & ~     & ~            & x       \\ \hline
        Mulige butikker til indkøb valgt   &  ~   & x     & ~            & x       \\ \hline
        Begivenhed planlagt                & ~    & x     & x            & ~       \\ \hline
        Indkøbsliste tilknyttet begivenhed & ~    & ~     & x            & ~       \\ \hline
        Medlemskab af selskab startet      & ~    & x     & ~            & x       \\ \hline
        Medlemskab af selskab sluttet      & ~    & x     & ~            & x       \\ \hline
        Tilbud oprettet                    & x    & ~     & ~            & x       \\ \hline
        Tilbud afsluttet                   & x    & ~     & ~            & x       \\ \hline 
        Tilbuds-avis udsendt               & ~    & x     & ~            & x       \\ \hline
        Ny butik åbnet                     & ~    & ~     & ~            & x       \\ \hline
        Butik lukket                       & ~    & ~     & ~            & x       \\ \hline
        Vare købt                          & x    & ~     & x            & ~       \\
\hline
    \end{tabular}

\caption{Tabel for de forskellige hændelser tilhørende de mest essentielle klasser som vare, kunde, indkøbsliste og selskab.}
\label{tab:haendelsestabel}
\end{table}
