\section{Fjerde test}
Her beskrives detaljerne omkring fjerde omgang af tests.

\subsection{Testpersonerne}
Denne gang testede tre af gruppemedlemmerne systemet, for at se om testen var klar til testpersonerne. Dagen efter testede to udefrakommende personer systemet. Begge testpersoner var studerende kvinder på 20 år, med en middel teknologisk erfaring.


\subsection{Systemets tilstand}
På tidspunktet for fjerde test var alle kernefunktioner i systemet udviklede, og skulle derfor gennemtestes, så kritiske fejl kunne findes og rettes. 

\subsection{Kvalitetsmål i fokus}
Kvalitetsmålene \gaas{pålideligt}, \gaas{effektiv}, \gaas{korrekt}, \gaas{testbar} og \gaas{flytbar} blev forsøgt testet. Kvalitetsmålet \gaas{flytbart} blev testet da det bl.a. ønskedes testet om en bruger kunne bruge en del af systemet på en almindelig computer, og derefter arbejde videre med en anden del på en mobil platform. \gaas{Effektiv}, \gaas{korrekt} og \gaas{pålidelig} blev testet ved at testpersonerne i usability laboratoriet blev bedt om at lade som om de skulle handle ind i virkeligheden. Dermed blev det opdaget, hvis systemet pludseligt gik ned eller ikke reagerede som forventet. 

\subsection{Testmetoden}
Der blev foretaget tests af hele hjemmesiden i universitets usability laboratorium. Testpersonerne blev filmet mens de udførte seks opgaver med systemet, og fortalte hvad de gjorde og hvorfor. Opgaverne gik ud på at teste de forskellige dele af systemet på computer og smartphone. Desuden gik en af opgaverne ud på, at testpersonerne skulle bruge systemmet mens de gik rundt som om de handlede. Efter opgaverne var gennemført blev der afholdt et semi-struktureret interview med testpersonen inde i testlokalet der også blev filmet. Et af gruppemedlemmerne sad inde hos testpersonen og spurgte ind til det der skete under testen og hjalp hvis en testperson sad fast i en af opgaverne. Desuden var det også denne person der interviewede testpersonerne. Fra et kontrolrum fulgte en tidslogger samt en testobservatør, hvis opgave var at skrive ned, hvis der skete noget interessant og derudover en person til at styre kameraerne. Udgangspunktet for denne testmetode er \citep{DEBUL}.

\subsection{Observationer fra fjerde test}
I dette afsnit beskrives observationerne fra testene. Det er delt op i afsnit efter de forskellige funktioner der blev testet. Dette giver overblik over evalueringen af de forskellige dele af systemet.\newline
\newline
\textbf{Indkøbslister på computeren:}\newline
Alle testpersonerne, inklusiv de tre gruppemedlemmer, glemte at give indkøbslisten et navn, da de ville lave en ny liste. Desuden klagede flere testpersoner over, at det ikke var muligt at se hele teksten i et tilbud, og derfor kom de til at tilføje forkerte tilbud til indkøbslisten. Flere af testpersonerne kom også til at lave fejlen, at de krydsede en vare af i stedet for at åbne indstillingerne flere gange. En af testpersonerne kom til at ramme forsideknappen i stedet for feltet til at skrive en vare ind, da disse felter ligger tæt på hinanden.

Det var et problem for to af testpersonerne at man ikke kunne se at et tilbud man havde valgt rent faktisk blev tilføjet. Den ene ønskede at pop op-vinduet automatisk lukkede. Der skete en underlig sortering af listen der fik en af varerne til at flytte op og ned på listen. Flere at testpersonerne klagede over, at titlerne på tilbuddene kunne være meget intetsigende og ønskede at kunne se en beskrivelse. Flere af testpersonerne ledte efter en \gaas{del}-knap under selve indkøbslisten. Desuden bemærkede en af testpersonerne, at han først ikke kunne se nogen tilknyttede brugere på listen, og da han åbnede den igen stadig ikke kunne se sig selv. Den ene af testpersonerne forsøgte at tilgå indkøbslisterne uden at logge ind/lave en bruger først. En af testpersonerne forsøgte at sortere listen efter butikker uden at have fundet nogen tilbud. Desuden foreslog en af testpersonerne at systemet selv skulle kunne finde tilbud ud fra ens indkøbsliste, så man kunne få det billigst muligt, men nøjes med at handle i én butik.

En af testpersonerne lavede desuden fejlen, at testpersonen kom til at slette en vare, da testpersonen ville se hele titlen på et tilbud og derfor trykkede på pilen ud for \gaas{fjern varen}. En af testpersonerne oplevede, at da et tilbud blev valgte og derefter et andet tilbud skete der ikke noget anden gang. Begge testpersoner var i tvivl om de havde valgt det bedste tilbud. De klagede også over der manglede detaljer i tilbuddet som mængde, mængdepris, type og beskrivelse. Begge testpersoner overser også \gaas{eller} i teksten ved \gaas{del liste} hvilket gør de efterspørger en mail, når de allerede har et brugernavn. Dette gør også at den ene testperson bliver forvirret, over at der er to tekstfelter.

Den ene testperson bemærker at der også i pop op-vinduet er en dårlig sortering af listen over tilbud. En anden testperson følte ikke det var intuitivt, at man skulle oprette en indkøbsliste før man kunne tilføje varer via systemet. Desuden blev der foreslået at markøren automatisk skulle returnere til boksen, hvor man skriver vare ind, efter at have tilføjet en vare og at afkrydsede varer automatisk skulle fjernes efter et døgn. \newline
\newline
\textbf{Indkøbslister på en smartphone:}\newline
En af testpersonerne klagede over at det var svært at skrive varer ind på smartphonen da tastaturet var for småt. Også på smartphonen havde flere testpersoner problemer med at kende forskel på menuen og hvordan man markerede en vare som købt. En af testpersonerne overså blandt andet feltet til at skrive antal i, da testpersonen arbejdede med smartphonen. Her blev der også klaget over man ikke kunne se alle detaljer i pop op-vinduet.

En af testpersonerne ønskede desuden at have mulighed for at klikke på billedet af tilbuddet og så se det i stor udgave. Samme testperson overser sorteringsknappen og kan derfor ikke finde ud af, hvad listen er sorteret efter. Markeringerne blev underlige når to personer arbejdede samtidigt, og det gav fejl i markeringerne. Desuden gik der noget tid før testpersonerne lagde mærke til at en anden havde købt noget, og i de første tre test opdagede de det slet ikke. Der blev klaget over at systemet var langsomt og at der kunne gå op til 12 sekunder inden systemet registrerede, at en vare var blevet markeret, og dette gjorde at en testperson trykkede flere gange uden resultat. Desuden skete der en visningsfejl i den ene af testene. En af testpersonerne glemte helt at krydse varerne af, da testpersonen handlede trods det stod i opgaven. Desuden fandt en af testpersonerne listen uoverskuelig og ønskede at kunne sortere den efter varekategori.\newline
\newline
\textbf{Foreslåede tilbud:}\newline
En fejl er, at der bliver forslået tilbud der allerede findes på indkøbslisten.Testpersonerne fandt hurtigt ud af, hvordan funktionen virkede. De to kvindelige testpersoner leder desuden efter funktionen under indkøbslisterne. Et andet problem var, at der ingen respons var omkring at en vare rent faktisk var tilføjet til indkøbslisten. En af testpersonerne mente stadig der var mange irrelevante tilbud og klagede over det samme tilbud optrådte flere gange, trods det var i forskellige butikker, og foreslog de skulle kombineres under ét hvis det var samme vare.\newline
\newline
\textbf{Indstillinger:}\newline
En af testpersonerne trykkede \gaas{bekræft} i indstillingerne, selv om dette ikke var nødvendigt. Samme person forsøgte at finde \gaas{gem knappen} i toppen af butikslisten, mens den var i bunden. Den ene testperson overser fuldstændig muligheden for at indstille valgte butikker i indstillingerne og må til sidst hjælpes derind. Desuden overser en af testpersonerne knapperne til at slå alle butikker fra/til og sidder derfor og slår alle butikkerne fra enkeltvis. Flere testpersoner klager desuden over at navnene på butikkerne står så langt til venstre at det er svært at læse. En af testpersonerne mente at formuleringen på \gaas{forbliv logget ind}, \gaas{alle til} og \gaas{alle fra} knapperne var misvisende og ikke gav mening. Dette gjorde at testpersonen byttede om på knapperne \gaas{alle til} og \gaas{alle fra}.\newline
\newline
\textbf{Generelt:}\newline
Alle testpersoner mente at man burde blive automatisk logget ind efter at have oprettet en ny bruger. Den ene af de kvindelige testpersoner mente det var nemmest at bruge smartphonen til at skrive en indkøbsliste, mens den anden hellere ville bruge computeren. Begge testpersoner synes designet er en smule kedeligt, trods det er konsistent, og den ene ønskede at designet blev mere computervenligt. Dog mente de designet var pænt på en smartphone.

\subsubsection{Opsummering}
Vurderingskriterierne for vurderinger af brugervenlighedsfejl kan ses på tabel \ref{tab:Brugervejledningsfejlvurdering}.

\begin{table}[H]
\centering
    \begin{tabular}{|l|c|c|c|}
	\hline
	~                           & Forsinkelse 	& Irritation	& Forventet vs. faktisk resultat~ 	\\ \hline
	Kosmetisk       			& < 1 minut   			  		& Lav						 & Lille forskel \\ \hline
	Seriøs       				& Flere minutter 				& Medium					 & Signifikant forskel \\ \hline	
	Kritisk       			    & Total (bruger stopper)  		& Stærk						 & Stor forskel \\ \hline
	
	\end{tabular}
	
\caption{Tabel som viser hvordan burgervenlighedsfejl bliver vurderet. \citep{debslide}}
\label{tab:Brugervejledningsfejlvurdering}

\end{table}

Observationerne er kort opsummeret i tabel \ref{tab:opsummeringaffejlfrausabilitylab}.

\begin{table}[H]
\centering
    \begin{tabular}{|l|c|}
	\hline
	Fejl 																& Type  \\ \hline
	
	Glemte indkøbslistens navn      									& Kosmetisk						   \\ \hline
	Ikke muligt at se hele teksten i et tilbud							& Kosmetisk						   \\ \hline	
	Krydse vare af i stedet for at åbne indstillinger					& Kosmetisk						   \\ \hline
	Ingen feedback når tilbud på vare blev valgt						& Kosmetisk						   \\ \hline
	Varer flyttede ulogisk rundt på listen								& Seriøst						   \\ \hline
	Brugeren optræder ikke selv på listen over tilknyttede brugere		& Kosmetisk						   \\ \hline
	Både muligt at dele indkøbsliste via e-mail og brugernavn			& Kosmetisk						   \\ \hline
	Dårlig sortering af tilbud i pop op									& Kosmetisk						   \\ \hline
	Svært at kende forskel på menu og markering af vare					& Kosmetisk						   \\ \hline
	Systemet kan ikke håndtere at to personer ændrer listen samtidig	& Kritisk						   \\ \hline
	Systemet kan virke langsomt											& Seriøst						   \\ \hline
	Der kan blive foreslået tilbud som allerede er på ens indkøbsliste	& Kosmetisk						   \\ \hline
	Funktionen \gaas{Slå alle butikker fra/til} overses					& Kosmetisk						   \\ \hline
	Man bliver ikke automatisk logget ind efter oprettelse af profil	& Kosmetisk						   \\ \hline	
	Kan ikke finde indstillinger for butikker	& Kritisk						   \\ \hline	
	
	
	\end{tabular}
	
\caption{Opsummering af fejl fundet i usability laboratorium.}
\label{tab:opsummeringaffejlfrausabilitylab}

\end{table}

Resultaterne fra tabel \ref{tab:opsummeringaffejlfrausabilitylab} kan ses talt sammen i tabel \ref{tab:antalaffejlfejlfrausabilitylab}.
\begin{table}[H]
\centering
\begin{tabular}{|c|c|}
\hline
Type fejl & Antal\\ \hline
Kritisk   & 2 \\ \hline
Seriøst   & 2 \\ \hline
Kosmetisk & 11 \\ \hline
\end{tabular}
\caption{Antal af fejl fundet i usability laboratorium.}
\label{tab:antalaffejlfejlfrausabilitylab}
\end{table}

\subsection{Konsekvenser af fjerde test}
Ud af 22 forslag til forbedringer er 15 blevet implementeret, fire er blevet fravalgt fordi de ikke blev fundet muligt eller relevante, og tre blev fravalgt grundet mangel på tid og ressourcer.\newline
Det der er blevet implementeret er:
\begin{itemize}
\item Knappen \gaas{Opdater} under delt liste er blevet omdøbt til \gaas{Del liste}. 
\item Felterne til mail og brugernavn blevet slået sammen og håndterer nu begge inputs i et felt.
\item systemet logger nu brugeren automatisk ind, efter man har lavet en ny bruger.
\item \gaas{Forbliv logget ind} ændret til man ikke logger af automatisk.
\item Inde i \gaas{Tilvalg af butikker} er til og fra knapperne omdøbt til \gaas{Tilvælg alle butikker} og \gaas{Fravalg alle butikker} og knapperne er gjort blå, så de er mere synlige.
\item Der er nu lavet et pop op-vindue der beder om et navn, når man vil oprette en ny indkøbsliste.
\item Pop op-vinduet med indstillinger inde i indkøbslisten lukker nu sig selv efter et tilbud er blevet tilføjet.
\item Der er nu under tilbuddene i indkøbslisten blevet tilføjet flere detaljer, herunder mængdepris.
\item Hvis titlen er for lang på tilbuddene, bliver det opdelt i flere linjer.
\item Der er blevet byttet om på placeringen af indstillingsknappen og knappen der markerer varer som \gaas{købt} på indkøbslisten.
\item Markøren flytter sig nu automatisk til tekstfeltet, hvor man kan indskrive en vare når siden genloades.
\item For at gøre sorteringsfunktionen mere synlig inde under indkøbslisterne, er den nu inde i en grå boks på hjemmesiden, hvilket skaber mere kontrast.
\item Det oprindelige søgeord bliver nu bibeholdt til senere søgninger af tilbud.
\item Pop op-vinduet med indstillinger ude i oversigten over en brugers indkøbslister opdateres efter en liste er blevet delt.
\item Brugeren kan nu bedre se at systemet har registeret en handling, repræsenteret ved prikker der løber rundt i en cirkel.
\end{itemize}
Det der er fravalgt på grund af relevans:
\begin{itemize}
\item At kunne filtrere funktionen foreslåede tilbud mere så der ikke kommer irrelevante tilbud.
\item At systemet selv skulle slette markerede varer efter et døgn.
\item At systemet selv skulle sammenlægge ens tilbud, hvis de har samme butik og pris.
\end{itemize}
Det der er fravalgt på grund af mangel på tid og ressourcer:
\begin{itemize}
\item Indstillingerne af tilvalgte butikker skal være listespecifik. 
\item At kunne sortere varerne efter varekategori og hvilken rækkefølge varerne kommer i butikken.
\item Gøre så det er muligt at vælge, om man vil have mange detaljer på varerne i indkøbslisten eller have mulighed for at se flere på skærmen.
\end{itemize}

\subsection{Afrunding fjerde test}
De vigtigste problemer der blev observeret i fjerde test var:
\begin{itemize}
\item Der var flere problemer under indkøbslisten som blev vurderet som problematiske.
\item Flere knapper blev overset da de ikke var tydelige nok, og flere af formuleringerne på knapperne blev fundet misvisende.
\end{itemize}
De vigtigste problemer der blev rettet i fjerde test var:
\begin{itemize}
\item Flere problemer omkring indkøbslisten blev rettet, så det passede bedre med den metode testpersonerne brugte systemet på.
\item De fleste af knapperne i systemet er nu blevet blå eller fremhævet på anden måde.
\end{itemize}
Disse blev vurderet som vigtigste på baggrund af tabel \ref{tab:opsummeringaffejlfrausabilitylab}