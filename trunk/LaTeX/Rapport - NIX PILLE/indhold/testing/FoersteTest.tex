\section{Første test - Prototype}
Inden løsningens design var lavet, blev der lavet en papirprototype, hvilket skabte et udgangspunkt for designet af løsningen.

\figur{0.5}{Papirprototype.png}{Papirprototype af forsiden.}{fig:papirprototypeforside}

\subsection{Testpersonerne}
Gruppens medlemmer

\subsection{Systemets tilstand}
Prototypen blev lavet ved at brainstorme over, hvad der skulle være i løsningen, og derefter blev det diskuteret i projektgruppen, hvordan de forskellige sider burde se ud. Dette blev til at starte med skitseret på en tavle. Da gruppen var blevet enige om et udgangspunkt, blev tavleskitserne tegnet ind på et stykke papir, med en størrelse der passede hvis problemløsningen skulle vises på en smartphone. Da alle indledende ideer var tegnet på papir, blev prototypen gennemgået. Ved at lave en papirprototype blev der skabt et godt grundlag for, hvordan designet skulle se ud og hvilke funktioner der skulle arbejdes på for at klare opgaven til problemløsningen.
Det første der blev lavet, var hovedmenuen for problemløsningen, hvilket kan ses på figur \ref{fig:papirprototypeforside}. Som det kan ses, ligner det færdige design \ref{fig:GUIFor} for hovedmenuen meget det i prototypen. Måden man navigerer i problemløsningen har nogle forandringer i forhold til prototypen. I problemløsningen er der en menu øverst på siden, uanset hvor man er i problemløsningen, denne menu indeholder link til forsiden, indkøbslister og log ind/log ud (afhængig af om man er logget ind eller ej). Heraf ses det, at har været udvikling siden prototypen blev lavet. Derudover er afstandene mellem billederne og kanterne lavet mindre i forhold til prototypen, for at udnytte pladsen bedst muligt.
På prototypens forside var der fire billedknapper, hvor en endnu ikke var helt gennemtænkt. Der var nogle ideer, om at den skulle indeholde noget med tilbud, enten ugens bedste tilbud, eller tilbud baseret på indkøbsvaner. Ved problemløsningen har alle fire billedknapper en funktion. Det er tænkt, at billedknapperne skulle gå til de sider, der indeholder de forskellige hovedfunktioner: Søg, indkøbslister, tilbudsfunktion og indstillinger. Funktionen \gaas{søg} var tænkt som en side, hvor man ville kunne søge efter tilbud i databasen over tilbud. Indkøbslistefunktionen var tænkt som en side, hvor man kunne oprette indkøbslister og tilføje varer samt tilbud til listerne. Selvfølgelig skulle det også være muligt at fjerne varerne/tilbudene.
Tilbudsfunktionen skulle som tidligere nævnt være en form for afgrænsede tilbud. Indstillingssiden skulle være en simpel side, hvor brugerne kunne indstille forskellige indstillinger relaterede til deres bruger, heriblandt adgangskode, e-mail og til- og fravælge butikker fra søgeresultater. I problemløsningen blev tilbudsfunktionen, til en funktion som ud fra tidligere indkøb, foreslår tilbud til brugeren. 

\subsection{Kvalitetsmål i fokus}
Ved prototypen blev det undersøgt om løsningen var brugbar og forståelig. Dette blev gjort ved at se på hvor hurtig og nemt man kunne navigere i prototypen.

\subsection{Testmetoden}
Gruppens medlemmer prøvede at bruge prototypen, som om det var et reelt produkt. Ved dette kan ideen bag prototypen blive afprøvet og se om de holder i praksis, uden at bruge for meget tid på det. 

\subsection{Konsekvens af første test}
Log ind i prototypen var lavet som en pop op, mens den i problemløsningens har en separat side, hvor der er nogle funktioner såsom: Muligheden for at oprette en bruger til problemløsningen og hvis brugeren har glemt en adgangskode eller brugernavn, kan man få tilsendt en e-mail således brugeren har mulighed for at generhverve adgang til deres konto.

Det var bestemt at funktionen \gaas{søg} skulle være med lige fra starten, men ved prototypen kunne man kun tilgå den via forsiden. Ved problemløsningen kan man derimod også finde tilbud på varer på ens indkøbsliste, ved at gå ind på indkøbslisten, og så vælge den vare man vil finde tilbud på. Man vil så blive præsenteret for et pop op-vindue med indstillinger, hvor en af disse er \gaas{søg}, således behøver man ikke at gå ind på selve siden gaas{søg}, for at finde et tilbud.

\subsection{Afrunding første test}
Prototypen er blevet brugt i gruppen til at skabe en grund-idé til designet på løsningen. Prototypen blev ikke fremvist til eksterne personer, men blev som den første test prøvet på personerne internt i gruppen. Det gav også et godt indblik i hvad der var krævet at systemet.