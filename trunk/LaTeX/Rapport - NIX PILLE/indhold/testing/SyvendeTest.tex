\section{Syvende test}
Her beskrives alt omkring syvende test.

\subsection{Testpersonen}
Testpersonen er en mand på 20 år. Testpersonen har en begrænset økonomi, og er derfor interesseret i at handle billigt. Teknologisk erfaring: Middel. Ejer en smartphone. 

\subsection{Systemets tilstand}
På tidspunktet for syvende test var systemt mere eller mindre færdigudviklet, men manglede stadig at blive testet i en rigtig indkøbssituation. Den syvende test blev netop lavet for at afsløre fejl, som eventuelt kunne opstå når man brugte systemet i et supermarked, hvor forhold for bl.a. data-forbindelsen ikke er optimale.

\subsection{Kvalitetsmål i fokus}
I denne test blev specielt fokuseret på kvalitetsmålene \gaas{brugbart}, \gaas{effektivt} og \gaas{forståeligt}. \gaas{Brugbart} var vigtigt at få testet for at se om programmet kunne bruges i en rigtig handelssituation. Der blev også testet efter om systemet var forståeligt ved at testpersonen ikke kendte systemet, og kun fik en kort introduktion til systemet. Det sidste der blev testet, var om systemet var effektivt at bruge i en rigtig handelssituation.

\subsection{Testmetoden}
Testpersonen fik en kort intro til programmet og en forudlavet indkøbsliste. Derefter gik testpersonen ud og handlede, mens et gruppemedlem observerede hvad testpersonen gjorde og hvad der skete. Derefter blev der holdt et kort interview omkring hvad testpersonen synes om at bruge programmet.

\subsection{Observationer fra syvende test}
Der skete to fejlvisninger i løbet af testen. Ved den ene blev typografiarket ikke loadet ordentligt hvilket gjorde at alt på siden blev vist som tekst. Den anden fejl var at hver gang mobilen blev vendt i landskabstilstand og så vendt til portrættilstand igen kunne siden ikke finde tilbage i de rigtige proportioner igen. Dette er forsøgt løst, men problemet forekommer stadig somme tider på iOS enheder. Bortset fra dette, mente testpersonen at programmet var godt at bruge, når han handlede og at ventetiden på at siden genindlæses efter en vare er blevet markeret, ikke er et problem da han gik et stykke mellem de varer han skulle have, og derfor ikke kom til at vente. Han fandt det dog problematisk hvis siden også skulle genindlæses, hver gang han kom til at vende smartphonen til landskabstilstand. Testpersonen sagde det ville være rart hvis varerne var kategoriseret efter varekategorier, men det ikke var en nødvendighed for programmets brugbarhed.

\subsection{Konsekvenser af syvende test}
Problemstillingen med at programmet ikke kunne finde tilbage til de rigtige proportioner efter at være vendt på siden er forsøgt løst bedst muligt med den tid og ressourcer, der var til rådighed. Den anden visningsfejl skyldes sandsynligvis en kort afbrydelse af data-overførslen af typografiarket og der er derfor ikke blevet rettet noget. Der er desuden blevet lavet så det nu er muligt selv at vælge en varekategori til en given vare og sortere listen efter denne.

\subsection{Afrunding syvende test}
De vigtigste problemstillinger der blev observeret i syvende test var:
\begin{itemize}
\item Ventetiden der blev observeret i test fire var ikke noget problem i en rigtig handelssituation.
\item Systemet virkede efter hensigten med udtagelse af de to visningsfejl.
\end{itemize}
De vigtigste problemstillinger der blev rettet i syvende test var:
\begin{itemize}
\item Den ene visningsfejl blev rettet bedst muligt.
\item Det er nu muligt at vælge en varekategori til varerne og sortere efter denne.
\end{itemize}