\section{Contextual Analysis}
\label{contextual}
%Til retter - læs hvad der står under contextual analysis i deres rapport, og tjek om det ikke er det rigtige
Because we are using ANTLR, we get an parse tree, with a basic visitor, to work with. By expanding the basic visitor, three modified visitors have been made. The first visitor is for checking if the scope rules are in order, the errors that are found is put into an error list which will be shown to the programmer to indicate what is wrong with the code. The second visitor is for type checking, it will visit the abstract syntax tree to see if all the types are used together correctly, again errors will be put into a list and shown to the programmer. The last visitor will generate code from our language into a language that can be used by Arduino.