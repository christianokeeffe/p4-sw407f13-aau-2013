\subsection{Lexical Analyzer}
A lexical analyzer or "lexer" reads the input file, and returns a series of tokens based on the input \citep{CraftingACompiler}. More specifically it is the scanner in the lexical analyzer which does this. These tokens are matched by rules, usually described by regular expressions. An example of such grammar rules can be seen on table \ref{tab:tokenspecification}. Formally a token consists of two parts: The token type, and the token value \citep{CraftingACompiler}. As an example the IDEND token seen on table \ref{tab:tokensexample} has the token type IDEND and the value 'c'. 

\begin{table}[H]
\begin{tabular}{|l|l|}
\hline
    	Terminal  	& Regular expression 	\\ \hline
    	dcl       	& "$[a-z]$"      		\\ 
    	assign    	& "="      				\\ 
    	digit     	& "$[0-9]^+$"   		\\ 
    	endassign 	& ";"      				\\
    	blank 		& " "$~^+$				\\
    \hline
\end{tabular}
\caption{Sample token specification.}
\label{tab:tokenspecification}
\end{table}
The specification of tokens on \ref{tab:tokenspecification}, would be used by the scanner to determine how tokens look, and thereby which text-elements are tokens. 

\begin{code}{simplecode}{Simple example of code.}
\begin{lstlisting}
c = 42;
\end{lstlisting}
\end{code}

As an example the code seen on listing \ref{lst:simplecode} might be read as the tokens seen on table \ref{tab:tokensexample}.
\begin{table}[H]
\begin{tabular}{|l|l|}
\hline
    \textbf{Token}	&	\textbf{Lexeme} \\ \hline
    IDEND     		&	c      			\\ 
    ASSIGN    		&	=      			\\ 
    DIGIT     		&	42    			\\ 
    SEMICOLON 		&	;      			\\ \hline
\end{tabular}
\caption{Example of tokens.}
\label{tab:tokensexample}
\end{table}
The scanner produces a stream of tokens, which is returned to the parser. The parser checks if the tokens conform to the language-specification \citep{CraftingACompiler}.