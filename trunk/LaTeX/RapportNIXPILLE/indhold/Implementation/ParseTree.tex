\subsection{Parse Tree}
\label{sec:parsetree}
A parse tree is nearly the same as an abstract syntax tree, but here all internal nodes are labeled with a non-terminal symbol and all leaves are labeled with a terminal symbol. A sub-tree describes one instance of an abstraction of a sentence.

To help understand the difference on an abstract syntax tree as seen on figure \ref{fig:abstract-syntax-tree} and a parse tree as seen on figure \ref{fig:ParseTreeEKS}, two trees are made, one of each, from the same code for a variable declaration seen on listing \ref{lst:VariableDeclarationEKS} in the projects language.

\begin{code}{VariableDeclarationEKS}{A simple variable declaration in the project language.}
	\begin{lstlisting}
		int x <-- 3+2;
	\end{lstlisting}
\end{code}

\begin{figure}[H]
\Tree[.program [.<-~- [.x
]
                    [.+ [.3
]
                        [.2
                    ]]]]
\caption{An abstract syntax tree.}
\label{fig:abstract-syntax-tree}
\end{figure}
\figur{0.7}{parsetree.png}{A parse tree made from the code on listing \ref{lst:VariableDeclarationEKS} from the BNF of SPLAD.}{fig:ParseTreeEKS}