\subsection{Visitor Pattern}
\label{sec:VisitorPattern}
The visitor pattern is ideal for traversing an abstract syntax tree or a parse tree during the semantic analysis and code generation, to help manage the large number of phase- and node interactions. A phase is started by visiting the first node in the AST, to reach every node there must be a call to \emph{visit} in the preceding node.

Single dispatch is used by most object-oriented languages to determine which \emph{visit} method that must be used. The method is based on the type of the parameter. There are a few problems with single dispatch, such as the fact that it finds a match based on the declared type of its parameters when it is called. This is where the visitor pattern comes in handy with a form of double dispatch. The idea is to make use of the abstract class such as the one seen on listing \ref{lst:GenericVisitor} that all nodes implement.

\begin{code}{GenericVisitor}{A generic Visitor.}
\begin{lstlisting}
class Visitor
	procedure visit(AbstractNode n)
		n.ACCEPT(this)
	end
end
\end{lstlisting}
\end{code}

If the visit method is called without the accept method, it will try to visit the abstract node. For example if a phase contained a method visit(\textit{IfNode n}) like on listing \ref{lst:ConcreteVisitor}, it will not invoke the actual \textit{IfNode}, this is because the supplied parameter is based on the declared type (\textit{AbstractNode}).

\begin{code}{ConcreteVisitor}{A concrete Visitor.}
\begin{lstlisting}
class TypeChecking extends Visitor
	procedure VISIT(IfNode i)
	end
end
\end{lstlisting}
\end{code}

Therefore the specific node accepts a visitor as seen on listing \ref{lst:ConcreteNode} and thus determines the type of the node which allows it to be visited because it now knows the actual \textit{IfNode} and the code it contains can be executed.

\begin{code}{ConcreteNode}{A concrete Node.}
\begin{lstlisting}
class IfNode extends AbstractNode
	procedure ACCEPT(Visitor v)
		v.VISIT(this)
	end
	...
end
\end{lstlisting}
\end{code}