\subsection{Language Processing Strategy}
\label{sec:LanguageProcessingStrategy}
This section will describe the language processing strategy in this project. To introduce what is called tombstone diagrams, a small example of a tombstone diagram for the Java-language is presented in figure \ref{fig:tombjava}. This figure shows that the Java-compiler takes some java-code and turns it into byte code on machine M. The Java-virtual machine (JVM) can then execute the byte code. This makes the bytecode platform-independent, because if there is a JVM-implementation for an architecture, Java byte code can be compiled on another architecture and be executed on the first architecture. The tombstone to the left should be read as: Compiler written in machine-code $M$, takes a Java-source program and turns it into byte-code while running on architecture $M$.

\figur{0.5}{javacompiler.png}{Tombstone diagram for the Java-compiler}{fig:tombjava}

The left part of figure \ref{fig:tombjava}, represents the JVM, which can actually execute the byte-code (BC) on architecture $M$.

Figure \ref{fig:tombspladcompilercompilation} illustrates the compilation of the SPLAD-compiler. The SPLAD-compiler is written in Java and converts the SPLAD source-program to Arduino-code. As the SPLAD-compiler is written in Java, it is therefore compiled into Java byte-code using the Java-compiler. It can be seen in figure \ref{fig:tombspladcompilercompilation}, that everything is bootstrapped together. The SPLAD-compiler is bootstrapped to the Java-compiler which returns a SPLAD-to-Arduino compiler written in Java byte-code. The Java-compiler is also written in byte-code, and runs on the JVM, which is written in machine code for architecture $M$, and runs on architecture $M$.
\figur{0.5}{spladcompilercompilationprocess.png}{Tombstone diagram for the SPLAD-compilers compilation process}{fig:tombspladcompilercompilation}

Now we have the SPLAD to Arduino compiler. Figure \ref{fig:spladcompiler} illustrates how the compiler works. The SPLAD-compiler takes a SPLAD-source program, and compiles it to Arduino-code, while running on the JVM. Therefore the user must use the Arduino compiler to compile the generated Arduino program, to a Arduino machine code, and upload it to the Arduino platform. 
\figur{0.7}{spladcompiler.png}{Tombstone diagram for the SPLAD-compiler compilation process (ABC is an acronym for Arduino Byte Code)}{fig:spladcompiler}