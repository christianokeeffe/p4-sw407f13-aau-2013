\section{Problem Statement}
\label{sec:problemstatement}
In this section a problem statement will be presented, which will be used as a basis for this project. In this project it has been decided to examine, how a drinks machine could be programmed using Arduino as a platform for the processing. The programming language usually used for Arduino is based on C and C++, which is not aimed at programming a drinks machines, see section \ref{sec:hardwarearduino}. It could be useful to have a niche programming language aimed directly at programming drinks machines on an Arduino platform. This will be the goal of this project.

The programming language in this project is aimed at the hobbyist programmer who wants to program his own drinks machine. Because of this, the programs written in this language must be simple to understand and maintain. This however sacrifices some write-ability of the programs, because of constraints imposed to ensure programs are easier to understand. These trade-offs will be further discussed in section \ref{sec:grammarchoice}. A hobbyist programmer is defined as a programmer who knows the basic structure of programming, but does not have an education in programming or work with software development.

This leads to the following problem statement:
\begin{itemize}
	\item \textbf{How can a programming language be developed, which makes it suitable for the hobbyist programmer to program drinks machines based on Arduino platforms?}
\end{itemize}
The purpose of this problem statement is to guide the development of the programming language for this project, so when it reaches its final state, it is as simple as possible for hobbyist programmers to program when using the language. 

\subsection{Sub Statements}

On the basis of the problem statement, a number of sub-statements arises:
\begin{itemize}
	\item \textbf{How can a programming language be specified, which makes it suitable for novice programmers?} 
	Because the language of this project is aimed at hobbyist programmers, the programming language should be specified in a way which is suited for the target group.
	\item \textbf{How can a compiler be developed, which recognizes the language, and translates the source program into suitable code for Arduino?}
	Of course it is not enough to have a simple-to-understand language, if no compiler exists for that language. The language would then be rendered useless for the purpose. This is the reason why a compiler must be developed, either by compiling the program code directly to Arduino machine code, or by first compiling the program code to an intermediate language, and then use the Arduino compiler to compile that code further. 
\end{itemize}