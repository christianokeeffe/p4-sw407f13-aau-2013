\subsection{Theory of Scope Rules}
\label{sec:scoperules}
The scope of a variable is the block of the program in which it is accessible. A variable is local to a block, if it is declared in that block. A variable is non-local to a block if it is not declared in that block, but is still visible in that block (ex. global variables) \citep{sebesta}.
The scope rules a the language determines how a variable name is associated with a value in a particular occurrence. When working with a functional language, the program needs to know how a name is associated with an expression, when a variable is declared in a program unit or block. When a variable is declared in a block it is local to that part. The non-local variables are visible within the block if they are not declared there. Lastly global variables are a special category of non-local variables.

\subsubsection{Static Scope}
Static scoping was introduced in ALGOL 60 \citep{sebesta}. This type of scoping binds names to non-local variables. There are two categories of statically-scoped languages. The first one is where sub-programs can be nested and where sub-programs creates nested static scopes. The other is static scoping where sub-programs cannot be nested, but sub-programs do create static scopes. In the latter nested scopes are only created by blocks or class definitions.

Blocks are used to define new static scopes in many languages. The idea is that it allows a section of code to have its own local variables.

An example on the use of blocks can be seen on listing \ref{lst:BlockCode}. Before the block, the variable $x$ is declared and set to the value 5. In the block, $x$ is set to 10 and an extra variable $y$ is declared and is set to the value 15. $y$ is only visible inside the block, thereby it cannot be called outside of the block. After the block, $x$ still has the value 10 which was given inside the block.

\begin{code}{BlockCode}{A simple code example of the use of blocks.}
\begin{lstlisting}
int x = 5
{
    int y = 15
    x = 10
} 
\end{lstlisting}
\end{code}

\subsubsection{Dynamic Scope}
With dynamic scope, the scope is determined at run time, because it is based on the calling sequence of sub-programs and not their spatial relationship to one another.

\subsubsection{Declaration Order}
The main thing about declaration order is how the data declarations are made. They can be before functions, such as C89, before they are used, such as C\#, or they can be anywhere in the code, like C99, C++, Java and JavaScript \citep{sebesta}.

\subsubsection{Global Scope}
Some languages allow a program structure to be a sequence of function definitions, such as C, C++, PHP. Definitions outside functions in a file create global variables, which makes it visible to those functions \citep{sebesta}.

\subsubsection{Dynamic vs. Static Scope}
To better understand the difference between dynamic and static scope a code example can be seen in listing \ref{lst:StaticDynamicScoping}.
\begin{code}{StaticDynamicScoping}{A simple code example showing the difference between static and dynamic scoping \citep{StaticvsDynamic}.}
\begin{lstlisting}
int b = 5;

int foo()
{
	int a = b + 5;
	return a;
}

int bar()
{
	int b = 2;
	return foo();
}

int main()
{
	foo();
	bar();
	return 0;
}
\end{lstlisting}
\end{code}
Listing \ref{lst:StaticDynamicScoping} will in both cases return 10 in the foo() function, but in bar() the result will differ. With static scoping the bar() function will return 10 because at compile time $b$ was set to 5, while with dynamic scoping it will return 7 because at run time $b$ is set to 2.

\subsection{Scoping in this Project}
In SPLAD, static scoping is used. This means that scopes are computed at compile time, based on the program code. The main reason for this, is that programs for the Arduino platform are mainly written in C, which also uses static scoping \citep{sebesta}. This makes the compilation from SPLAD to C simpler for the compiler. Static scoping means that a hierarchy of scopes are maintained during compilation. To determine the name of used variables, the compiler must first check if the variable is in the current scope. If it is, the value of the variable is found, and the compiler can proceed, else it must recursively search the scope hierarchy for the variable. When done, if the variable is still not found, the compiler returns an error, because an undeclared variable is used \citep{sebesta}.