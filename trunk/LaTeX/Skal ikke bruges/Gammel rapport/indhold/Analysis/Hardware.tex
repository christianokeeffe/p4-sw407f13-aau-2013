\section{Hardware}
This section will be about the hardware components used in this project, describing them and the reasons they are used in this project. In the description it will be looked at the more basic technical specification that will be relevant for this project.

\subsection{Hardware platform}
\label{sec:hardwarearduino}
Arduino UNO is a powerful micro controller board which provides the user with ways to communicate with other components such as LCDs, diodes, sensors and other electronic bricks(building blocks) which is a desirable feature in this project. Arduino uses the ATmega328 chip which provides more memory than its predecessors \citep{ArduinoUno}.

There exists alternatives to Arduino product which could be considered for this project. Teensy is similar to Arduino in many ways but Teensy differences in the actual size of the product. Teensy is also cheaper than Arduino but does require soldering for simple set-ups where Arduino comes with a board and pin-ports which means that it requires little pre work before using it \citep{Teensy}.
Seeeduino is a near replica of Arduino, an example could be Seeeduino Stalker which offers features as SD-card slot, flat-coin battery holder and X-bee module-headers. X-bee is a module for radio communication between one or more of these modules. Seeeduino is compatible with the same components as 
Arduino that makes it suited for acting as a replacement \citep{Seedui}.
Netduino is a faster version of Arduino but it comes at higher cost. Netduino also require the .net framework so it will only work together with windows operating systems. Netduino uses a micro-USB instead of regular USB as Arduino does \citep{Netdui}.

Arduino is more accessible because the Aalborg university already has some in stock that could be used where the other alternatives have to be bought first. Arduino, Teensy and Seeeduino are all compatible with the other equipments that will be used it this project. Netduino is limited to the .net framework were Arduino and the other alternatives are more flexible and therefore more ideal because they work with more platforms.
So it comes down to that Arduino have all the necessary features and is the most convenient one to obtain. The project group has found this platform suited for this project based on these reflections. 

Arduino UNO board has 14 digital input/output pins where six of them can emulate an analogy output through PWM (Pulse-Width modulation) which are available on the Arduino board. The Arduino board also provides the user with six analogy inputs which enables the reading of a alternating current and provides the user with the currents voltages. These pins can be used to control or perform readings on other components and in that way provides interaction with the environment around the board.
The Arduino board is also mounted with an USB-port and jack socket. The board can be hooked up with a USB cable or an AC-to-DC (Alternating Current to Direct Current) adapter through the jack socket to power the unit. Arduino UNO operates at 5v (volts) but the recommend range is 7-12v because lower current than 7v may cause instability if the unit needs to provide a lot of power to the attached electronic brick. The USB is also used to program the unit with the desired program through a computer \citep{ArduinoUno}.

Programs for Arduino are commonly made in Arduino's own language that are based on C and C++. The produces of the Arduino platform provides a development environment (Arduino IDE) that makes it possible to write and then simply upload the code to the connected Arduino platform. This process also provides a library with functions to communicate with the platform and compatible components \citep{ArduinoLanguage}.
Arduino is suited for this project because it makes it possible to demonstrate the language and illustrate that the translation works.

\subsection{RFID}
To administrate the users collection of purchased drinks the plan is to store the number and kind of drinks on an RFID tag that the customer then can use at the drink machine to get their drinks served.

RFID (Radio Frequency IDentification) is used to identify individual objects using radio waves.
The communication between the reader and the RFID tag can go both ways, and it is possible to both read and write to most tag types. 
The objects that are able to be read differs a lot. It can be clothes, food, documents, pets, packaging and a lot of other kinds. 
All tags contains a unique ID that can in no way be changed once made. This ID is used to identify an individual tag.
Tags can be either passive or active. Passive tags do not do anything until a signal from a reader transfers energy to the tag once activated it sends a signal back in return. Active tags have a power source and therefore is able to send a signal on their own, making the read-distance greater.
The tags can also be either \textit{read only tag} or \textit{read/write tag}. A \textit{read only tag} only sends its ID back when it connects with a reader, while a \textit{read/write tag} have a memory for storing additional information it then sends with the ID \citep{RFID}.

\subsection{Other components}
The demonstration situation will require something to illustrate more advanced parts of theoretical machine. The plan is to use LEDs (light emitting diode) to illustrate the different function of the machine, when they are active or inactive. The LED is made of a semiconductor which produces a light when a current runs through the unit.

LEDs are normally easy to use by simply running a current the correct way through the LED.
The reason why LEDs are being using instead of making the machine is that there are neither time for it nor is it the main focus of this project.

It would also be good to be able to print a form of text to the customer. To do this there will be used an LCD 16-pin (Liquid Crystal Display). Arduino's Liquid Crystal library provides the functions to write to LCD so no low level code is needed to communicate with the LCD \citep{ArduinoLCD}.

As input switches/buttons will be used that will allow interaction with the program at runtime. The switches will illustrate a more advanced control unit but in the project switches will be sufficient.