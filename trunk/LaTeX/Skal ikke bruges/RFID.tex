\section{RFID}

RFID (Radio Frequency IDentification) are used to identify individual objects using radio waves.
The communication between the reader and the RFID tag can go both ways, and you are able to read and write to most tag types. 
The objects able to be read differ a lot. It can be clothes, food, documents, pets, packaging and a lot of other kinds. 
All tags contains a unique ID that can in no way be changed from when they were made. This ID is used to identify an individual tag.
Tags can be either passive to active tags. Passive tags do not do anything until their antenna catches a signal from a reader. This signal transfers enough energy to the tag for it to send a signal in return. active tags have a power source and therefore is able to send a signal on their own, making the read-distance greater.
The tags can also be either \textit{read only tag} or \textit{read/write tag}. A \textit{read only tag} only sends its ID back when it connects with a reader, while a \textit{read/write tag} have a memory for storing additional information it then sends with the ID.

Kilde http://www.rfid-specialisten.dk/rfid.asp brugt den 15-02-13