\subsection{Transition System}
A transition system is used to describe how the constructions of a programming language behaves given a state and a cause. The cause is represented as a transition and given a state it will result in a movement in the system.

A transition system is defined as a tuple with three sets, hence a triple. These three sets are: A set of configurations ($\Gamma$), a set of transition relations $(\rightarrow)$ and a set of terminal configurations $(\Tau)$.
A configuration is a description of a program and the state of it. Transition relations describes how the program moves from state to state. The terminal configuration describes the states where the program ends, meaning that when one of these are reached the program terminates \citep{HHTree}.

We uses the transition system to describe the semantics of our programming language. It gives us a tool to formally describe precisely how the language should behave. This results in a more precise way of implementing the rules we want, compared to if it was described in a informal way. There are two ways of using a transition system, big-step and small-step, see section \fxfatal{ref til big-step vs small step }. In this section we also describe which of the two we have chosen to use and the reason for this choice.