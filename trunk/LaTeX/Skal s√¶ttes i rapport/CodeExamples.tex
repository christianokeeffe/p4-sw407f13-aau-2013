\section{SPLAD Code Examples}
This section will contain small code examples written in the SPLAD language. Each example will be described line by line, and in the end put together to a working SPLAD program.
When writing a SPLAD program, it is important to keep in mind that there are functions that must be implemented. These functions are "pour" and "RFIDFound". First we will see how an implementation of the "pour" function could look like. This can be seen on listing \ref{lst:expour}. On line 1 of the "pour" example, the function is declared. The function takes two parameters as input, a container and an amount. Amount is actually implemented as time because it represents how long a nozzle will have to be open for a given amount to be poured. So if "pour" is called with container x and amount 100 it would in principle open the valve on container x for 100 seconds. It is the function on line 3 on listing \ref{lst:expour} that opens the valve. Then on line 4, the Arduino is told to wait for a given number of milliseconds times 1000, which if "amount" was 100, corresponds to the Arduino waiting for 100 seconds. On line 5, the container valve is closed again, and on line 6 the function returns "nothing". The example is very close to the implementation used in the test program which can be seen in appendix \ref{chap:testprocapp}.

\begin{code}{expour}{Implementation of the pour function.}
\begin{lstlisting}
function pour return nothing using(container c, int Amount)
begin
	call digitalWrite(c, HIGH);
	call delay(Amount*1000);
	call digitalWrite(c, LOW);
	return nothing;
end
\end{lstlisting}
\end{code}

Now to give an example of the "RFIDFound" function, which is also required in a SPLAD program. An example of the RFIDFound function can be seen on listing \ref{lst:exrfidfound}. On line 1, the function is declared - it takes a drink id and an amount as input. In this simple example, the amount is not used. The RFIDFound functions is called when the RFID-reader reads a RFID-tag. In this example we assume two drinks, "Tom Collins", which has drink id 0 and "Tequila Sunrise" which has drink id 1. On line 3, it is checked if the provided drink id is 0, if it is, the pourDrink function is called with TomCollins as parameter. The pourDrink function would then pour the drink. If the drink id is not 0, the Tequila Sunrise drink is poured.   

\begin{code}{exrfidfound}{Implementation of the RFIDFound function.}
\begin{lstlisting}
function RFIDFound return nothing using(int DrinkID, int Amount)
begin
	if(DrinkID = 0)
	begin
		pourDrink(TomCollins);
	end
	else
	begin
		pourDrink(TequilaSunrise);
	end
	return nothing;
end
\end{lstlisting}
\end{code}

Now we have the required functions, but the two drinks in the above example has not been declared. The drink declarations can be seen on listing \ref{lst:exdrinkdec}. On line 1, the declaration the drink TomCollins begins, next, from line 3 to line 6 the various ingredients are added. The same principles are valid for the declaration of the TequilaSunrise. It is clear that the declaration of a drink is very much like a recipe for a drink. 

\begin{code}{exdrinkdec}{Declaring the drinks.}
\begin{lstlisting}
drink TomCollins is
begin
	add 3 of Gin;
	add 2 of Sugar;
	add 2 of Lime;
	add 5 of Soda;
end

drink TequilaSunrise is 
begin
	add 2 of Tequila;
	add 5 of OrangeJuice;
	add 1 of RedGrenadine;
end
\end{lstlisting}
\end{code}

Lastly we need to declare the ingredients for the drinks. The ingredient is implemented as a container which holds the specific ingredient. This is even simpler than declaring the drinks. The declaration of the ingredients can be seen on listing \ref{lst:exindec}. It is pretty straightforward to declare the various containers. The only requirement is that the containers are assigned to a pin on the Arduino board. These pins range from A0-A5 and 1-13 on an Arduino Uno platform.

\begin{code}{exindec}{Declaring the containers.}
\begin{lstlisting}
container Gin <-- A0;
container Sugar <-- A1;
container Lime <-- A2;
container Soda <-- A3;
container Tequila <-- A4;
container OrangeJuice <-- A5;
container RedGrenadine <-- 1;
\end{lstlisting}
\end{code}

Now that every part of the example program is defined, it can be put together, the whole example can be seen on listing \ref{lst:exfull}. The RFIDFound function is called when the RFID reader reads a RFID-tag. The tag contains the id 0 or 1, depending on the drink. If the tag contains 0, the drink Tom Collins is poured. 

\begin{code}{exfull}{The example program structured together.}
\begin{lstlisting}
container Gin <-- A0;
container Sugar <-- A1;
container Lime <-- A2;
container Soda <-- A3;
container Tequila <-- A4;
container OrangeJuice <-- A5;
container RedGrenadine <-- 1;

drink TomCollins is
begin
	add 3 of Gin;
	add 2 of Sugar;
	add 2 of Lime;
	add 5 of Soda;
end

drink TequilaSunrise is 
begin
	add 2 of Tequila;
	add 5 of OrangeJuice;
	add 1 of RedGrenadine;
end

function RFIDFound return nothing using(int DrinkID, int Amount)
begin
	if(DrinkID = 0)
	begin
		pourDrink(TomCollins);
	end
	else
	begin
		pourDrink(TequilaSunrise);
	end
	return nothing;
end

function pour return nothing using(container c, int Amount)
begin
	call digitalWrite(c, HIGH);
	call delay(Amount*1000);
	call digitalWrite(c, LOW);
	return nothing;
end
\end{lstlisting}
\end{code}