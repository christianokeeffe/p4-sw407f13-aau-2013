\section{Overview of the Compiler}
Figure \ref{fig:CraftingACompiler2} shows an abstract overview of each of the different phases in the compiler, what each phase requires as input, and what each step returns to the next phase.
\figur{0.8}{OverviewCompiler.PNG}{This is an abstract overview of how the compiler is structured. The figure is from \citep{OverviewCompiler2}.}{fig:OverviewCompiler} 

A compiler is a fundamental part of modern computing. Their job is to translate programming language into machine language. A compiler allows programs to make a virtual computer to ignore the machine-dependency details of machine language and therefor be portable across different computers  \citep{CraftingACompiler}.

A compiler consists of 3 different phases. The different phases roughly correspond to the different parts in a language specification which can be seen on figure \ref{fig:OverviewCompiler}. The syntax analysis correspond to the syntax, the contextual analysis to the contextual constraints and the code generation phase roughly corresponds to the semantics.

Given a simple compiler it will go through more than three phases. This can be seen on figure \ref{fig:OverviewCompiler2}. In the syntax analysis phase the compiler consists of a scanner and a parser. The scanner takes the source program and transforms it into a stream of tokens. The parser then uses the tokens to create an abstract syntax tree (AST). In the contextual analysis a symbol table is created from the abstract syntax tree. At the end, the semantic analysis decorate the AST, and translates this into the target language.

The different phases will be described more thorough later in the rapport. 

\figur{0.8}{OverviewCompiler2.PNG}{This is an more detailed overview of how the compiler is structured. The figure is from \citep{CraftingACompiler}.}{fig:OverviewCompiler2}