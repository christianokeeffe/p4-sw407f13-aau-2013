\section{Contextual Analysis}
because we are using ANTLR, we get a complete parse tree with a visitor to work on. With the visitor to work on there have been made three different, first visitor that is made, is for checking if the scopes rules are in order, the errors that are found is put into a error list which will be shown to the programmer to show that is going wrong with the code. Then the visitor for type checking is use on the\fxfatal{skal hvis være "a"} parse tree to see if the types are correct, again will the error be put into a list and shown to the programmer. As the last visitor the code generator will make our language into language that can be used by a Arduino.

For making a parse tree with ANTLR generate functions over a given code,This can be seen on the code snippet \fxfatal{kodestykke fra main hvor den laver parseren til os}.

It first need to go through a lexer, can be seen on line \fxfatal{linje}, where it can be seen that the lexer need the program code. Then finding the tokens the command that be seen on line \fxfatal{linje} is used, here it can be seen that the token generator need the lexer output for making the tokens. And for making the parser the tokens are needed, the command that be seen on line \fxfatal{linje}.

With the parse tree it is possible to use the scope checker, type checker and the code generator. The scope checker is call in line \fxfatal{linje}. After scope checking comes types checking with is called in line \fxfatal{linje}. When scope checking and types checking is complete and there is no errors it will start generating.