\subsection{Design Criteria in this Project}
To determine how a programming language should be syntactically described, the trade-offs of designing a programming language must be taken into account. The different characteristics of a programming language, that will be used to evaluate trade-offs can be seen on table \ref{tab:langCharacteristics}.

\begin{table}[H]
	\begin{tabular}{|l|l|}
		\hline
		Readability 		& How easy it is to understand and comprehend a computation			  \\ \hline
		Write-ability 		& How easy it is for the programmer to write a computation clearly,	  \\
		~ 					& correctly, concisely and quickly 									  \\ \hline
		Reliability 		& Assures a program behaves the way it is suppose to				  \\ \hline
		Orthogonality 		& A relatively small set of primitive constructs can be				  \\
		~ 					& combined legally in a relatively small number of ways				  \\ \hline	
		Uniformity 			& If some features are similar they should look and behave similar	  \\ \hline
		Maintainability 	& Errors can be found and corrected and new features can be added	  \\
		~ 					& easily 															  \\ \hline
		Generality 			& Avoid special cases in the availability or use of constructs and by \\ 
		~ 					& combining closely related constructs into a single more general one \\ \hline
		Extensibility 		& Provide some general mechanism for the programmer to 				  \\
		~ 					& add new constructs to a language 									  \\ \hline
		Standardability 	& Allow programs to be transported from one computer 				  \\
		~ 					& to another without significant change in language structure 		  \\ \hline
		Implementability	& Ensure a translator or interpreter can be written 				  \\ \hline
	\end{tabular}
	\caption{Brief explanation of language characteristics \citep{sebesta}.}
	\label{tab:langCharacteristics}
\end{table}

These characteristics are used to evaluate the the trade-offs of programming language. An overview of these can be seen on table \ref{tab:langTradeOffs}.

\begin{table}[H]
	\begin{tabular}{l|c|c|c|}
\textbf{Characteristic} & \rotatebox{90}{Readability} &\rotatebox{90}{Writability} & \rotatebox{90}{Reliability} \\ \hline
		Simplicity & $\bullet{•}$ & $\bullet{•}$ & $\bullet{•}$ \\ \hline
		Orthogonality & $\bullet{•}$ & $\bullet{•}$ & $\bullet{•}$ \\ \hline
		Data types & $\bullet{•}$ & $\bullet{•}$ & $\bullet{•}$ \\ \hline
		Syntax design & $\bullet{•}$ & $\bullet{•}$ & $\bullet{•}$ \\ \hline
		Support for abstraction & ~ & $\bullet{•}$ & $\bullet{•}$ \\ \hline
		Expressivity & ~ & $\bullet{•}$ & $\bullet{•}$ \\ \hline
		Type checking & ~ & ~ & $\bullet{•}$ \\ \hline
		Exception handling & ~ & ~ & $\bullet{•}$ \\ \hline
		Restricted aliasing & ~ & ~ & $\bullet{•}$ \\ \hline
	\end{tabular}
	\caption{Overview of trade-offs \citep{sebesta}.}
	\label{tab:langTradeOffs}
\end{table}

In table \ref{tab:langTradeOffs} $\bullet$ means that the characteristic affects the feature of the programming language where the $\bullet$ is placed. If there is no $\bullet$ in front of a feature, it means that this particular characteristic is not affected by the feature. 

Based on these trade-offs, it is clear that having a simple programming language affects both readability, write-ability and reliability. 
This is because having a simple-to-understand language, might not make it very write-able, since code that are simple-to-understand could take longer to write. 
On the other hand, having a simple-to-write programming language, might not make it very readable. An example of this is the if-statement in C, which can be written both with the 'if'-keyword, or more compact. This can be seen by comparing listing \ref{lst:ifstmtnormal} with listing \ref{lst:ifstmtcompact}, which both yield the same result. It is then clear, that the compact if-statement might be faster to write, but slower to read and understand, and opposite with the if-statement.

\begin{code}{ifstmtnormal}{Simple example of if-statement in C using the 'if'-keyword.}
	\begin{lstlisting}
		if (x > y)
		{
    		res = 1;
		}
		else
		{
    		res = 0;
		}
	\end{lstlisting}
\end{code}

\begin{code}{ifstmtcompact}{Simple example of if-statement in C without using the 'if'-keyword.}
	\begin{lstlisting}
		res = x > y ? 1 : 0;
	\end{lstlisting}
\end{code}

When defining the syntax of a programming language, it should balance these characteristics to achieve the right amount of trade-offs for that particular language. For the language of this project, it is important that the language is simple to read and understand, because the target group is the hobbyist-programmer, who might not have much experience in programming.

% nyt fra her
\begin{table}[H]
	\begin{tabular}{l|c|}
\textbf{Characteristic} 		& \rotatebox{90}{Chosen focus for this project} \\ \hline
		Simplicity 				& high \\ \hline
		Orthogonality 			& low \\ \hline
		Data types 				& low \\ \hline
		Syntax design 			& high \\ \hline
		Support for abstraction 		& medium \\ \hline
		Expressivity 				& low \\ \hline
		Type checking 			& high \\ \hline
		Exception handling 			& low \\ \hline
		Restricted aliasing 			& low \\ \hline
	\end{tabular}
	\caption{Overview of choice of focus.}
	\label{tab:choiceoffocus}
\end{table}
Table \ref{tab:choiceoffocus} shows what paradigms that is in the focus for the project. 
\begin{itemize}
%alle punkter skal skrives om og skrives rigtigt - tror nogle ting er misforstået.
\item \textbf{Simplicity:} The project languages do not have many ways the same simple function can be done in code. This is to make writing code simpler for beginners.  e.g. is it not possible to increase a variable by 1 by writing "A++", instead it have to be written as "A=A+1".

\item \textbf{Orthogonality:} The project languages have a only have 3 primitive types, double, int and char, and 2 special types, container and string. it is not possible to make classes or constructs, this is done to help types handling simple for beginners.

\item \textbf{Data types:} \fxfatal{regler ikke lavet inu, eks. kan en bool skrives som bool A = true og/eller bool A = 1}

\item \textbf{Syntax design:} It is not possible to make classes in project languages but there is both "for" and "while" loops because they are relatively simple to use and understand for beginners. All functions and commands start with "begin" and end with "end" to better get understanding where they start and end.

\item \textbf{Support for abstraction:} This is close to non-existent in the project languages, because to help beginners with using the language, a clear types is better to understand.

\item \textbf{Expressivity:} \fxfatal{ikke helt sikker på hvad vi har bestemt os for}

\item \textbf{Type checking:} Is a major part of the project. The type checking will be done on compile time rather then run time, to help beginners see their errors in their code, instead of running time.

\item \textbf{Exception handling:} \fxfatal{ikke helt sikker på hvad vi har bestemt os for}

\item \textbf{Restricted aliasing:} Since at it is know that using aliasing is a dangerous feature in a programming language, the project languages will not have this function, to help beginners.
\end{itemize}
% til her
