\section{Overview of the Compiler}

\figur{0.8}{OverviewCompiler.PNG}{This is an overview of how the compiler is structured.}{fig:OverviewCompiler}

A compiler consists of 3 different phases. The different phases roughly correspond to the different parts in a language specification. The syntax analysis correspond to the syntax, the contextual analysis to the contextual constrains and the code generation correspond to the semantics.


Phases of a Simple Compiler
• Scanner: source ac program -> tokens
– Chap. 3
• Parser: tokens -> abstract syntax tree (AST)
– Chap. 5 & 6
• Symbol table: created from AST
– Chap. 8
• Semantic analysis: AST decoration
– Chap. 9
• Translation
– Chap. 11 and Chap 13.

Scanner and parser under the syntax analysis

