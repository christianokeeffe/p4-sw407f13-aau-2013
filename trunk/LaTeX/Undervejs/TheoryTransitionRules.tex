\documentclass[a4paper,11pt,fleqn,twoside,openright]{memoir} % Brug openright hvis chapters skal starte på højresider; openany, oneside

%%%% PACKAGES %%%%

%  Oversættelse og tegnsætning  %
\usepackage[utf8]{inputenc}					% Gør det muligt at bruge æ, ø og å i sine .tex-filer

\raggedbottom
\usepackage[all]{nowidow}
\usepackage[T1]{fontenc}  % Hjælper med orddeling ved æ, ø og å. Sætter fontene til at være ps-fonte, i stedet for bmp	
\usepackage{syntax}
\usepackage{everyshi}
\makeatletter
\let\totalpages\relax
\newcounter{mypage}
\EveryShipout{\stepcounter{mypage}}
\AtEndDocument{\clearpage
   \immediate\write\@auxout{%
    \string\gdef\string\totalpages{\themypage}}}
\makeatother
\usepackage{longtable}
\usepackage{lscape}
\usepackage[lined,boxed,linesnumbered]{algorithm2e}
\usepackage{latexsym}										% LaTeX symboler
\usepackage{xcolor,ragged2e,fix-cm}			% Justering af elementer
\usepackage{pdfpages} % Gør det muligt at inkludere pdf-dokumenter med kommandoen \includepdf[pages={x-y}]{fil.pdf}	
\usepackage{fixltx2e}					% Retter forskellige bugs i LaTeX-kernen
\usepackage{color}
\definecolor{darkgray}{rgb}{0.95,0.95,0.95}
\usepackage{listings}
\usepackage{tikz}
\usepackage{qtree}


 \lstloadlanguages{% Check Dokumentation for further languages ...
         %[Visual]Basic
         %Pascal
         C
	%[Sharp]C
         %C++
         %XML
         %HTML
        % Java
 }
\lstset{ %
inputencoding=utf8,
literate=%
{æ}{{\ae}}1
{å}{{\aa}}1
{ø}{{\o}}1
{Æ}{{\AE}}1
{Å}{{\AA}}1
{Ø}{{\O}}1,
language=[Sharp]C,                % the language of the code
basicstyle=\footnotesize\ttfamily,       % the size of the fonts that are used for the code
float = H,
xleftmargin = 10pt,
xrightmargin = 10pt,
rulecolor = \color{black},
numbers=left,                   % where to put the line-numbers
numberstyle=\footnotesize,      % the size of the fonts that are used for the line-numbers
stepnumber=1,                   % the step between two line-numbers. If it's 1, each line 
                                % will be numbered
numbersep=5pt,                  % how far the line-numbers are from the code
showspaces=false,               % show spaces adding particular underscores
showstringspaces=false,         % underline spaces within strings
showtabs=false,                 % show tabs within strings adding particular underscores
tabsize=2,                      % sets default tabsize to 2 spaces
captionpos=b,                   % sets the caption-position to bottom
breaklines=true,                % sets automatic line breaking
breakatwhitespace=false,        % sets if automatic breaks should only happen at whitespace
               % show the filename of files included with \lstinputlisting;
                                % also try caption instead of title
escapeinside={\%*}{*)},         % if you want to add a comment within your code
keywordstyle=\color[rgb]{0,0,1},
commentstyle=\color[rgb]{0.133,0.545,0.133},
stringstyle=\color[rgb]{0.627,0.126,0.941},
morekeywords={begin, function, end, nothing, bool, string, OR, AND, using, from, to, step, container, HIGH, LOW}
}

% add frame environment
\usepackage[%
    style=1,
    skipbelow=\topskip,
    skipabove=\topskip
]{mdframed}
\mdfsetup{%
    leftmargin=0pt,
    rightmargin=0pt,
    backgroundcolor=darkgray,
    middlelinecolor=black,
    roundcorner=10
}

% needed for \lstcapt
\def\ifempty#1{\def\temparg{#1}\ifx\temparg\empty}

% make new caption command for listings
\usepackage{caption}
\newcommand{\lstcapt}[2][]{%
    \ifempty{#1}%
        \captionof{lstlisting}{#2}%
    \else%
        \captionof{lstlisting}[#1]{#2}%
    \fi%
    \vspace{0.75\baselineskip}%
}

\usepackage{tabularx}
\usepackage{changepage}

																			
%  Figurer og tabeller floats %
\pdfoptionpdfminorversion=6	% Muliggør inkludering af pdf dokumenter, af version 1.6 og højere
\usepackage{graphicx} 		% Pakke til jpeg/png billeder
\usepackage{rotating}	

%  Matematiske formler og maskinkode 
\usepackage{amsmath,amssymb,stmaryrd} 	% Bedre matematik og ekstra fonte
\usepackage{textcomp}                 	% Adgang til tekstsymboler
\usepackage{mathtools}			% Udvidelse af amsmath-pakken.
\usepackage{siunitx}			% Flot og konsistent præsentation af tal og enheder med \SI{tal}{enhed}

%  Referencer, bibtex og url'er  %
\usepackage{url}	% Til at sætte urler op med. Virker sammen med ref
%\usepackage[danish]{varioref} % Giver flere bedre mulighed for at lave krydshenvisninger
\usepackage[english]{varioref} % Giver flere bedre mulighed for at lave krydshenvisninger
\usepackage{natbib}	% Litteraturliste med forfatter-år og nummerede referencer
\usepackage{xr}		% Referencer til eksternt dokument med \externaldocument{<NAVN>}
\usepackage{nomencl}	% Pakke til at danne nomenklaturliste
\makenomenclature		% Nomenklaturliste

%  Floats  %
\let\newfloat\relax 	% Memoir har allerede defineret denne, men det gør float pakken også
\usepackage{float}
%\usepackage[footnote,draft,danish,silent,nomargin]{fixme}	% Indsæt rettelser og lignende med \fixme{...} Med final i stedet for draft, udløses en error for hver fixme, der ikke er slettet, når rapporten bygges.
\usepackage[draft,silent]{fixme}

%%%% CUSTOM SETTINGS %%%%

%  Marginer  %
\setlrmarginsandblock{3.5cm}{2.5cm}{*}	% \setlrmarginsandblock{Indbinding}{Kant}{Ratio}
\setulmarginsandblock{2.5cm}{3.0cm}{*}	% \setulmarginsandblock{Top}{Bund}{Ratio}
\checkandfixthelayout 

%  Litteraturlisten  %
\bibpunct[,]{[}{]}{;}{a}{,}{,} 	% Definerer de 6 parametre ved Harvard henvisning (bl.a. parantestype og seperatortegn)
\bibliographystyle{bibtex/harvard}	% Udseende af litteraturlisten. Ligner dk-apali - mvh Klein

%  Indholdsfortegnelse  %
\setsecnumdepth{subsubsection}	% Dybden af nummerede overkrifter (part/chapter/section/subsection)
\maxsecnumdepth{subsubsection}	% Ændring af dokumentklassens grænse for nummereringsdybde
\settocdepth{subsection} 		% Dybden af indholdsfortegnelsen


%  Visuelle referencer  %
\usepackage[colorlinks, bookmarksnumbered, bookmarksdepth=4]{hyperref} % Giver mulighed for at ens referencer bliver til klikbare hyperlinks. .. [colorlinks]{..}
%\usepackage{memhfixc}
\hypersetup{pdfborder = 0 0 0}	% Fjerner ramme omkring links i fx indholsfotegnelsen og ved kildehenvisninger 
\hypersetup{			%	Opsætning af farvede hyperlinks
    colorlinks = false,
    linkcolor = black,
    anchorcolor = black,
    citecolor = black
}

\definecolor{gray}{gray}{0.80}					% Definerer farven grå

%  Kapiteludssende  %
\definecolor{numbercolor}{gray}{0.7}			% Definerer en farve til brug til kapiteludseende
\newif\ifchapternonum

\makechapterstyle{jenor}{			% Definerer kapiteludseende -->
  \renewcommand\printchaptername{}
  \renewcommand\printchapternum{}
  \renewcommand\printchapternonum{\chapternonumtrue}
  \renewcommand\chaptitlefont{\fontfamily{pbk}\fontseries{db}\fontshape{n}\fontsize{25}{35}\selectfont\raggedleft}
  \renewcommand\chapnumfont{\fontfamily{pbk}\fontseries{m}\fontshape{n}\fontsize{1in}{0in}\selectfont\color{numbercolor}}
  \renewcommand\printchaptertitle[1]{%
    \noindent
    \ifchapternonum
    \begin{tabularx}{\textwidth}{X}
    {\let\\\newline\chaptitlefont ##1\par} 
    \end{tabularx}
    \par\vskip-2.5mm\hrule
    \else
    \begin{tabularx}{\textwidth}{Xl}
    {\parbox[b]{\linewidth}{\chaptitlefont ##1}} & \raisebox{-15pt}{\chapnumfont \thechapter}
    \end{tabularx}
    \par\vskip2mm\hrule
    \fi
  }
}			% <--

\chapterstyle{jenor}	% Valg af kapiteludseende - dette kan udskiftes efter ønske
\usepackage{wrapfig}


%\renewcommand{\headrulewidth}{0.4pt}
%\renewcommand{\footrulewidth}{0.4pt}

\usepackage{enumitem}
% Sidehoved %

\makepagestyle{custom} % Definerer sidehoved og sidefod - kan modificeres efter ønske -->
\makepsmarks{custom}{																						
\def\chaptermark##1{\markboth{\itshape\thechapter. ##1}{}} % Henter kapitlet den pågældende side hører under med kommandoen \leftmark. \itshape gør teksten kursiv
\def\sectionmark##1{\markright{\thesection. ##1}{}}	% Henter afsnittet den pågældende side hører under med kommandoen \rightmark
} % Sidetallet skrives med kommandoen \thepage	
\makeevenhead{custom}{\leftmark}{P4, Aalborg University}{Group SW407F13} % Definerer lige siders sidehoved efter modellen \makeevenhead{Navn}{Venstre}{Center}{Højre}
\makeoddhead{custom}{Group SW407F13}{P4, Aalborg University}{\leftmark} % Definerer ulige siders sidehoved efter modellen \makeoddhead{Navn}{Venstre}{Center}{Højre}

\usepackage{lastpage}
\usepackage{ifthen}
\usepackage{intcalc}
\usepackage{nth}

\makeevenfoot{custom}{Side \thepage}{}{}	% Definerer lige siders sidefod efter modellen \makeevenfoot{Navn}{Venstre}{Center}{Højre}
\makeoddfoot{custom}{}{}{Side \thepage}% Definerer ulige siders sidefod efter modellen \makeoddfoot{Navn}{Venstre}{Center}{Højre}		
\makeheadrule{custom}{\textwidth}{0.5pt}	 % Tilføjer en streg under sidehovedets indhold
\makefootrule{custom}{\textwidth}{0.5pt}{1mm}	% Tilføjer en streg under sidefodens indhold

\copypagestyle{nychapter}{custom} % Følgende linier sørger for, at sidefoden bibeholdes på kapitlets første side
\makeoddhead{nychapter}{Group SW407F13, Aalborg Universitet}{P4}{\leftmark}
\makeevenhead{nychapter}{Group SW407F13, Aalborg Universitet}{P4}{\leftmark}
\makeheadrule{nychapter}{\textwidth}{0pt}
\aliaspagestyle{chapter}{nychapter}	% <--
\aliaspagestyle{cleared}{custom}

\pagestyle{custom}

%%%% CUSTOM COMMANDS %%%%

% Billede hack %
\newcommand{\figur}[4]{
		\begin{figure}[H] \centering
			\includegraphics[width=#1\textwidth]{billeder/#2}
			\caption{#3}\label{#4}
		\end{figure} 
}

% højrepil %
\newcommand{\ra}[0]{\rightarrow}
% epsilon %
\newcommand{\eps}{\varepsilon}

% Vektor hack %
\newcommand{\vektor}[3]{
			$\begin{pmatrix}
				#1 \\ #2 \\ #3
			\end{pmatrix}$
}

%kode
\newcommand{\kode}[3]{
\noindent
\begin{minipage}{\textwidth}
\begin{mdframed}
		\lstinputlisting{kode/#3}

\end{mdframed}\lstcapt{#1}\label{lst:#2}
\end{minipage}
}

\newenvironment{code}[2]
{\def\fooNoI{#1} \def\fooNoII{#2}\noindent \begin{minipage}{\textwidth}\begin{mdframed}}{\end{mdframed}\lstcapt{\fooNoII}\label{lst:\fooNoI}\end{minipage}}

% quotering
\newcommand{\gaas}[{1}]{``#1''}

% lilleitem
%\newenvironment{noindlist}
 %{\vspace{-5mm}\begin{list}{\labelitemi}{\leftmargin=1em \itemindent=0em }
%\addtolength{\itemsep}{-0.5\baselineskip}}
 %{\end{list}
%\vspace{-20em}}

\newenvironment{noindlist}
 {\vspace{-5mm}\begin{list}{\labelitemi}{\leftmargin=1em \itemindent=0em }
        \setlength{\topsep}{0pt}
        \setlength{\parskip}{0pt}
        \setlength{\partopsep}{0pt}
        \setlength{\parsep}{0pt}         
        \setlength{\itemsep}{0pt} }
 {\end{list}}

\newcommand{\doublesignaturestart}[2]{%
  \parbox{\textwidth}{
    \centering Aalborg \today\\
    \vspace{2cm}

    \parbox{7cm}{
      \centering
      \rule{6cm}{1pt}\\
       #1 
    }
    \hfill
    \parbox{7cm}{
      \centering
      \rule{6cm}{1pt}\\
      #2
    }
  }
}

\newcommand{\longtablesetting}[1]{
\endhead
\multicolumn{#1}{c}{\textit{Continued on the next page}} \\
\endfoot
\endlastfoot
}

\newcommand{\subsubsubsection}[1]{
\textbf{#1}
}

\newcommand{\doublesignature}[2]{%
  \parbox{\textwidth}{
\vspace{2cm}
    \parbox{7cm}{
      \centering
      \rule{6cm}{1pt}\\
       #1 
    }
    \hfill
    \parbox{7cm}{
      \centering
      \rule{6cm}{1pt}\\
      #2
    }
  }
}


%%%% ORDDELING %%%%

\hyphenation{hvad hvem hvor}
\newcolumntype{R}{>{\raggedright\arraybackslash}X}

%----------------SPROG------------------------
%----------------dk
%\usepackage[danish]{babel}							% Dansk sporg, f.eks. tabel, figur og kapitel
%\renewcommand{\algorithmcfname}{Algoritme}
%\renewcommand*{\lstlistingname}{Kodeudsnit}
%----------------en
\usepackage[english]{babel}
\begin{document}
\section{Transition Rules}
In this section some of the transition rules in SPLAD will be explained. The complete list of all the rules can be seen in appendix \ref{sec:transitionrules}.
In the following text we use the following names to represent different syntactic categories.
\fxfatal{bruger vi bigstep eller smallstep? hvad med envP?}
%% Vi bruger Bigstep. Desuden er envP ikke brugt endnu så derfor er den ikke med.
\begin{itemize}
\item $n \in \textbf{Num}$ - Numerals
\item $v$ - Values
\item $x \in \textbf{Var}$ - Variables 
\item $r \in \textbf{Arrays}$ - Array names
\item $a \in A_{exp}$ - Arithmetic expression
\item $b \in B_{exp}$ - Boolean expression
\item $e \in A_{exp} \cup B_{exp}$ - expressions
\item $C \in \textbf{Com}$ - Commands
\end{itemize}

\subsection{Environment-Store Model}
In our project we use the \textit{environment-store model} to represent how a variable is bound to a storage cell (called a \textit{location}), in the computer, and that the value of the variable is the content of the bound location. All the possible locations are denoted by \textbf{Loc} and a single location as $l \in \textbf{Loc}$. We assume all locations are integer \fxfatal{hvorfor det?? det holder jo ikke i forhold til vores sprog? }
%% Det er bare noget vi gør og har intet med vores sprog at gøre. Det handler om hvordan vi gemmer ting og ikke at alt er integers. se afsnit 6.2 i Hans Hüttels bog
, and therefore $\textbf{Loc} = \mathbb{Z}$. Since all locations are integers you can define a function to find the next location: $\textbf{Loc} \rightarrow \textbf{Loc}$, where $l = l + 1$. \fxfatal{Er dette korrekt?} %% Ja se afsnit 6.2 i Hans Hüttels bog

We define the set of stores to be the mappings from locations to values $\textbf{Sto } = \textbf{ Loc } \rightharpoonup \mathbb{Z}$, where $sto$ is an single element in $\textbf{Sto}$.

The following names represent the different environments.
\begin{itemize}
\item $env_V \in Env_V$ - Variable declarations
\item $env_A \in Env_A$ - Array declarations
\item $env_C \in Env_C$ - Constant declarations
\item $env_E \in Env_E$ - Expressions declarations
\end{itemize}

\subsection{Arithmetic Expressions}
The transition rule for multiplication in SPLAD can be seen on table \ref{tab:MultExp}. The rule states, that if $a_1$ evaluates to $v_1$ and $a_2$ evaluates to $v_2$, using any of the rules from $A_{exp}$, then $a_1 * a_2$ evaluates to $v$ where $v = v_1 * v_2$.

\begin{table}[H]
\begin{tabular}{l l}
[MUL] & \[\frac{env_E, \: sto \vdash a_1 \rightarrow_a v_1 \;\; env_E, \: sto \vdash a_2 \rightarrow_a v_2}{env_E, \: sto \vdash a_1 * a_2 \rightarrow_a v}\] \\
~ & ~ \\
~ & \indent\indent where $v=v_1 * v_2$ \\ 
~ & ~ \\
\end{tabular}
\caption{The transition rule for the arithmetic multiplication expression.}
\label{tab:MultExp}
\end{table}

\subsection{Boolean Expressions}
The transition rule for logical-or in SPLAD can be seen on table \ref{tab:OrExp}. The rules have to parts: [OR-TRUE] and [OR-FALSE]. The [OR-TRUE] rule states that either $b_1$ or $b_2$ evaluates to \textit{true}, using any of the rules from $B_{exp}$ then the expression $b_1 OR b_2$ evaluates to \textit{true}. [OR-FALSE] states that if both $b_1$ and $b_2$ evaluates to \textit{false} then the expression $b_1 OR b_2$ evaluates to \textit{false}.

\begin{longtable}{l l}
\longtablesetting{2}
[OR-TRUE] & \[\frac{env_E, \: sto \vdash b_1 \vee b_2 \rightarrow_b \text{true}}{env_E, \: sto \vdash b_1  \; \text{OR} \; b_2 \rightarrow_b \text{true}}\] \\
~ & ~ \\

[OR-FALSE] & \[\frac{env_E, \: sto \vdash b_1 \wedge b_2 \rightarrow_b \text{false}}{env_E, \: sto \vdash b_1  \; \text{OR} \; b_2 \rightarrow_b \text{false}}\] \\
~ & ~ \\
\caption{Transition rule for the boolean expression logical-or.}
\label{tab:OrExp}
\end{longtable}\fxfatal{mangler der ikke noget her? en sidebetingelse?}
% Jo tænkte jeg også... men det er der ikke i de regler der står i vores bilag.. evt skal begge rettes.

\subsection{Declarations}
\fxfatal{Kan ikke laves siden at transition rules til dette punkt ikke er lavet endnu}
%% Ja det er så mig der har skrevet denne fxfatal

\subsection{Assignments}
The transition rule for variable assignment in SPLAD can be seen on table \ref{tab:VarAssign}. When a variable is assigned the contents of $l$ is updated to $v$, where $l$ is the location of $x$ found in the $env_V$ and $v$ is the result of evaluation $e$.

\begin{longtable}{l l}
\longtablesetting{2}
[VAR-ASS] & \[env_C, \: \vdash \langle x=e, \; sto \rangle \rightarrow sto[l \mapsto v]\] \\
~ & ~ \\
~ & \indent\indent where $env_C, \; sto \vdash e \rightarrow_e v$ \\
~ & \indent\indent and $env_V \; x=l$ \\
~ & ~ \\
\caption{Transition rule for variable assignment.}
\label{tab:VarAssign}
\end{longtable}
\fxfatal{Hedder det ikke envC, envV i første udtryk i reglen?}
%% Jo tror jeg du har ret i. Se Tabel 6.1 i Hans Hüttels bog og være sikker på at gemme det i bilaget også. 

\subsection{Commands}
The transition rule for the while statement in SPLAD can be seen on table \ref{tab:WhileStatement}. The rule have to parts: [WHL-TRUE] and [WHL-FALSE]. If the condition $b$ evaluates to \textit{true} then the [WHL-TRUE] states that $C$ will be executed which will update the \textit{store} (sto) and again call the expression and evaluate the new $b$. If the condition $b$ evaluates to \textit{false} then $C$ is \underline{not} executed and the \textit{store} is not updated. The program exits the while statement. \fxfatal{Hans Hüttels danske bog}

\begin{longtable}{l l}
\longtablesetting{2}
[WHL-TRUE] & \[\frac{env_C \: \vdash \langle C, \: sto \rangle \rightarrow sto'' \; env_C \: \vdash \langle \text{\textbf{while}}(b)\;\text{begin}\;C\; \text{end}, \: sto'' \rangle \rightarrow sto'}{env_C \: \vdash \: \langle \text{\textbf{while}}(b) \: \text{begin}\;C\;\text{end}, \: sto \rangle \rightarrow sto' }\] \\
~ & ~ \\
~ & \indent\indent if $env_C, \; sto \vdash b \rightarrow_b \text{true}$ \\
~ & ~ \\

[WHL-FALSE] & \[env_C \: \vdash \langle \text{\textbf{while}}(b) \: \text{begin} \: C \: \text{end}, \: sto \rangle \rightarrow sto\] \\
~ & ~ \\
~ & \indent\indent if $env_C, \; sto \vdash b \rightarrow_b \text{false}$ \\
~ & ~ \\
\caption{Transition rules for the while statement.}
\label{tab:WhileStatement}
\end{longtable}
\end{document}