\section{Code Generation}
The idea by using a high-level language is that it should be easier and faster to write programs. But by using a high-level language or any other kind of language, a compiler is needed to produce object code that, if the code is without errors, should result in a running program. This section will be about how the code generation is implemented in this project and will also describe the choices that have affected the code generator.

The code produced by the code generator will need to run on an Arduino platform before it satisfies the project formulation, hence that the target code should be runnable on the targeted platform. Arduino's IDE make some transformation to the code to secure that it is correct c/c++ code. After this it calls the avr-gcc which complies c/c++ code to object files, which are then uploaded to the Arduino unit using the AVRDUDE \fxfatal{http://arduino.cc/en/Hacking/BuildProcess}. AVRDUDE is a tool for uploading to an AVR micro-controller \fxfatal{http://www.nongnu.org/avrdude/}.

For our code generator it would be the must correct way to implement these function into the compiler itself, but because of other more critical task at hard these features have been substituted by using the Arduino IDE to compile the c/c++ code and uploading this code to the Arduino board. Therefore the target code for our compiler should be the Arduino c/c++ code which we then uses the Arduino IDE for further compilation and uploading.