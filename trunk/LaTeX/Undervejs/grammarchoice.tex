
\section{Choice of grammar}
It was decided that the grammar for this project should have a high level of readability. This was decided because the programmer could be a hobby programmer, who want to make a drink machine, but does not have a high level of education in programming. A high readability will ensure that this person easily can read and understand their program - also if they have to edit it later on. The method to assign a value to a variable is by typing "$variable$ <-- $value to assign$". By using the arrow, it is clearly indicated that the value is assign to the variable, and therefore ensuring readability - especially for the hobby programmer.
The functions must always return something, but it can return the value "nothing". This will ensure the understanding and readability, when the programmer can see that it returns, but no value i parsed. To indicate that $return$ is the last thing which will be executed in a function, the $return$ must always be at the end the function. To indicate that a program is called there must be written "call $functionname$".
There are used words instead of symbols (compared to most other programming languages) were suitable to improve the understading of the program. To indicate a block (eg. a if statement) are there used "begin" and "end". To combine logical operators the words "AND" and "OR". To end a line ";" is used, also to improve readability.