\section{Known lexers and parsers}
In this section some of the different lexers and parsers that are available on the internet will be descriped.

\subsection{Lexers}
These programs generate lexical analysers also known as scanners, that turns code into tokens which a parser uses.

\subsubsection{Jflex}
Jflex is based on Flex that focuses on speed and full Unicode support. It can be used as a standalone tool or together with  the LALR parser generators Cup and BYacc/J

\subsubsection{Lex}
Files are divided into three sections separated by lines containing two percent signs. The first is the "definition section" this is where macros can be defined and where headerfiles are imported. The second is the "Rules section" where regular expressions are read in terms of C statements. The third is the "C code section" which contains C statements and functions that are copied verbatim to the generated source file.

\subsubsection{Jlex}
Based on lex put used for java.

\subsection{Parsers}


\subsubsection{Yacc}


\subsubsection{Java cup}
More or less like Yacc, just for java.

\subsection{Lexers and parsers}


\subsubsection{Sablecc}


\subsubsection{ANTLR}


\subsubsection{Javacc}
Javacc generate a parser from a formal grammar written in EBNF notation the output is Java source code. JavaCC generates top-down parsers, which limits it to the LL(k) class of grammars (in particular, left recursion cannot be used). JavaCC also generates lexical analyzers in a fashion similar to lex. The tree builder that accompanies it, JJTree, constructs its trees from the bottom up.




handmade