\documentclass[a4paper,11pt,fleqn,twoside,openright]{memoir} % Brug openright hvis chapters skal starte på højresider; openany, oneside

%%%% PACKAGES %%%%

%  Oversættelse og tegnsætning  %
\usepackage[utf8]{inputenc}					% Gør det muligt at bruge æ, ø og å i sine .tex-filer

\raggedbottom
\usepackage[all]{nowidow}
\usepackage[T1]{fontenc}  % Hjælper med orddeling ved æ, ø og å. Sætter fontene til at være ps-fonte, i stedet for bmp	
\usepackage{syntax}
\usepackage{everyshi}
\makeatletter
\let\totalpages\relax
\newcounter{mypage}
\EveryShipout{\stepcounter{mypage}}
\AtEndDocument{\clearpage
   \immediate\write\@auxout{%
    \string\gdef\string\totalpages{\themypage}}}
\makeatother
\usepackage{longtable}
\usepackage{lscape}
\usepackage[lined,boxed,linesnumbered]{algorithm2e}
\usepackage{latexsym}										% LaTeX symboler
\usepackage{xcolor,ragged2e,fix-cm}			% Justering af elementer
\usepackage{pdfpages} % Gør det muligt at inkludere pdf-dokumenter med kommandoen \includepdf[pages={x-y}]{fil.pdf}	
\usepackage{fixltx2e}					% Retter forskellige bugs i LaTeX-kernen
\usepackage{color}
\definecolor{darkgray}{rgb}{0.95,0.95,0.95}
\usepackage{listings}
\usepackage{tikz}
\usepackage{qtree}


 \lstloadlanguages{% Check Dokumentation for further languages ...
         %[Visual]Basic
         %Pascal
         C
	%[Sharp]C
         %C++
         %XML
         %HTML
        % Java
 }
\lstset{ %
inputencoding=utf8,
literate=%
{æ}{{\ae}}1
{å}{{\aa}}1
{ø}{{\o}}1
{Æ}{{\AE}}1
{Å}{{\AA}}1
{Ø}{{\O}}1,
language=[Sharp]C,                % the language of the code
basicstyle=\footnotesize\ttfamily,       % the size of the fonts that are used for the code
float = H,
xleftmargin = 10pt,
xrightmargin = 10pt,
rulecolor = \color{black},
numbers=left,                   % where to put the line-numbers
numberstyle=\footnotesize,      % the size of the fonts that are used for the line-numbers
stepnumber=1,                   % the step between two line-numbers. If it's 1, each line 
                                % will be numbered
numbersep=5pt,                  % how far the line-numbers are from the code
showspaces=false,               % show spaces adding particular underscores
showstringspaces=false,         % underline spaces within strings
showtabs=false,                 % show tabs within strings adding particular underscores
tabsize=2,                      % sets default tabsize to 2 spaces
captionpos=b,                   % sets the caption-position to bottom
breaklines=true,                % sets automatic line breaking
breakatwhitespace=false,        % sets if automatic breaks should only happen at whitespace
               % show the filename of files included with \lstinputlisting;
                                % also try caption instead of title
escapeinside={\%*}{*)},         % if you want to add a comment within your code
keywordstyle=\color[rgb]{0,0,1},
commentstyle=\color[rgb]{0.133,0.545,0.133},
stringstyle=\color[rgb]{0.627,0.126,0.941},
morekeywords={begin, function, end, nothing, bool, string, OR, AND, using, from, to, step, container, HIGH, LOW}
}

% add frame environment
\usepackage[%
    style=1,
    skipbelow=\topskip,
    skipabove=\topskip
]{mdframed}
\mdfsetup{%
    leftmargin=0pt,
    rightmargin=0pt,
    backgroundcolor=darkgray,
    middlelinecolor=black,
    roundcorner=10
}

% needed for \lstcapt
\def\ifempty#1{\def\temparg{#1}\ifx\temparg\empty}

% make new caption command for listings
\usepackage{caption}
\newcommand{\lstcapt}[2][]{%
    \ifempty{#1}%
        \captionof{lstlisting}{#2}%
    \else%
        \captionof{lstlisting}[#1]{#2}%
    \fi%
    \vspace{0.75\baselineskip}%
}

\usepackage{tabularx}
\usepackage{changepage}

																			
%  Figurer og tabeller floats %
\pdfoptionpdfminorversion=6	% Muliggør inkludering af pdf dokumenter, af version 1.6 og højere
\usepackage{graphicx} 		% Pakke til jpeg/png billeder
\usepackage{rotating}	

%  Matematiske formler og maskinkode 
\usepackage{amsmath,amssymb,stmaryrd} 	% Bedre matematik og ekstra fonte
\usepackage{textcomp}                 	% Adgang til tekstsymboler
\usepackage{mathtools}			% Udvidelse af amsmath-pakken.
\usepackage{siunitx}			% Flot og konsistent præsentation af tal og enheder med \SI{tal}{enhed}

%  Referencer, bibtex og url'er  %
\usepackage{url}	% Til at sætte urler op med. Virker sammen med ref
%\usepackage[danish]{varioref} % Giver flere bedre mulighed for at lave krydshenvisninger
\usepackage[english]{varioref} % Giver flere bedre mulighed for at lave krydshenvisninger
\usepackage{natbib}	% Litteraturliste med forfatter-år og nummerede referencer
\usepackage{xr}		% Referencer til eksternt dokument med \externaldocument{<NAVN>}
\usepackage{nomencl}	% Pakke til at danne nomenklaturliste
\makenomenclature		% Nomenklaturliste

%  Floats  %
\let\newfloat\relax 	% Memoir har allerede defineret denne, men det gør float pakken også
\usepackage{float}
%\usepackage[footnote,draft,danish,silent,nomargin]{fixme}	% Indsæt rettelser og lignende med \fixme{...} Med final i stedet for draft, udløses en error for hver fixme, der ikke er slettet, når rapporten bygges.
\usepackage[draft,silent]{fixme}

%%%% CUSTOM SETTINGS %%%%

%  Marginer  %
\setlrmarginsandblock{3.5cm}{2.5cm}{*}	% \setlrmarginsandblock{Indbinding}{Kant}{Ratio}
\setulmarginsandblock{2.5cm}{3.0cm}{*}	% \setulmarginsandblock{Top}{Bund}{Ratio}
\checkandfixthelayout 

%  Litteraturlisten  %
\bibpunct[,]{[}{]}{;}{a}{,}{,} 	% Definerer de 6 parametre ved Harvard henvisning (bl.a. parantestype og seperatortegn)
\bibliographystyle{bibtex/harvard}	% Udseende af litteraturlisten. Ligner dk-apali - mvh Klein

%  Indholdsfortegnelse  %
\setsecnumdepth{subsubsection}	% Dybden af nummerede overkrifter (part/chapter/section/subsection)
\maxsecnumdepth{subsubsection}	% Ændring af dokumentklassens grænse for nummereringsdybde
\settocdepth{subsection} 		% Dybden af indholdsfortegnelsen


%  Visuelle referencer  %
\usepackage[colorlinks, bookmarksnumbered, bookmarksdepth=4]{hyperref} % Giver mulighed for at ens referencer bliver til klikbare hyperlinks. .. [colorlinks]{..}
%\usepackage{memhfixc}
\hypersetup{pdfborder = 0 0 0}	% Fjerner ramme omkring links i fx indholsfotegnelsen og ved kildehenvisninger 
\hypersetup{			%	Opsætning af farvede hyperlinks
    colorlinks = false,
    linkcolor = black,
    anchorcolor = black,
    citecolor = black
}

\definecolor{gray}{gray}{0.80}					% Definerer farven grå

%  Kapiteludssende  %
\definecolor{numbercolor}{gray}{0.7}			% Definerer en farve til brug til kapiteludseende
\newif\ifchapternonum

\makechapterstyle{jenor}{			% Definerer kapiteludseende -->
  \renewcommand\printchaptername{}
  \renewcommand\printchapternum{}
  \renewcommand\printchapternonum{\chapternonumtrue}
  \renewcommand\chaptitlefont{\fontfamily{pbk}\fontseries{db}\fontshape{n}\fontsize{25}{35}\selectfont\raggedleft}
  \renewcommand\chapnumfont{\fontfamily{pbk}\fontseries{m}\fontshape{n}\fontsize{1in}{0in}\selectfont\color{numbercolor}}
  \renewcommand\printchaptertitle[1]{%
    \noindent
    \ifchapternonum
    \begin{tabularx}{\textwidth}{X}
    {\let\\\newline\chaptitlefont ##1\par} 
    \end{tabularx}
    \par\vskip-2.5mm\hrule
    \else
    \begin{tabularx}{\textwidth}{Xl}
    {\parbox[b]{\linewidth}{\chaptitlefont ##1}} & \raisebox{-15pt}{\chapnumfont \thechapter}
    \end{tabularx}
    \par\vskip2mm\hrule
    \fi
  }
}			% <--

\chapterstyle{jenor}	% Valg af kapiteludseende - dette kan udskiftes efter ønske
\usepackage{wrapfig}


%\renewcommand{\headrulewidth}{0.4pt}
%\renewcommand{\footrulewidth}{0.4pt}

\usepackage{enumitem}
% Sidehoved %

\makepagestyle{custom} % Definerer sidehoved og sidefod - kan modificeres efter ønske -->
\makepsmarks{custom}{																						
\def\chaptermark##1{\markboth{\itshape\thechapter. ##1}{}} % Henter kapitlet den pågældende side hører under med kommandoen \leftmark. \itshape gør teksten kursiv
\def\sectionmark##1{\markright{\thesection. ##1}{}}	% Henter afsnittet den pågældende side hører under med kommandoen \rightmark
} % Sidetallet skrives med kommandoen \thepage	
\makeevenhead{custom}{\leftmark}{P4, Aalborg University}{Group SW407F13} % Definerer lige siders sidehoved efter modellen \makeevenhead{Navn}{Venstre}{Center}{Højre}
\makeoddhead{custom}{Group SW407F13}{P4, Aalborg University}{\leftmark} % Definerer ulige siders sidehoved efter modellen \makeoddhead{Navn}{Venstre}{Center}{Højre}

\usepackage{lastpage}
\usepackage{ifthen}
\usepackage{intcalc}
\usepackage{nth}

\makeevenfoot{custom}{Side \thepage}{}{}	% Definerer lige siders sidefod efter modellen \makeevenfoot{Navn}{Venstre}{Center}{Højre}
\makeoddfoot{custom}{}{}{Side \thepage}% Definerer ulige siders sidefod efter modellen \makeoddfoot{Navn}{Venstre}{Center}{Højre}		
\makeheadrule{custom}{\textwidth}{0.5pt}	 % Tilføjer en streg under sidehovedets indhold
\makefootrule{custom}{\textwidth}{0.5pt}{1mm}	% Tilføjer en streg under sidefodens indhold

\copypagestyle{nychapter}{custom} % Følgende linier sørger for, at sidefoden bibeholdes på kapitlets første side
\makeoddhead{nychapter}{Group SW407F13, Aalborg Universitet}{P4}{\leftmark}
\makeevenhead{nychapter}{Group SW407F13, Aalborg Universitet}{P4}{\leftmark}
\makeheadrule{nychapter}{\textwidth}{0pt}
\aliaspagestyle{chapter}{nychapter}	% <--
\aliaspagestyle{cleared}{custom}

\pagestyle{custom}

%%%% CUSTOM COMMANDS %%%%

% Billede hack %
\newcommand{\figur}[4]{
		\begin{figure}[H] \centering
			\includegraphics[width=#1\textwidth]{billeder/#2}
			\caption{#3}\label{#4}
		\end{figure} 
}

% højrepil %
\newcommand{\ra}[0]{\rightarrow}
% epsilon %
\newcommand{\eps}{\varepsilon}

% Vektor hack %
\newcommand{\vektor}[3]{
			$\begin{pmatrix}
				#1 \\ #2 \\ #3
			\end{pmatrix}$
}

%kode
\newcommand{\kode}[3]{
\noindent
\begin{minipage}{\textwidth}
\begin{mdframed}
		\lstinputlisting{kode/#3}

\end{mdframed}\lstcapt{#1}\label{lst:#2}
\end{minipage}
}

\newenvironment{code}[2]
{\def\fooNoI{#1} \def\fooNoII{#2}\noindent \begin{minipage}{\textwidth}\begin{mdframed}}{\end{mdframed}\lstcapt{\fooNoII}\label{lst:\fooNoI}\end{minipage}}

% quotering
\newcommand{\gaas}[{1}]{``#1''}

% lilleitem
%\newenvironment{noindlist}
 %{\vspace{-5mm}\begin{list}{\labelitemi}{\leftmargin=1em \itemindent=0em }
%\addtolength{\itemsep}{-0.5\baselineskip}}
 %{\end{list}
%\vspace{-20em}}

\newenvironment{noindlist}
 {\vspace{-5mm}\begin{list}{\labelitemi}{\leftmargin=1em \itemindent=0em }
        \setlength{\topsep}{0pt}
        \setlength{\parskip}{0pt}
        \setlength{\partopsep}{0pt}
        \setlength{\parsep}{0pt}         
        \setlength{\itemsep}{0pt} }
 {\end{list}}

\newcommand{\doublesignaturestart}[2]{%
  \parbox{\textwidth}{
    \centering Aalborg \today\\
    \vspace{2cm}

    \parbox{7cm}{
      \centering
      \rule{6cm}{1pt}\\
       #1 
    }
    \hfill
    \parbox{7cm}{
      \centering
      \rule{6cm}{1pt}\\
      #2
    }
  }
}

\newcommand{\longtablesetting}[1]{
\endhead
\multicolumn{#1}{c}{\textit{Continued on the next page}} \\
\endfoot
\endlastfoot
}

\newcommand{\subsubsubsection}[1]{
\textbf{#1}
}

\newcommand{\doublesignature}[2]{%
  \parbox{\textwidth}{
\vspace{2cm}
    \parbox{7cm}{
      \centering
      \rule{6cm}{1pt}\\
       #1 
    }
    \hfill
    \parbox{7cm}{
      \centering
      \rule{6cm}{1pt}\\
      #2
    }
  }
}


%%%% ORDDELING %%%%

\hyphenation{hvad hvem hvor}
\newcolumntype{R}{>{\raggedright\arraybackslash}X}

%----------------SPROG------------------------
%----------------dk
%\usepackage[danish]{babel}							% Dansk sporg, f.eks. tabel, figur og kapitel
%\renewcommand{\algorithmcfname}{Algoritme}
%\renewcommand*{\lstlistingname}{Kodeudsnit}
%----------------en
\usepackage[english]{babel}
\begin{document}
\section{Specification of the language to the purpose}
To specify the language, so it will make the program as suitable as possible for writing drink machines, we looked at what the central aspects of a drink machine is:
\begin{inddes}
\item[A drink:] A drink is central to this machine. A drink should be the product made by the machine, defined by a number of ingredients. A drink should be like a recipe.
\item[An ingredient:] An ingredient is the elements of a drink. It will in the machine be contained in a container. 
\item[A RFID-tag:] A RFID-tag with a drink ID and an amount of how many drinks there are left.
\item[A RFID-RW:] A RFID-reader and writer for Arduino, to write and read the content of a RFID-tag.
\item[A LCD:] For communicating with the user, a LCD is in most situations preferable.
\item[Buttons:] For getting input from users, buttons are a possibility. 
\item[Mechanism for pouring ingredients:] A mechanism for pouring the right amount of ingredients into the drink.
\end{inddes}
These are by our assessment the most central aspects of the drink machine system. We will now make a judgment of each of the listed aspects, and see if it is possible to make a structure in the language which will support the programmer on in any other way make it easier implement this aspect in the system.
\subsection{A drink}
The concept of a drink is one of the most central aspects of a drink machine system. A drink should contain a recipe as a list of ingredients, and how much of the ingredients to pour. A drink do also have a name, and should have a form for ID which can be stored on a RFID-tag. We should make a structure which can implement a drink type and assign the recipe to the drink. The structure should be following the same design criterias as the rest of the language, and should be inspirited by the other structures in the language. To fulfill these requirements, the declaration of a drink should have a block with "begin" and "end", and in the block have the "recipe". The result can be seen on listing \ref{lst:drink1}.

\begin{code}{drink1}{The first example of a declaration of a drink}
\begin{lstlisting}
drink [drinkname] is
begin
	[recipe]
end
\end{lstlisting}
\end{code}

We have now decided how to define a element of the type drink. We must now look at the body of the block in the declaration. To make as readable as possible, it should be written a little like a normal recipe. Because of that, we have decided to state it in the form "add [number] of [ingredient]", where [number] is a number representing the amount of the ingredient to add. The declaration of a drink will now be defined as seen on listing \ref{lst:drink2}.

\begin{code}{drink2}{The final structure of how to declare a drink}
\begin{lstlisting}[mathescape]
drink [drinkname] is
begin
	add [number] of [ingredient];
	add [number] of [ingredient];
	$\vdots$
	add [number] of [ingredient];
end
\end{lstlisting}
\end{code}

In some situations, a drink could be very much alike from an other drink, with only a few changes. For an example could it be a drink with a double shot of alcohol. In this case, it would be the exactly same drink, but with more of one ingredient. It could also be an ingredient that should be remove, for example alcohol to make it non-alcohol ore it someone is allergic to some of the elements in a drink. Because of these situations and many more, it could be preferable to have a way to inherit the recipe from an other drink, and then modify it. The declaration of a drink which inherits from an other drink should be very much alike the normal drinkdeclaration as seen on listing \ref{lst:drink2}, but also with so significant modifications, so it is easy to see that this drink inherits from an other drink. The block structure should be the same, and the way of add an amount of an ingredient should be the same. We have to add a now command to the block statement: "remove". With the remove statement, it should be possible to completely remove an ingredient from a drink. It should be easy to read, and for that we have decided that the declaration statement before the block should be "drink [drinkname] as [drinkname of drink to inherit] but" to say that the drink is as the other drink, but with changes. The final structure will therefore be as seen on listing \ref{lst:drinkinherit}.

\begin{code}{drinkinherit}{The structure of how to declare a drink which inherits the recipe from an other drink}
\begin{lstlisting}[mathescape]
drink [drinkname] as [drinkname of drink to inherit] but
begin
	add [number] of [ingredient];
	remove [ingredient];
	$\vdots$
	add [number] of [ingredient];
end
\end{lstlisting}
\end{code}

We have now defined how the structure of the type drink should be. To formally define the syntax, we use BNF which can be seen on grammar \ref{gra:drinkgram}.

\begin{grammatik}{drinkgram}{The grammar for the drink declaration}
<drinkdcl> $\ra$ drink <id> is begin <drinkstmts> end
\alt drink <id> as <id> but begin <changedrinkstmts> end

<drinkstmts> $\ra$ <drinkstmt> <drinkstmtsend>

<drinkstmt> $\ra$ add <numeric> of <id>

<drinkstmtsend> $\ra$ ; <drinkstmts>
\alt ;

<changedrinkstmts> $\ra$ <changedrinkstmt> <changedrinkstmtsend>

<changedrinkstmt> $\ra$ <drinkstmt>
\alt remove <id>

<changedrinkstmtsend> $\ra$ ; <changedrinkstmts>
\alt ;
\end{grammar}

\subsection{An ingredient}
With the drink type defined above, we should now focus on, how to define an ingredient. 

\end{document}