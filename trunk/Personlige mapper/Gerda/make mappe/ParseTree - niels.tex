\section{Parse Tree}
Parse tree are nearly the same as a abstract syntax tree, but here all internal nodes are labelled with a non-terminal symbol and all leaf labelled with a terminal symbol. Also a sub-tree describe one instance abstraction of a sentence.

To help understand the different between abstract syntax tree and parse tree. 2 trees are made, one of each, over the same code for a variable declaration in the projects language. Under this text there is a abstract syntax tree for the same code \ref{VariableDeclarationEKS}. \ref{ParseTreeEKS} is a parse tree over the code \ref{VariableDeclarationEKS}.

\begin{code}{VariableDeclarationEKS}{A simple variable declaration in the project language.}
	\begin{lstlisting}
		int x <-- 3+2;
	\end{lstlisting}
\end{code}

\Tree[.program [.<-- [.x
]
                    [.+ [.3
]
                        [.2
                    ]]]]

\figur{0.7}{parsetree.png}{A parse tree that is made by using ANTLR.}{fig:ParseTreeEKS}